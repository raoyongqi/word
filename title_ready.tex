\documentclass{article}
\usepackage{hyperref}
\usepackage{graphicx}
\usepackage{amsmath}
\usepackage{geometry}
\geometry{a4paper, margin=1in}
\usepackage{ctex}  % 处理中文支持
\usepackage{booktabs}
\usepackage{listings}
\usepackage{comment}
\usepackage{titlesec}
\usepackage{float} 
% 设置标题编号深度
\setcounter{secnumdepth}{4}  % 允许最多四级标题

% 定义新的标题命令 \subsubsubsection
\newcounter{subsubsubsection}[subsubsection]  % 按照 subsubsection 编号
\renewcommand{\thesubsubsubsection}{\thesubsubsection.\arabic{subsubsubsection}}  % 设置编号格式

\newcommand{\subsubsubsection}[1]{%
	\par
	\refstepcounter{subsubsubsection} % 增加计数器
	\noindent
	\textbf{\thesubsubsubsection\quad #1} % 打印编号和标题
	\vspace{0.5em} % 调整标题与正文的间距
}




\title{基于机器学习的植物病害分析与预测模型构建}
\author{}
\date{}

\begin{document}
	
	\maketitle
	
	\tableofcontents
	
	\section{第一章 前言}
	\subsection{研究背景}
	\subsubsection{植物病害概述}
植物在生长过程中受到多种病原体的侵害,其中病原真菌和卵菌是主要的病原体,严重影响植物的生长和发育。相关研究表明,锈菌、白粉菌以及卵菌中的疫霉菌和霜霉菌是导致植物病害的主要原因,造成植物形态异常、功能受损和生理受限,进而引发一系列植物病害。这些病害不仅影响植物的生长,还对农业生产造成显著威胁,导致农作物减产和品质下降。

研究者对锈病、白粉病和叶斑病等植物病害进行了深入的调查与分析。锈病通常表现为植物叶片和茎秆上出现小斑点,随着病情加重,可能导致叶片脱落和植株枯死。相关研究发现,锈病的发生与环境湿度、温度及病原菌的传播密切相关。在高湿环境下,锈病病原体更易繁殖,导致病害的迅速扩散。

白粉病则主要表现为植物表面覆盖一层白色粉状物,严重影响植物的光合作用,进而影响其生长。研究者通过观察发现,白粉病的病原菌在温暖、干燥的环境中更容易传播,导致大规模的植物感染。对该病害的控制措施主要包括改善栽培管理和应用防治药剂,以降低病原菌的侵染。

叶斑病的特征是叶片上出现各种颜色的斑点,随着病情的加重,斑点逐渐扩散,最终导致叶片的枯萎。研究表明,叶斑病的病原体在不同植物种类中存在差异,这使得防治措施需根据具体病原体进行调整。相关文献指出,采取适当的轮作、施肥和病害监测措施,可以有效减轻叶斑病的危害。

Jones等(2022)通过转录组测序揭示了某些植物病原真菌的致病机制,提供了新的靶点用于抗病性品种的育种\cite{Jones2022}。此外,Zhang等(2023)研究了新型植物病毒的基因组特征,阐明了其在植物中的传播机制,为植物病毒病害的监测和防控提供了理论基础\cite{Zhang2023}。

植物的免疫机制是植物病害研究的另一个重要领域。研究发现,植物通过感知病原体的特征,激活自身的免疫反应,从而抵御病害的侵袭。Duan等(2022)通过基因编辑技术,揭示了植物中关键免疫受体的功能,推动了植物抗病性研究的进展\cite{Duan2022}。此外,Li等(2023)研究了植物激素在免疫反应中的作用,指出一些植物激素不仅可以激活免疫反应,还可以调节植物的生长发育,促进植物的抗病能力\cite{Li2023}。

在病害管理策略方面,科学家们正致力于开发新型的病害防治方法。Wang等(2024)提出了一种结合生物防治与化学防治的新策略,通过引入拮抗微生物与植物保护剂的联用,提高了病害防治的效果\cite{Wang2024}。此外,智能农业技术的应用也为病害监测与管理提供了新机遇。Chen等(2024)研究了基于物联网的植物病害监测系统,通过实时数据分析与处理,能够快速识别病害并采取相应措施\cite{Chen2024}。

最后,新型抗病材料的开发也在植物病害防治中展现出广阔前景。研究者们探索了天然提取物、纳米材料及生物基材料在植物抗病性提升中的应用。Liu等(2023)研究表明,某些植物提取物具有显著的抗病作用,可以增强植物的免疫反应,从而提高植物对病害的抵抗能力\cite{Liu2023}。
	
	\subsubsection{ 死体病原菌与活体病原菌}
病原物大体分为两类:一类病原物杀死寄主,然后从上面获得营养物质,即所谓的死体营养寄生物;另一类是需要获得寄主以完成它们的生活史,即活体营养寄生物。活体病原菌的一个短暂阶段代表了半活体营养病原菌。这类真菌在开始转向杀死寄主之前具有一个活体营养生长阶段。
Fitzpatrick 和 Stajich (2015) 讨论了真菌病原体的比较基因组学,强调宿主与病原体之间的相互作用以及致病机制的演变,为理解病原体如何适应宿主提供了重要视角\cite{Fitzpatrick2015}。Huang 和 Wang (2018) 通过比较基因组学分析病原性真菌的进化,探讨了不同病原体如何适应宿主环境以完成生活史\cite{Huang2018}。Pappas 和 Kauffman (2019) 的综述聚焦于免疫系统受损宿主中的真菌感染,强调流行病学特征和管理策略\cite{Pappas2019}。Zhang 和 Zhang (2020) 研究了真菌在腐生与寄生生活阶段之间的转换,讨论了这一过程对农业病害管理的启示\cite{Zhang2020}。Brunner 和 Kottke (2021) 则探讨了真菌病原体的复杂生活周期,分析了其从土壤获取营养到侵染宿主的机制,并强调了对植物病害管理的影响\cite{Brunner2021}。
	\subsubsection{气候变化与植物病害的关系}
	
	最新研究表明,气候变化和全球变暖导致温度的升高和部分地区降水格局的改变,正在加剧这些病害的发生和传播。温暖潮湿的环境有利于病原体的繁殖和扩散,导致病害在更大范围内更频繁地发生。例如,科学家发现全球变暖导致的温度升高和降水模式的改变,正促使一些病原真菌和卵菌向新的地理区域扩展,这些区域以前并不适合它们的生存和繁殖。气候变化还影响了植物的生理状态,使其更易受到病害侵染。实际上 , 温度和降水是影响叶片真菌病害的主要环境因子。 叶片真菌病害往往在高温、高湿的环境下较为严重。根据样点,使用机器学习方法预测全国病害有助于更好地认识到中国范围内病害的空间格局。
	
	植物病害对全球农业生产力和粮食安全构成重大挑战。及时准确地预测这些病害对于有效的病害管理和减轻策略至关重要。近年来,数据收集技术的进步促使了多样化数据集的获取,涵盖了气象条件、土壤特性、植物物种信息以及植物病害严重程度。
草地对动物产业、土壤保护和生物多样性至关重要,但植物病害会降低产量和营养价值\cite{Chakraborty2018}。病害选择性地影响了某些物种,从而减少了群落内的物种多样性和丰富度\cite{Grunberg2023}。

植物病理学家 Sarah J. Gurr 等人(2018)使用广义线性模型的研究发现,真菌和昆虫每年向两极迁移约7公里。相比之下,蠕虫(如线虫)则显示出向低纬度地区移动的趋势。对于其他分类群,如螨虫、细菌、双翅目、半翅目、膜翅目、等翅目、卵菌、原生动物、缨翅目和病毒,未观察到显著的纬度变化趋势。气候变化可能对不同害虫分类群的地理分布产生影响,其中一些群体正逐渐向两极迁移以适应新的环境条件。与此同时,CO₂浓度的升高导致植物病原体的感染能力增强 \cite{Sukumar2018}。

Anne Ebeling 等人(2023)的研究分析了不同植物类型在不同年均温度和年均降水条件下受病害和无脊椎动物损害的情况,揭示了它们对环境变化的不同响应。研究发现,在年均降水增加和年均温度升高的条件下,杂草表现出显著的病害和无脊椎动物损害增加的趋势,尤其是在高温高湿的环境中更为明显。相反,草类和豆科植物对这些环境因素的响应相对稳定,没有显示出明显的损害程度增加的趋势 \cite{Ebeling2023}。

Deepa S. Pureswaran 等人(2024)探讨了气候变化对森林害虫的影响。他们综合了2013-2017年间的最新文献,深入讨论了气候变化如何影响昆虫的分布范围、数量、森林生态系统及昆虫群落的影响。研究发现,气候变化可以促进害虫爆发或破坏食物链,进而减少害虫爆发的严重程度。通过广义线性模型和大尺度空间分析,该研究揭示了气候变化对不同昆虫类群的地理分布和生态影响。此外,气候变化导致英国部分地区的极端天气增多 \cite{Angelotti2024}。
	\subsection{研究内容和研究区概况}
	
	\subsubsection{研究内容}
	
	根据研究目标,本文需要进行数据采集和数据整合。数据整合,将数据自不同来源的数据整合到一起。数据整合的过程包括数据表的合并、连接或关联,以创建一个包含完整信息的数据集。最后进行预测分析并展现分析后的结果。
	
	\subsubsection{中国草地概述}
	草地在生态系统中扮演着多种角色。首先,它们是重要的碳汇,能够吸收大气中的二氧化碳并将其储存在植被和土壤中。其次,草地是许多野生动植物的栖息地,为它们提供了食物和庇护所。此外,草地还有助于防止土壤侵蚀,通过根系固定土壤,减少水土流失。在农业方面,草地是畜牧业的基础,为牲畜提供食物来源。通常,草地在全球的分布受到气候、地形和人类活动等多种因素的影响。例如,在非洲的萨瓦纳地区、北美的大草原以及南美的潘帕斯草原都是草地生态系统的典型代表。
	
	至于中国,它拥有世界上最大的草地面积之一,约占全球草地面积的百分之10左右。
	
	中国的草地主要分布在西部和北部地区,中国的草地资源非常丰富,草地类型多样,这些地区包括主要分布在内蒙古草原、青藏高原草地、新疆草地、东北草地和黄土高原草地等区域。这些草地生态系统在地理上呈现出明显的纬度和海拔梯度。
	
	内蒙古高原是中国最大的草地区域之一,这里的草地以温带草原为主,是重要的畜牧业基地。内蒙古的草地覆盖了广阔的平原和低山丘陵地带,为众多的牲畜提供了丰富的食物资源。内蒙古自治区是中国最大的草原分布区,涵盖了呼伦贝尔草原、锡林郭勒草原和阿拉善草原。呼伦贝尔草原以其平坦广袤的草地和优质的牧草著称,是优良的天然牧场。锡林郭勒草原以丰富的生物多样性和独特的自然景观闻名,而阿拉善草原则以其干旱和半干旱的生态环境为特色。
	
	青藏高原则以其高海拔草地著称,这里的草地属于高山草甸类型,由于海拔高,气候寒冷,草地生长的植物种类相对较少,但它们对于维持高原生态系统的稳定和生物多样性具有重要作用。青藏高原位于中国西南部,草地主要集中在青海省、西藏自治区和四川省的部分地区。这里的草地包括高寒草甸和高寒草原,以高寒冷湿的气候和复杂的地形为特征,是许多珍稀野生动物的栖息地,如藏羚羊和野牦牛。
	
	新疆地区则有干旱和半干旱的草地,这里的草地生态系统适应了干旱的环境条件,多为耐旱和耐盐碱的植物种类。新疆的草地在支持当地畜牧业和保护生态平衡方面发挥着关键作用。新疆维吾尔自治区的草地主要分布在天山山脉和阿尔泰山脉地区,包括天山草甸草原和阿尔泰山草原。这里的草地气候干旱,植被稀疏,但却是重要的畜牧业基地。
	
	东北地区的草地主要集中在吉林省和黑龙江省的部分地区,如松嫩平原和三江平原,受季风气候影响,夏季湿润,适宜草原植物生长。
	
	黄土高原位于中国西北部,草地主要分布在陕西省和甘肃省的部分地区,由于气候干旱,多为干旱草原和荒漠草原,植被稀疏,土壤贫瘠。
	
	然而,草地生态环境也面临着过度放牧、气候变化和土地荒漠化等挑战,需要加强保护和可持续管理。 整体来看,中国的草地生态系统在地理分布上呈现出多样化的特点,从温带草原到高山草甸,再到干旱草原,它们不仅为畜牧业提供了基础,对水土保持、防风固沙和维护生物多样性具有重要作用,对于维持区域乃至全球的生态平衡具有不可替代的作用。同时,这些草地也是中国重要的自然景观和生态旅游资源,对于促进地方经济发展和生态旅游具有重要意义。
	
	
		\subsubsection{中国草地分类情况}
		
		中国草地可以分为草原,草原和草丛。
	
	
		\subsubsubsection{新疆地区草地分类}
	对于新疆地区,草甸草原主要分布在新疆的北部和西部地区,尤其是在靠近阿尔泰山、天山等高海拔地区。这些区域因为海拔较高、降水较多,适合草甸类植被的生长。草原主要出现在新疆西北部和南部的部分山地区域,例如巴音郭楞蒙古自治州周边地区。这些草原植被适应了较为干旱的气候,覆盖范围较大。
	
	
			 
	\begin{figure}[H]  % 使用 [H] 来强制图片在此处显示
		\centering
		\includegraphics[width=0.8\textwidth, keepaspectratio]{pic/xinjiang_dem.png} % 自动保持纵横比
	\end{figure}
	
	
	\begin{figure}[H]  % 使用 [H] 来强制图片在此处显示
		\centering
		\includegraphics[width=0.8\textwidth, keepaspectratio]{pic/xinjiang_grassland.png} % 自动保持纵横比
		\caption{数据来源于中国寒旱区冻土研究所官网 \cite{FRST}.}
		\label{fig:gansu_grassland}
	\end{figure}
	
		\subsubsubsection{青海省草地分类情况}
		\par 对于青海省,在青海省的东部高山地区,如黄南、海南藏族自治州、果洛藏族自治州等地,这些区域因为海拔较高,气候湿润,适合草甸类植被生长。草原主要分布在青海省的西部和北部区域,例如海西蒙古族藏族自治州和青海湖附近。这些地区气候干燥,降水较少,植被类型多为耐旱的草原。
		
			\begin{figure}[H]  % 使用 [H] 来强制图片在此处显示
			\centering
			\includegraphics[width=0.8\textwidth, keepaspectratio]{pic/qinghai_dem.png} % 自动保持纵横比
		\end{figure}
		
		
		\begin{figure}[H]  % 使用 [H] 来强制图片在此处显示
			\centering
			\includegraphics[width=0.8\textwidth, keepaspectratio]{pic/qinghai_grassland2.png} % 自动保持纵横比
			\caption{数据来源于中国寒旱区冻土研究所官网 \cite{FRST}.}
			\label{fig:gansu_grassland}
		\end{figure}
		
		
		
		
		\subsubsubsection{甘肃省草地分类情况}
		
		 对于甘肃省,草甸植被主要分布在甘肃省的东南部和中部的高海拔地区,尤其是临夏回族自治州、甘南藏族自治州等山区,这些地区气候相对湿润,适合草甸植被生长。草原植被广泛分布在甘肃省的西部和北部地区,如酒泉市、张掖市、嘉峪关市等,这些区域气候干燥,降水较少,因此草原植被更为常见。
		 
		 	\begin{figure}[H]  % 使用 [H] 来强制图片在此处显示
		 	\centering
		 	\includegraphics[width=0.8\textwidth, keepaspectratio]{pic/gansu_dem.png} % 自动保持纵横比
		 \end{figure}
		 
		 
	\begin{figure}[H]  % 使用 [H] 来强制图片在此处显示
		\centering
		\includegraphics[width=0.8\textwidth, keepaspectratio]{pic/gansu_grassland_arcgis.png} % 自动保持纵横比
		\caption{数据来源于中国寒旱区冻土研究所官网 \cite{FRST}.}
		\label{fig:gansu_grassland}
	\end{figure}
			\subsubsubsection{四川省草地分类情况}
	\par 对于四川省,草甸植被主要分布在西部的高原和山区,特别是甘孜藏族自治州、阿坝藏族羌族自治州等地。这些区域海拔较高,气候较为湿润,适合草甸植被的生长。这里的草甸植被通常与丰富的降水和冷凉的气候条件相关。在四川省,草丛分布较为稀疏,集中在部分东南部地区,如攀枝花市及周边地区,这些区域的气候较为温暖,适合草丛植被的生长。
			 
	\begin{figure}[H]  % 使用 [H] 来强制图片在此处显示
		\centering
		\includegraphics[width=0.8\textwidth, keepaspectratio]{pic/sichuan_dem.png} % 自动保持纵横比
	\end{figure}
	
		\begin{figure}[H]  % 使用 [H] 来强制图片在此处显示
		\centering
		\includegraphics[width=0.8\textwidth, keepaspectratio]{pic/sichuan_grassland2.png} % 自动保持纵横比
		\caption{数据来源于中国寒旱区冻土研究所官网 \cite{FRST}.}
		\label{fig:gansu_grassland}
	\end{figure}
		\subsubsubsection{西藏自治区草地分类情况}
		\par 西藏自治区的草地主要以草甸为主,广泛分布在西藏的中部和北部地区,包括那曲市、阿里地区以及部分拉萨市周边的高原区域。这些地区地势较高,气候寒冷湿润,适合草甸的生长,草甸在西藏的高原地带显得尤为广泛。草原主要分布在西藏自治区的西部和南部边缘,如日喀则市、山南市等区域。这些地区的气候相对干燥,草原植被适应了这种相对恶劣的环境条件,因此占据了这些干旱半干旱地区。
						 	\begin{figure}[H]  % 使用 [H] 来强制图片在此处显示
			\centering
			\includegraphics[width=0.8\textwidth, keepaspectratio]{pic/xizang_dem.png} % 自动保持纵横比
		\end{figure}
		
		
		\begin{figure}[H]  % 使用 [H] 来强制图片在此处显示
			\centering
			\includegraphics[width=0.8\textwidth, keepaspectratio]{pic/xizang_grassland2.png} % 自动保持纵横比
			\caption{数据来源于中国寒旱区冻土研究所官网 \cite{FRST}.}
			\label{fig:gansu_grassland}
		\end{figure}
		
		
		\subsubsubsection{内蒙古自治区草地分类情况}
		\par 对于内蒙古,草原在自治区内占据了广大的面积,是草地的主要类型。草原分布较为广泛,尤其集中在东部的呼伦贝尔、锡林郭勒以及赤峰等地区,这些地区地势平缓,气候适宜,是典型的草原牧区,拥有丰富的草原资源。草甸也出现在内蒙古的一些地方,主要分布在北部和中部地区的局部,如兴安盟和部分呼和浩特市的周边,受当地水源和湿润气候的影响,草甸类型的植被较为丰富。
				 	\begin{figure}[H]  % 使用 [H] 来强制图片在此处显示
			\centering
			\includegraphics[width=0.8\textwidth, keepaspectratio]{pic/neimeng_dem.png} % 自动保持纵横比
		\end{figure}
		

				\begin{figure}[H]  % 使用 [H] 来强制图片在此处显示
			\centering
			\includegraphics[width=0.8\textwidth, keepaspectratio]{pic/neimeng_grassland.png} % 自动保持纵横比
			\caption{数据来源于中国寒旱区冻土研究所官网 \cite{FRST}.}
			\label{fig:gansu_grassland}
		\end{figure}

		\subsubsection{中国西北各省土地利用情况概述}
		
		\subsubsubsection{甘肃省的土地分类情况}
		\par 甘肃省的土地利用类型呈现出多样化的特点,主要包括草原、裸地、耕地和落叶阔叶林等几种主要分类。草原是甘肃省最为广泛的土地类型之一,主要分布在省内的高原、丘陵及一些半干旱地区。草原地区生态环境较为脆弱,植被覆盖率较低,主要依靠天然草地的生长和放牧活动维持其生态功能。由于降水量相对较少,这些区域的植被生长主要依赖于季节性降水,形成了典型的干旱和半干旱草原景观。
		
		裸地在甘肃省的土地利用分类中也占有重要地位,尤其是在一些山地和丘陵地区,由于水土流失、沙化等因素的影响,大面积的裸露土地呈现出荒漠化特征。裸地的出现与当地气候条件以及人类活动密切相关,这些地区土壤贫瘠,水分缺乏,通常缺乏较为丰富的植被覆盖,因此裸地类型的面积相对较大。
		
		耕地在甘肃省的土地利用中占有较为显著的地位,尤其是在黄土高原的平原和山间盆地等地带,农业开发较为集中。耕地类型主要以粮食作物的种植为主,尤其是小麦、玉米等作物。人类活动对这类土地的开发利用较为密集,农业活动对土地的改造和利用程度较高,但同时也面临着土地退化、沙化和水土流失等环境问题的挑战。
		
		落叶阔叶林分布在甘肃省的部分山区,尤其是在海拔较高的地方。这些地区的气候相对湿润,植被类型丰富,森林资源较为丰富。落叶阔叶林的植被类型主要以落叶树种为主,生态环境良好,在维持水土、保护生物多样性和调节气候方面起着重要作用。随着生态保护力度的增加,这些区域的森林资源得到了有效的保护和恢复。
		
	
		综上所述,甘肃省的土地利用类型呈现出明显的区域性差异,从广袤的草原和裸地,到耕地和森林区域,各类土地利用形态在不同的自然条件和人类活动的共同作用下形成了复杂多样的土地利用格局。
		
		
		
		甘肃省的土地利用情况在2001年至2020年间变化较为平稳。荒地或稀疏植被的比例呈现出轻微下降趋势,从2001年的43.04\%逐渐减少至2020年的40.25\%。草地的比例基本稳定,维持在36\%左右,最高为2018年的36.36\%,最低为2007年的36.13\%。耕地的比例在12\%上下波动,略有下降,从2001年的11.88\%增加至2020年的12.61\%。落叶阔叶林的比例逐年增加,从2001年的2.98\%增加至2020年的4.10\%,表现出逐渐上升的趋势。其他土地类型的比例也在逐年增加,从2001年的5.51\%增至2020年的6.98\%。总体来看,甘肃省的土地利用结构保持较为稳定,荒地或稀疏植被有所减少,草地和耕地保持稳定,而落叶阔叶林和其他土地类型有所增长。
		
		\begin{table}[H]
			\centering
			\begin{tabular}{|c|c|c|c|c|c|}
				\hline
				年份 & 荒地或稀疏植被 & 草地 & 耕地 & 落叶阔叶林 & 其他 \\
				\hline
				2001 & 43.04\% & 36.59\% & 11.88\% & 2.98\% & 5.51\% \\
				2002 & 42.91\% & 36.47\% & 12.10\% & 3.03\% & 5.48\% \\
				2003 & 42.86\% & 36.36\% & 12.23\% & 3.11\% & 5.44\% \\
				2004 & 42.78\% & 36.25\% & 12.27\% & 3.16\% & 5.54\% \\
				2005 & 42.68\% & 36.14\% & 12.28\% & 3.26\% & 5.64\% \\
				2006 & 42.58\% & 36.15\% & 12.25\% & 3.32\% & 5.70\% \\
				2007 & 42.28\% & 36.13\% & 12.26\% & 3.40\% & 5.92\% \\
				2008 & 42.13\% & 36.07\% & 12.33\% & 3.47\% & 6.01\% \\
				2009 & 42.00\% & 36.16\% & 12.32\% & 3.53\% & 6.00\% \\
				2010 & 41.86\% & 36.18\% & 12.37\% & 3.61\% & 5.98\% \\
				2011 & 41.70\% & 36.11\% & 12.45\% & 3.68\% & 6.07\% \\
				2012 & 41.59\% & 35.86\% & 12.61\% & 3.70\% & 6.23\% \\
				2013 & 41.57\% & 35.81\% & 12.59\% & 3.68\% & 6.35\% \\
				2014 & 41.58\% & 35.85\% & 12.48\% & 3.64\% & 6.45\% \\
				2015 & 41.58\% & 35.83\% & 12.37\% & 3.61\% & 6.61\% \\
				2016 & 41.47\% & 35.91\% & 12.19\% & 3.59\% & 6.83\% \\
				2017 & 41.17\% & 36.06\% & 12.11\% & 3.59\% & 7.06\% \\
				2018 & 40.83\% & 36.23\% & 12.16\% & 3.63\% & 7.15\% \\
				2019 & 40.30\% & 36.36\% & 12.48\% & 3.87\% & 6.99\% \\
				2020 & 40.25\% & 36.07\% & 12.61\% & 4.10\% & 6.98\% \\
				\hline
			\end{tabular}
			\caption{甘肃省的土地利用情况}
		\end{table}
	\begin{figure}[H]  % 使用 [H] 来强制图片在此处显示
			\centering
			\includegraphics[width=0.8\textwidth, keepaspectratio]{pic/gansu_idgp2.png} % 自动保持纵横比
		\end{figure}
		
		\begin{figure}[H]  % 使用 [H] 来强制图片在此处显示
		\centering
		\includegraphics[width=0.8\textwidth, keepaspectratio]{pic/gansu_land_use_1.png} % 自动保持纵横比
	\end{figure}
	\begin{figure}[H]  % 使用 [H] 来强制图片在此处显示
		\centering
		\includegraphics[width=0.8\textwidth, keepaspectratio]{pic/gansu_land_use_2.png} % 自动保持纵横比
	\end{figure}
		\begin{figure}[H]  % 使用 [H] 来强制图片在此处显示
		\centering
		\includegraphics[width=0.8\textwidth, keepaspectratio]{pic/gansu_land_use_3.png} % 自动保持纵横比
	\end{figure}
		\begin{figure}[H]  % 使用 [H] 来强制图片在此处显示
		\centering
		\includegraphics[width=0.8\textwidth, keepaspectratio]{pic/gansu_land_use_4.png} % 自动保持纵横比
	\end{figure}
	
		\subsubsubsection{青海省的土地分类情况}
		\par 青海省的 MODIS 土地利用分类展现了该地区以高原为主的独特地理特征,土地覆盖类型与气候、地形和海拔密切相关,草原是青海省最为显著的土地利用类型,广泛分布于省内的青藏高原、黄河上游以及其他中高海拔地区。草地资源种类繁多,从湿润的高原草甸到干旱的荒漠草原,再到高寒的草甸,每一种草地都有其特定的生态功能。青海的草地不仅支撑着当地的畜牧业经济,也为生物多样性提供了重要栖息地,但同时面临过度放牧和草地退化的生态挑战。
		
		除了草地外,裸地在青海省的土地利用中占有重要地位,尤其在省内的干旱和半干旱地区,如柴达木盆地和塔尔寺周边的戈壁沙漠区。裸地主要由沙漠、戈壁和荒漠地形构成,植被稀少或几乎不存在,气候条件极为干旱,土地开发利用困难。随着全球气候变化,裸地面积在某些区域可能有所增加,荒漠化问题日益严峻,对生态环境的稳定构成了威胁。
		
		青海省的湖泊和湿地资源在 MODIS 土地利用分类中也占据一定份额。省内的湖泊如青海湖、察尔湖等不仅是重要的水资源储备地,还是丰富的生态栖息地。湖泊和湿地的生态功能在调节水循环、提供生物栖息地以及保持区域气候平衡方面具有不可替代的作用。尽管这些湿地和湖泊资源在青海省的总面积中并不占主导地位,但它们在维持生态稳定性方面发挥着重要作用。	
		青海省的冻土区是该省独特的土地类型之一,主要分布在省内的高海拔区域。冻土区在 MODIS 数据中表现为一种特殊的土地利用类型,冻土不仅影响着植物生长,还对水文过程和碳储存具有重要影响。随着气候变暖,冻土区的变化对当地生态系统和环境造成了深远的影响。
		
		总体而言,青海省的土地利用类型以草原、裸地、湖泊和冻土为主,森林相对较少,且随着气候变化和人类活动的影响,土地的利用和生态环境面临一定的压力。如何在保障生态安全的前提下合理利用和保护土地资源,已成为青海省土地管理的重要课题。
		
		青海省的土地利用情况在2001年至2019年间呈现出一定的稳定性和小幅波动。草地的比例始终保持在65\%至67\%之间,表现出较为稳定的趋势。荒地或稀疏植被的比例逐年略有下降,从2001年的31.45\%降至2019年的29.97\%,但始终维持在30\%附近。水体的比例逐渐上升,从2001年的1.46\%增加至2019年的1.61\%。耕地的比例保持在0.77\%至0.91\%之间,呈现小幅波动。永久雪冰的比例略有上升,从2001年的0.30\%增加至2019年的0.44\%。其他土地类型的比例也有小幅波动,从2001年的0.26\%增至2019年的0.32\%。总体来看,青海省的土地利用结构保持稳定,草地比例占主导地位,而荒地或稀疏植被、耕地和其他类型土地的比例逐渐减少或保持稳定。
		
		\begin{table}[H]
			\centering
			\caption{青海省土地利用情况}
			\begin{tabular}{|c|c|c|c|c|c|c|}
				\hline
				年份 & 草地 & 荒地或稀疏植被 & 水体 & 耕地 & 永久雪冰 & 其他 \\
				\hline
				2001 & 65.63\% & 31.45\% & 1.46\% & 0.90\% & 0.30\% & 0.26\% \\
				2002 & 65.76\% & 31.30\% & 1.47\% & 0.90\% & 0.30\% & 0.26\% \\
				2003 & 65.89\% & 31.14\% & 1.47\% & 0.91\% & 0.31\% & 0.27\% \\
				2004 & 66.00\% & 31.02\% & 1.48\% & 0.91\% & 0.32\% & 0.27\% \\
				2005 & 66.18\% & 30.83\% & 1.48\% & 0.90\% & 0.33\% & 0.28\% \\
				2006 & 66.31\% & 30.69\% & 1.49\% & 0.89\% & 0.34\% & 0.28\% \\
				2007 & 66.37\% & 30.60\% & 1.51\% & 0.88\% & 0.36\% & 0.28\% \\
				2008 & 66.42\% & 30.54\% & 1.52\% & 0.88\% & 0.36\% & 0.28\% \\
				2009 & 66.57\% & 30.41\% & 1.53\% & 0.87\% & 0.35\% & 0.27\% \\
				2010 & 66.71\% & 30.28\% & 1.54\% & 0.85\% & 0.34\% & 0.27\% \\
				2011 & 66.73\% & 30.28\% & 1.53\% & 0.84\% & 0.34\% & 0.27\% \\
				2012 & 66.72\% & 30.31\% & 1.54\% & 0.83\% & 0.34\% & 0.27\% \\
				2013 & 66.66\% & 30.38\% & 1.54\% & 0.81\% & 0.34\% & 0.26\% \\
				2014 & 66.60\% & 30.45\% & 1.54\% & 0.80\% & 0.34\% & 0.26\% \\
				2015 & 66.63\% & 30.45\% & 1.54\% & 0.78\% & 0.35\% & 0.26\% \\
				2016 & 66.62\% & 30.45\% & 1.53\% & 0.77\% & 0.36\% & 0.27\% \\
				2017 & 66.62\% & 30.43\% & 1.55\% & 0.76\% & 0.37\% & 0.28\% \\
				2018 & 66.85\% & 30.12\% & 1.58\% & 0.78\% & 0.38\% & 0.30\% \\
				2019 & 66.83\% & 29.97\% & 1.61\% & 0.82\% & 0.44\% & 0.32\% \\
				\hline
			\end{tabular}
		\end{table}
		
			\begin{figure}[H]  % 使用 [H] 来强制图片在此处显示
			\centering
			\includegraphics[width=0.8\textwidth, keepaspectratio]{pic/qingahai_idgp2.png} % 自动保持纵横比
		\end{figure}
		
		
		\subsubsubsection{新疆地区的土地分类情况}
		\par 新疆维吾尔自治区的土地利用类型丰富多样,主要包括裸地、草原、耕地和冻土等几种主要分类。裸地在新疆的土地利用中占据了较大比重,尤其是在沙漠和荒漠化地区。由于新疆地处亚欧大陆腹地,气候干旱且降水稀少,广袤的戈壁、沙漠和荒漠化地带形成了大面积的裸露土地。这些区域通常土壤贫瘠,植被稀疏,风沙天气频繁,导致土壤的表层裸露,无法支撑大面积的植物生长。裸地主要分布在塔克拉玛干沙漠、戈壁滩等地带,对当地的自然环境和生态系统产生了一定的影响。

		草原是新疆另一个重要的土地利用类型,尤其是在该自治区的北部和东部地区。新疆的草原主要位于准噶尔盆地和天山南北的广阔地区,这些草原地带以天然草地为主,具有较强的牧业价值。草原的植被覆盖以耐旱的草本植物为主,适宜牧民进行畜牧业活动。由于气候条件较为干旱,这些草原的生态系统相对脆弱,草地的过度放牧和气候变化等因素使得一些草原面临退化的风险,但总体上草原依然是新疆农业和牧业生产的主要基础。
		
		耕地在新疆的土地利用中也占据重要地位,尤其是在塔里木盆地及其周边地区。新疆的耕地分布相对集中,主要以小麦、棉花、玉米等粮食作物和经济作物的种植为主。由于地处干旱区,农业生产主要依赖灌溉,尤其是利用河流的水源进行灌溉。塔里木河流域和吐鲁番盆地是新疆重要的农业生产区,这些地区的农业集约化程度较高。然而,过度的灌溉和土地开发也带来了水土盐碱化等生态问题,成为新疆农业可持续发展的挑战。
		
		冻土主要分布在新疆的高纬度和高海拔地区,特别是天山和阿尔泰山脉的高山地区。由于气温低,冻土区域的土壤长期处于冻结状态,冻结层下的土壤和水分对植物的生长有很大限制。冻土区的生态环境较为特殊,这些地区的植被生长相对缓慢,且多以耐寒的植物为主。随着气候变化的影响,冻土的融化在一定程度上影响了当地的生态环境,尤其是高山地区的水源循环和生态平衡。
		
		总的来说,新疆的土地利用分类呈现出强烈的区域性特征,裸地、草原、耕地和冻土等多种土地类型相互交织,受到自然环境、气候变化和人类活动的综合影响。不同地区的土地利用类型不仅反映了该地区的生态特点,也反映了当地经济发展和农业生产的需求。
		
		整体来看,荒地或稀疏植被的比例在2001年至2020年间逐年略有下降,从73.88\%降至71.00\%,呈现出缓慢下降的趋势。草地的比例相对稳定,始终维持在21\%至22\%之间,略有波动,并在2020年达到22.43\%。耕地的比例逐年上升,从2001年的2.74\%增加至2020年的4.01\%,呈现出稳步增长的趋势。永久雪冰的比例也呈现出小幅波动,从2001年的0.97\%增长至2020年的1.48\%。其他土地类型的比例逐年轻微上升,从2001年的0.93\%上升至2020年的1.08\%。总体而言,荒地或稀疏植被比例有所下降,草地比例保持稳定,而耕地、永久雪冰和其他土地类型的比例则有所增加。
		
		\begin{table}[H]
			\centering
			\begin{tabular}{|c|c|c|c|c|c|}
				\hline
				年份 & 荒地或稀疏植被 & 草地 & 耕地 & 永久雪冰 & 其他 \\
				\hline
				2001 & 73.88\% & 21.48\% & 2.74\% & 0.97\% & 0.93\% \\
				2002 & 73.73\% & 21.55\% & 2.77\% & 1.03\% & 0.92\% \\
				2003 & 73.52\% & 21.59\% & 2.81\% & 1.18\% & 0.91\% \\
				2004 & 73.43\% & 21.58\% & 2.87\% & 1.21\% & 0.91\% \\
				2005 & 73.17\% & 21.60\% & 2.94\% & 1.37\% & 0.91\% \\
				2006 & 73.21\% & 21.58\% & 3.05\% & 1.26\% & 0.90\% \\
				2007 & 73.05\% & 21.54\% & 3.16\% & 1.34\% & 0.91\% \\
				2008 & 73.01\% & 21.54\% & 3.26\% & 1.29\% & 0.91\% \\
				2009 & 72.88\% & 21.55\% & 3.36\% & 1.29\% & 0.92\% \\
				2010 & 72.70\% & 21.60\% & 3.51\% & 1.26\% & 0.93\% \\
				2011 & 72.44\% & 21.66\% & 3.63\% & 1.32\% & 0.95\% \\
				2012 & 72.23\% & 21.70\% & 3.75\% & 1.37\% & 0.95\% \\
				2013 & 72.18\% & 21.66\% & 3.89\% & 1.31\% & 0.96\% \\
				2014 & 72.13\% & 21.64\% & 3.98\% & 1.30\% & 0.96\% \\
				2015 & 71.94\% & 21.56\% & 4.09\% & 1.42\% & 0.99\% \\
				2016 & 71.77\% & 21.58\% & 4.14\% & 1.49\% & 1.01\% \\
				2017 & 71.27\% & 21.88\% & 4.26\% & 1.56\% & 1.03\% \\
				2018 & 71.08\% & 22.10\% & 4.28\% & 1.49\% & 1.05\% \\
				2019 & 70.89\% & 22.31\% & 4.21\% & 1.51\% & 1.08\% \\
				2020 & 71.00\% & 22.43\% & 4.01\% & 1.48\% & 1.08\% \\
				\hline
			\end{tabular}
			\caption{新疆地区的土地利用情况(2001-2020)}
		\end{table}
		
		
				
		\begin{figure}[H]  % 使用 [H] 来强制图片在此处显示
			\centering
			\includegraphics[width=0.8\textwidth, keepaspectratio]{pic/xinjiang_idgp2.png} % 自动保持纵横比
		\end{figure}
		
			\begin{figure}[H]  % 使用 [H] 来强制图片在此处显示
			\centering
			\includegraphics[width=0.8\textwidth, keepaspectratio]{pic/xinjiang_land_use_1.png} % 自动保持纵横比
		\end{figure}
		\begin{figure}[H]  % 使用 [H] 来强制图片在此处显示
			\centering
			\includegraphics[width=0.8\textwidth, keepaspectratio]{pic/xinjiang_land_use_2.png} % 自动保持纵横比
		\end{figure}
		\begin{figure}[H]  % 使用 [H] 来强制图片在此处显示
			\centering
			\includegraphics[width=0.8\textwidth, keepaspectratio]{pic/xinjiang_land_use_3.png} % 自动保持纵横比
		\end{figure}
		\begin{figure}[H]  % 使用 [H] 来强制图片在此处显示
			\centering
			\includegraphics[width=0.8\textwidth, keepaspectratio]{pic/xinjiang_land_use_4.png} % 自动保持纵横比
		\end{figure}
		
		
		\subsubsubsection{西藏地区的土地分类情况}
		\par
		西藏自治区的土地利用类型具有显著的高原特色,主要包括裸地、草原、常绿针叶林和常绿阔叶林等几种主要分类。裸地是西藏地区土地利用中的一个重要类型,广泛分布于高原山地和河谷地区。由于西藏大部分地区海拔较高,气候寒冷且干燥,许多地方土地贫瘠,植被难以生长,形成了大面积的裸地。裸地的形成与该地区的严酷自然条件密切相关,特别是在高山、戈壁及一些荒漠化地带,裸土表面容易受到风蚀和水蚀,极为脆弱的生态环境使得这些地区的土地很难维持长期的植被覆盖。
		
		草原在西藏的土地利用中占有重要地位,尤其是在自治区的低海拔和中海拔地区。西藏的草原类型多样,主要包括高原草甸、沼泽草原和干旱草原等。这些草原区域以天然草地为主,植被覆盖以耐旱的草本植物为主,具有重要的牧业价值。草原不仅是藏族牧民放牧的重要场所,也是重要的生态系统,能够有效调节水土、保护生物多样性。草原的生态环境较为脆弱,气候变化和过度放牧可能导致草地退化,给当地的生态系统带来压力。
		
		常绿针叶林和常绿阔叶林在西藏的土地利用中分布较为局限,主要集中在西藏的东南部及一些较低海拔的山地地区。常绿针叶林主要分布在喜马拉雅山脉的一些湿润地区,这些地区气候较为温和,降水充沛,植被茂密。常绿针叶林以冷杉、云杉等针叶树为主,生长较为缓慢,但其生态功能极为重要,能够有效防止水土流失,保持水源涵养,并为多种野生动植物提供栖息环境。常绿阔叶林在西藏的分布则较为局限,主要出现在东南部的潮湿地区,常见的树种包括一些阔叶树,如橡树、枫树等,这些森林在生态保护和气候调节方面发挥着重要作用。
		
		综上所述,西藏自治区的土地利用分类呈现出显著的高原特色,从广阔的裸地到富饶的草原,再到局部分布的常绿针叶林和常绿阔叶林,构成了丰富多样的自然景观。每一种土地类型都与该地区独特的气候、地形和生态环境密切相关,反映了西藏特殊的地理和气候条件对土地利用的影响。
		
		
		整体来看,草地的比例始终维持在52\%至54\%之间,表现出较为稳定的趋势。荒地或稀疏植被的比例则保持在34\%至36\%之间,变化幅度较小,整体略有下降。常绿阔叶林的比例较为稳定,基本保持在2.6\%左右。常绿针叶林的比例在2001年至2020年间小幅波动,保持在1.9\%至2.0\%之间。其他土地类型的比例逐年上升,从2001年的7.11\%上升至2020年的7.67\%。总体而言,草地和荒地或稀疏植被的比例保持稳定,常绿阔叶林和常绿针叶林比例变化不大,而其他土地类型则逐步增加。
		
		
		\begin{table}[H]
			\centering
			\begin{tabular}{|c|c|c|c|c|c|}
				\hline
				年份 & 草地 & 荒地或稀疏植被 & 常绿阔叶林 & 常绿针叶林 & 其他 \\
				\hline
				2001 & 52.37\% & 35.88\% & 2.64\% & 2.00\% & 7.11\% \\
				2002 & 52.66\% & 35.69\% & 2.64\% & 2.00\% & 7.02\% \\
				2003 & 53.02\% & 35.35\% & 2.64\% & 1.96\% & 7.04\% \\
				2004 & 53.27\% & 35.10\% & 2.64\% & 1.94\% & 7.06\% \\
				2005 & 53.46\% & 34.92\% & 2.63\% & 1.91\% & 7.08\% \\
				2006 & 53.56\% & 34.81\% & 2.63\% & 1.89\% & 7.12\% \\
				2007 & 53.68\% & 34.70\% & 2.63\% & 1.87\% & 7.13\% \\
				2008 & 53.68\% & 34.68\% & 2.62\% & 1.88\% & 7.13\% \\
				2009 & 53.75\% & 34.62\% & 2.61\% & 1.89\% & 7.14\% \\
				2010 & 53.71\% & 34.58\% & 2.61\% & 1.88\% & 7.22\% \\
				2011 & 53.69\% & 34.52\% & 2.60\% & 1.87\% & 7.32\% \\
				2012 & 53.64\% & 34.47\% & 2.59\% & 1.87\% & 7.42\% \\
				2013 & 53.70\% & 34.46\% & 2.59\% & 1.89\% & 7.36\% \\
				2014 & 53.64\% & 34.50\% & 2.59\% & 1.91\% & 7.36\% \\
				2015 & 53.61\% & 34.51\% & 2.58\% & 1.89\% & 7.40\% \\
				2016 & 53.59\% & 34.49\% & 2.58\% & 1.88\% & 7.46\% \\
				2017 & 53.73\% & 34.24\% & 2.58\% & 1.93\% & 7.52\% \\
				2018 & 53.72\% & 34.14\% & 2.57\% & 2.01\% & 7.56\% \\
				2019 & 52.97\% & 34.68\% & 2.56\% & 1.98\% & 7.80\% \\
				2020 & 52.95\% & 34.79\% & 2.58\% & 2.02\% & 7.67\% \\
				\hline
			\end{tabular}
			\caption{西藏自治区的土地利用情况(2001-2020)}
		\end{table}
		
		\begin{figure}[H]  % 使用 [H] 来强制图片在此处显示
			\centering
			\includegraphics[width=0.8\textwidth, keepaspectratio]{pic/xizang_idgp3.png} % 自动保持纵横比
		\end{figure}
		
					\begin{figure}[H]  % 使用 [H] 来强制图片在此处显示
			\centering
			\includegraphics[width=0.8\textwidth, keepaspectratio]{pic/xizang_land_use_1.png} % 自动保持纵横比
		\end{figure}
		\begin{figure}[H]  % 使用 [H] 来强制图片在此处显示
			\centering
			\includegraphics[width=0.8\textwidth, keepaspectratio]{pic/xizang_land_use_2.png} % 自动保持纵横比
		\end{figure}
		\begin{figure}[H]  % 使用 [H] 来强制图片在此处显示
			\centering
			\includegraphics[width=0.8\textwidth, keepaspectratio]{pic/xizang_land_use_3.png} % 自动保持纵横比
		\end{figure}
		\begin{figure}[H]  % 使用 [H] 来强制图片在此处显示
			\centering
			\includegraphics[width=0.8\textwidth, keepaspectratio]{pic/xizang_land_use_4.png} % 自动保持纵横比
		\end{figure}
		
		\subsubsubsection{内蒙古自治区的土地分类情况}
		
		内蒙古自治区的土地利用类型丰富多样,主要包括裸地、草原、耕地和木本草原等几种主要分类,体现了该地区独特的地理和气候特征。裸地在内蒙古自治区的土地利用中占有重要地位,主要分布在沙漠、戈壁滩及荒漠化地区。由于内蒙古地处干旱和半干旱地区,降水稀少,土壤贫瘠,这些地区往往缺乏植被覆盖。裸地的广泛存在与该地区的极端气候和土地退化密切相关,风沙天气频繁,裸露的土地表面容易受到水土流失和风蚀的影响。裸地不仅反映了自然环境的严酷,还暴露了区域生态系统的脆弱性。
		
		草原是内蒙古的另一大土地利用类型,广泛分布在自治区的中部和西部,特别是呼伦贝尔草原、鄂尔多斯草原等地带。这些草原主要由天然草地组成,草本植物覆盖广泛,是内蒙古传统牧业的基础。草原生态系统具有较强的生产力和生物多样性,为牧民提供了丰富的放牧资源,同时也发挥着重要的水土保持功能。然而,草原生态环境较为脆弱,过度放牧和气候变化等因素可能导致草地退化,影响牧业生产和生态环境的稳定。
		
		耕地在内蒙古自治区的土地利用中占据较小但日益重要的地位,主要集中在自治区的农田灌溉区和冲积平原地带。内蒙古的耕地主要用于种植粮食作物,如小麦、玉米、大豆等,农业生产在这些区域逐渐得到发展,尤其在呼和浩特、鄂尔多斯等地的农业区,耕地的面积有所增加。由于内蒙古整体降水量较少,农业生产主要依赖灌溉系统,水资源的管理和利用成为耕地可持续发展的关键。然而,过度灌溉可能带来土壤盐碱化等问题,影响耕地的长期利用。
		
		木本草原是内蒙古土地利用分类中的一个独特类型,主要分布在草原地带的边缘或干旱区的过渡带。木本草原呈现出草本植物与小型树木共生的景象,树木通常分布稀疏,既没有形成连续的森林覆盖,也不像典型草原那样完全由草本植物构成。木本草原既能支持一定数量的小型树木,如灌木或矮树,也保留了草原生态系统的开放性,具有较高的生态多样性。虽然木本草原在内蒙古的土地利用中占比不大,但它作为草原与森林的过渡区,具有重要的生态功能,能够维持生态平衡,保护土壤和水源。
		
		总的来说,内蒙古自治区的土地利用类型展示了该地区的自然特征和经济活动的多样性。裸地、草原、耕地和木本草原等不同类型的土地利用相互交织,共同构成了内蒙古辽阔而复杂的自然景观和生态系统。每一种土地类型的分布和演变都与当地的气候、地形及人类活动密切相关,既反映了该地区自然资源的优势,也提出了生态保护与可持续发展的挑战。
		
		整体来看,草地占据了最大比例,始终维持在57\%至59\%之间,表现出较为稳定的趋势。荒地或稀疏植被的比例较为稳定,保持在23\%至25\%之间。耕地的比例略有上升,从2001年的5.59\%增长至2020年的6.38\%。林木稀树草原的比例波动较小,始终维持在5\%至6\%之间。其他土地类型的比例则有逐渐上升的趋势,从2001年的5.75\%上升至2020年的6.71\%。整体而言,草地和荒地比例较为稳定,而耕地和其他土地类型呈现逐步增加的趋势。

		\begin{table}[H]
			\centering
			\begin{tabular}{|c|c|c|c|c|c|}
				\hline
				年份 & 草地 & 荒地或稀疏植被 & 耕地 & 林木稀树草原 & 其他 \\
				\hline
				2001 & 58.47\% & 24.78\% & 5.59\% & 5.40\% & 5.75\% \\
				2002 & 58.69\% & 24.29\% & 5.76\% & 5.44\% & 5.82\% \\
				2003 & 58.73\% & 24.04\% & 5.80\% & 5.67\% & 5.77\% \\
				2004 & 58.47\% & 24.05\% & 5.70\% & 5.59\% & 6.19\% \\
				2005 & 58.44\% & 24.05\% & 5.59\% & 5.54\% & 6.37\% \\
				2006 & 58.66\% & 23.93\% & 5.46\% & 5.59\% & 6.37\% \\
				2007 & 58.88\% & 23.72\% & 5.32\% & 5.57\% & 6.51\% \\
				2008 & 58.75\% & 23.64\% & 5.40\% & 5.60\% & 6.61\% \\
				2009 & 58.58\% & 23.63\% & 5.36\% & 5.28\% & 7.15\% \\
				2010 & 58.34\% & 23.58\% & 5.45\% & 5.04\% & 7.59\% \\
				2011 & 58.22\% & 23.48\% & 5.55\% & 5.09\% & 7.65\% \\
				2012 & 58.05\% & 23.38\% & 5.71\% & 5.27\% & 7.59\% \\
				2013 & 57.69\% & 23.36\% & 5.84\% & 5.13\% & 7.97\% \\
				2014 & 57.48\% & 23.41\% & 5.99\% & 5.08\% & 8.04\% \\
				2015 & 57.32\% & 23.43\% & 6.16\% & 5.06\% & 8.03\% \\
				2016 & 57.19\% & 23.41\% & 6.31\% & 5.11\% & 7.98\% \\
				2017 & 57.30\% & 23.44\% & 6.26\% & 5.22\% & 7.78\% \\
				2018 & 57.68\% & 23.13\% & 6.42\% & 5.29\% & 7.48\% \\
				2019 & 58.19\% & 22.98\% & 6.44\% & 5.57\% & 6.82\% \\
				2020 & 58.14\% & 22.96\% & 6.38\% & 5.81\% & 6.71\% \\
				\hline
			\end{tabular}
			\caption{内蒙古自治区的土地利用情况(2001-2020)}
		\end{table}
		
				\begin{figure}[H]  % 使用 [H] 来强制图片在此处显示
			\centering
			\includegraphics[width=0.8\textwidth, keepaspectratio]{pic/neimeng_idgp2.png} % 自动保持纵横比
		\end{figure}
		
		
			\begin{figure}[H]  % 使用 [H] 来强制图片在此处显示
			\centering
			\includegraphics[width=0.8\textwidth, keepaspectratio]{pic/neimeng_land_use_1.png} % 自动保持纵横比
		\end{figure}
		\begin{figure}[H]  % 使用 [H] 来强制图片在此处显示
			\centering
			\includegraphics[width=0.8\textwidth, keepaspectratio]{pic/neimeng_land_use_2.png} % 自动保持纵横比
		\end{figure}
		\begin{figure}[H]  % 使用 [H] 来强制图片在此处显示
			\centering
			\includegraphics[width=0.8\textwidth, keepaspectratio]{pic/neimeng_land_use_3.png} % 自动保持纵横比
		\end{figure}
		\begin{figure}[H]  % 使用 [H] 来强制图片在此处显示
			\centering
			\includegraphics[width=0.8\textwidth, keepaspectratio]{pic/neimeng_land_use_4.png} % 自动保持纵横比
		\end{figure}
		
		
		
		\subsubsubsection{四川省的土地分类情况}
		
		
		四川省的土地利用类型多样,主要包括草原、混交林、稀树草原和木本草原等几种类型,体现了该省独特的地理、气候和生态特征。草原在四川省的土地利用中占有重要地位,尤其在四川的西部和北部的高原及山地地区,草原广泛分布。四川的草原通常分布在高原、山间盆地以及干湿季节变化明显的地区,这些地区草本植物生长旺盛,形成了丰富的草地景观。草原为该省的牧业提供了充足的放牧资源,同时也发挥着调节气候、保护水土的生态功能。然而,部分草原地区由于过度放牧和气候变化等因素,生态环境逐渐退化,草地退化和沙化现象逐步显现,造成了生态脆弱性增加。
		
		混交林在四川省的土地利用中具有重要地位,尤其在四川盆地的周边山区和川西高原地区。这些混交林由落叶阔叶树和常绿针叶树的混合种群构成,呈现出较为丰富的生物多样性。四川的气候条件相对湿润,降水丰富,使得混交林得以广泛分布,特别是在四川的东部和南部山区,这些地区的森林覆盖率较高。混交林不仅具有较强的生态服务功能,如调节气候、保护水源、固定碳等,还为多种野生动植物提供栖息地。随着生态保护力度的加强,四川省的森林资源逐渐恢复,混交林的生态地位也在不断提升。
		
		稀树草原在四川省的土地利用中主要分布在一些气候较为干旱或半干旱的地区,尤其是在四川盆地的边缘和川西高原的过渡带。稀树草原的植被特点是草本植物占主导地位,但在草原中也分布着较为稀疏的树木。树木的种类和数量相对较少,通常为一些适应干旱气候的耐旱树种或灌木。稀树草原是草原和森林之间的过渡带,它具有草原生态系统的开阔性和森林生态系统的部分特征,生物多样性较高,具有重要的生态意义。稀树草原通常承载着部分牧业和农耕活动,生态保护和合理利用是其可持续发展的关键。
		
		木本草原则是另一种过渡型生态系统,通常存在于草原与森林之间的过渡带,植被中既有丰富的草本植物,也有一定数量的小型树木和灌木。木本草原的特点是树木较为稀疏,但与草原相比,树木的覆盖度更高,且植被结构更为复杂。这种生态系统的存在通常与气候条件的多样性有关,既能适应干旱的环境,又能支持一定的森林植被。木本草原通常见于四川的一些山地、丘陵地区,尤其是在四川的西部和北部,气候相对温和,降水适中,适宜这种类型的植物生长。
		
		综上所述,四川省的土地利用类型展现了丰富的生态景观和复杂的自然环境。草原、混交林、稀树草原和木本草原等不同类型的生态系统,共同构成了四川独特的自然资源和生态功能。
		
		
		整体来看,草地的比例始终维持在31\%至33\%之间,变化幅度较小,表现出相对稳定的趋势。林木稀树草原的比例逐年略有下降,从2001年的19.62\%减少到2020年的18.83\%。稀树草原的比例波动较小,基本维持在14\%至15\%之间,整体略有下降。混交林的比例从2001年的10.40\%逐渐上升至12.37\%,呈现出逐步增长的趋势。其他土地类型的比例则持续增加,从2001年的21.56\%上升至2020年的24.23\%。整体来看,草地和林木稀树草原的比例保持稳定,而混交林和其他土地类型的比例则有所上升。
		
		
		\begin{table}[H]
			\centering
			\begin{tabular}{|c|c|c|c|c|c|}
				\hline
				年份 & 草地 & 林木稀树草原 & 稀树草原 & 混交林 & 其他 \\
				\hline
				2001 & 32.77\% & 19.62\% & 15.65\% & 10.40\% & 21.56\% \\
				2002 & 32.88\% & 19.65\% & 15.07\% & 10.48\% & 21.93\% \\
				2003 & 32.87\% & 19.74\% & 14.72\% & 10.46\% & 22.21\% \\
				2004 & 32.66\% & 19.81\% & 14.44\% & 10.47\% & 22.62\% \\
				2005 & 32.46\% & 19.91\% & 14.69\% & 10.44\% & 22.49\% \\
				2006 & 32.29\% & 20.00\% & 15.01\% & 10.45\% & 22.26\% \\
				2007 & 32.14\% & 20.06\% & 14.85\% & 10.38\% & 22.57\% \\
				2008 & 32.06\% & 20.22\% & 14.81\% & 10.25\% & 22.66\% \\
				2009 & 32.04\% & 20.36\% & 15.38\% & 10.18\% & 22.03\% \\
				2010 & 31.96\% & 20.46\% & 15.69\% & 10.09\% & 21.80\% \\
				2011 & 31.89\% & 20.45\% & 15.55\% & 10.13\% & 21.99\% \\
				2012 & 31.90\% & 20.47\% & 15.30\% & 10.15\% & 22.17\% \\
				2013 & 31.90\% & 20.19\% & 15.10\% & 10.43\% & 22.37\% \\
				2014 & 31.91\% & 19.94\% & 15.03\% & 10.71\% & 22.41\% \\
				2015 & 31.84\% & 19.58\% & 15.02\% & 11.04\% & 22.53\% \\
				2016 & 31.71\% & 19.26\% & 14.81\% & 11.27\% & 22.95\% \\
				2017 & 31.30\% & 18.74\% & 14.82\% & 11.54\% & 23.60\% \\
				2018 & 31.16\% & 18.71\% & 14.33\% & 11.66\% & 24.14\% \\
				2019 & 31.14\% & 18.72\% & 13.97\% & 12.00\% & 24.17\% \\
				2020 & 31.13\% & 18.83\% & 13.43\% & 12.37\% & 24.23\% \\
				\hline
			\end{tabular}
			\caption{四川省的土地利用情况(2001-2020)}
		\end{table}
		
						
		\begin{figure}[H]  % 使用 [H] 来强制图片在此处显示
			\centering
			\includegraphics[width=0.8\textwidth, keepaspectratio]{pic/sichuan_idgp.png} % 自动保持纵横比
		\end{figure}
		
		
		
		\begin{figure}[H]  % 使用 [H] 来强制图片在此处显示
			\centering
			\includegraphics[width=0.8\textwidth, keepaspectratio]{pic/sichuan_land_use_1.png} % 自动保持纵横比
		\end{figure}
		\begin{figure}[H]  % 使用 [H] 来强制图片在此处显示
			\centering
			\includegraphics[width=0.8\textwidth, keepaspectratio]{pic/sichuan_land_use_2.png} % 自动保持纵横比
		\end{figure}
		\begin{figure}[H]  % 使用 [H] 来强制图片在此处显示
			\centering
			\includegraphics[width=0.8\textwidth, keepaspectratio]{pic/sichuan_land_use_3.png} % 自动保持纵横比
		\end{figure}
		\begin{figure}[H]  % 使用 [H] 来强制图片在此处显示
			\centering
			\includegraphics[width=0.8\textwidth, keepaspectratio]{pic/sichuan_land_use_4.png} % 自动保持纵横比
		\end{figure}
		
		
		\subsubsection{绘制一个椭圆}
		
		\[
		x(\theta) = a \cdot \cos(\theta)
		\]
		\[
		y(\theta) = b \cdot \sin(\theta)
		\]
		\subsubsection{植被覆盖度计算}
		在遥感研究中,植被覆盖度是衡量地表植被生长状况的重要指标,通常通过归一化植被指数(NDVI)来计算。在本研究中,植被覆盖度 \( C_i \) 通过对 NDVI 数据进行归一化处理来估算。具体来说,植被覆盖度的计算公式如下:
		
		\[
		C_i = \frac{\text{NDVI}_{i} - \text{NDVI}_{\text{min}}}{\text{NDVI}_{\text{max}} - \text{NDVI}_{\text{min}}}
		\]
		
		其中,\( \text{NDVI}_{i} \) 表示在某一时刻或某个像素位置的 NDVI 值,\( \text{NDVI}_{\text{min}} \) 是在选定时间范围内研究区域所有像素的最小 NDVI 值,而 \( \text{NDVI}_{\text{max}} \) 是该区域内所有像素的最大 NDVI 值。该公式通过将每个像素的 NDVI 值与该区域内的最小值和最大值进行归一化,从而将 NDVI 值转换为一个介于 0 和 1 之间的比例值,表示该像素的植被覆盖度。
		
		归一化后的植被覆盖度 \( C_i \) 值的范围从 0 到 1,其中 \( C_i = 0 \) 表示该区域没有植被或植被极其稀少,而 \( C_i = 1 \) 则表示该区域的植被覆盖度达到了最大值。该方法能够有效地消除不同时间和地点 NDVI 值的差异,使得植被覆盖度在不同区域之间具有可比性。通过该归一化过程,植被覆盖度的估算不仅可以反映出土地表面植被的生长状况,还能够为生态监测、环境评估以及土地利用变化分析等研究提供重要的参考数据。
		
		在遥感图像处理时,首先创建水体和永久冰雪区域的掩膜。水体掩膜是通过土地覆盖分类图中的值为 17 的像素生成,而永久冰雪掩膜则是通过值为 15 的像素生成。通过合并水体和冰雪区域的掩膜,可以得到一个新的掩膜,表示水体或永久冰雪区域。接着,通过反转这个掩膜,去除水体和冰雪以外的区域,从而得到去除了水体和冰雪的地物类别图。
		
		使用这种方法,能够提取并研究没有水体和冰雪覆盖的地表区域,从而进行进一步的植被覆盖度分析。植被覆盖度的计算公式如上所示,将每个像素的 NDVI 值归一化,使得最终结果能够反映该区域的植被生长状况,并且能够比较不同时间或不同区域的植被变化。
				
		\subsubsubsection{操作步骤}
		
		\par
		首先,脚本导入了 arcpy 和空间分析模块 arcpy.sa,这两个库是ArcGIS环境中进行栅格数据处理的核心工具。接着,设置了 arcpy.env.overwriteOutput 为 True,这意味着如果输出文件已存在,程序将覆盖该文件。然后,定义了多个路径变量,包括输入栅格数据所在文件夹路径(input\_raster\_folder),掩膜矢量文件的路径(mask\_shapefile),以及两个输出文件夹路径(output\_raster\_folder 和 wgs84\_raster\_folder),分别用于存放裁剪后的栅格数据和转换为WGS84坐标系后的栅格数据。

		在创建输出文件夹的过程中,脚本首先检查输出文件夹是否存在,如果不存在,则使用 os.makedirs() 函数创建该文件夹。接下来,脚本定义了 WGS84 坐标系(EPSG:4326),这是后续进行栅格投影转换时需要使用的空间参考系统。
		
		接下来,脚本进入一个循环,遍历输入文件夹中的所有 TIFF 文件。对于每一个文件,首先检查文件扩展名是否为 .tif,确保仅处理 TIFF 格式的栅格数据。如果是 TIFF 文件,脚本将构建该文件的完整路径,并为裁剪后的栅格数据定义一个新的输出路径。
		
		在处理每一个栅格文件时,首先执行 ExtractByMask 操作,这个操作将栅格数据裁剪为与提供的掩膜矢量文件(mask\_shapefile)重叠的部分。裁剪后的栅格将保存到定义的输出路径下。如果裁剪操作成功,程序会打印一条信息,说明裁剪完成,并指明输出文件的位置。
		
		接下来,进行栅格数据的重分类。脚本使用 arcpy.sa.Reclassify() 函数对裁剪后的栅格数据进行值的重分类。具体的重分类规则是基于原始栅格数据的像素值,将其划分为六个不同的类别,分别为“水体和冰雪”、“极低覆盖度”、“低覆盖度”、“中覆盖度”、“中高覆盖度”和“高覆盖度”。这些类别对应着不同的数值范围,并被映射到新的整数值上。重分类后的栅格被保存到与裁剪结果相同的输出路径中。
		
		接下来,脚本在栅格的属性表中添加一个名为“Cover\_Type”的字段,用于存储每个像素的分类信息。通过 arcpy.da.UpdateCursor,脚本逐一遍历栅格数据中的每个像素,检查其分类值,并根据预设的分类规则为每个像素赋值。例如,当像素值为1时,分类为“水体和冰雪”;像素值为2时,分类为“极低覆盖度”;以此类推。更新完成后,脚本会打印一条信息,说明重分类和属性表更新已经完成。
		
		最后,脚本进行栅格数据的投影转换。原始数据使用的坐标系可能与WGS84坐标系不同,因此通过 arcpy.management.ProjectRaster() 函数,将重分类后的栅格数据重新投影到WGS84坐标系(EPSG:4326)。投影转换后的栅格会被保存到另一个指定的输出文件夹(wgs84\_raster\_folder)中。转换完成后,脚本会打印一条信息,确认投影转换已成功完成。
		
		\subsubsubsection{甘肃省的植被覆盖度}
		\par
			具体来说,原始值为 -1 的像素被重分类为 1,表示水体和冰雪区域;原始值在 0 到 0.3 范围内的像素被重分类为 2,表示极低覆盖度;原始值在 0.3 到 0.6 范围内的像素被重分类为 3,表示低覆盖度;原始值在 0.6 到 0.8 范围内的像素被重分类为 4,表示中覆盖度;原始值在 0.8 到 0.9 范围内的像素被重分类为 5,表示中高覆盖度;原始值在 0.9 到 1 范围内的像素被重分类为 6,表示高覆盖度。
			
			通过这种方式,原始的像素值被替换成代表不同土地覆盖类型或植被覆盖程度的新值。这些新的值使得栅格数据更具可读性,也便于进一步的分类分析。
		
		\begin{figure}[H]  % 使用 [H] 来强制图片在此处显示
			\centering
			\includegraphics[width=0.8\textwidth, keepaspectratio]{pic/gansu_fill_degrees.png} % 自动保持纵横比
		\end{figure}
		
		
		\par
		
		为了研究甘肃省不同土地覆盖类型的植被覆盖度变化,我们使用了 Google Earth Engine (GEE) 平台,通过 MODIS 数据集来提取不同土地类型的植被覆盖度指数(Ci)。首先,我们定义了一个包含甘肃省区域的 Shapefile,并指定了研究的土地覆盖类型,包括草地(Grassland)、荒地(Barrenland)、落叶阔叶林(Deciduous Broadleaf Forest)和耕地(Cropland)。在代码中,我们将这些土地类型与对应的 ID 进行了关联,并创建了一个空的字典来存储每一年的结果。然后,我们定义了研究的时间范围,选择了从 2001 年到 2020 年的数据。
		
		接下来,我们为每一年加载 MODIS 土地覆盖数据,并从中提取出 ``LC\_Type1'' 波段,表示土地类型分类。为了避免水体和永久冰雪影响分析,我们创建了水体(值为 17)和永久冰雪(值为 15)的掩膜,并将其从土地覆盖数据中去除。这样,我们得到了去除水体和冰雪后的土地覆盖图像。
		
		为了计算植被覆盖度(Ci),我们从 MODIS 的 NDVI 数据集中提取每年 4 月至 10 月期间的数据。我们选择 NDVI 数据波段并将其裁剪到甘肃省的研究区域,同时应用去除水体和冰雪区域的掩膜。接下来,通过计算 NDVI 数据的最大值和最小值,我们对 NDVI 值进行归一化处理,得到每个像素的植被覆盖度(Ci)。
		
		对于每一年的数据,我们遍历所有定义的土地覆盖类型,通过对每个土地类型的 Ci 值进行掩膜处理,计算其平均植被覆盖度(Ci)。使用 ``reduceRegion'' 函数,我们在指定的区域内对每个土地类型的植被覆盖度进行平均计算,并将其值存储到字典中。所有年度的结果最终被合并为一个表格,其中每一行代表某一年在不同土地覆盖类型下的平均植被覆盖度(Ci)。
		
		最后,我们将结果表格打印输出,并提供了一个导出功能,允许用户将数据以 CSV 格式保存到 Google Drive 上。导出的文件名为 ``Annual\_gansu\_ci'',并存储在名为 ``Annual\_ci'' 的文件夹中。

		这张表展示了甘肃省在 2001 至 2020 年间四种主要土地类型——荒地、耕地、落叶阔叶林和草地——的植被覆盖度(Ci)随时间变化的情况。每一行代表一个年份,而每一列则对应不同土地类型的平均植被覆盖度值。表中的数据反映了不同土地类型在各个年份中的植被覆盖度变化趋势。总体上,耕地的植被覆盖度在这些年里呈现出一定的波动,尤其在2007年后有明显增加,而草地的植被覆盖度则保持在较为稳定的水平,略有波动。落叶阔叶林的植被覆盖度普遍较高,且在大部分年份中呈现逐年上升的趋势。荒地的植被覆盖度变化较为波动,2005年以后有一定的下降。
		
		\begin{table}[H]
			\centering
			\begin{tabular}{|c|c|c|c|c|}
				\hline
				年份 & 荒地 & 耕地 & 落叶阔叶林 & 草地 \\
				\hline
				2001 & 0.511941 & 0.570904 & 0.704765 & 0.500067 \\
				2002 & 0.518370 & 0.528856 & 0.700443 & 0.510193 \\
				2003 & 0.536400 & 0.556638 & 0.708005 & 0.512364 \\
				2004 & 0.500902 & 0.541634 & 0.713605 & 0.512249 \\
				2005 & 0.496261 & 0.550009 & 0.713762 & 0.511970 \\
				2006 & 0.511710 & 0.553534 & 0.721376 & 0.504922 \\
				2007 & 0.504281 & 0.641827 & 0.764243 & 0.524006 \\
				2008 & 0.510210 & 0.540595 & 0.718590 & 0.515356 \\
				2009 & 0.499330 & 0.558798 & 0.726750 & 0.499441 \\
				2010 & 0.526189 & 0.566965 & 0.684029 & 0.531827 \\
				2011 & 0.504926 & 0.561961 & 0.727337 & 0.506348 \\
				2012 & 0.498815 & 0.565106 & 0.690385 & 0.514980 \\
				2013 & 0.486834 & 0.557065 & 0.704142 & 0.499227 \\
				2014 & 0.489472 & 0.586328 & 0.710333 & 0.521978 \\
				2015 & 0.487478 & 0.558841 & 0.734029 & 0.513624 \\
				2016 & 0.474402 & 0.577359 & 0.750653 & 0.506615 \\
				2017 & 0.491523 & 0.591081 & 0.698460 & 0.529953 \\
				2018 & 0.470034 & 0.589964 & 0.718601 & 0.510201 \\
				2019 & 0.505686 & 0.600033 & 0.730242 & 0.548370 \\
				2020 & 0.501887 & 0.616389 & 0.760607 & 0.530801 \\
				\hline
			\end{tabular}
			\caption{甘肃省不同土地类型的植被覆盖度随时间变化}
			\label{tab:vegetation_coverage}
			
			
			
		\end{table}
				\begin{figure}[H]  % 使用 [H] 来强制图片在此处显示
			\centering
			\includegraphics[width=0.8\textwidth, keepaspectratio]{pic/gansu_ci.png} % 自动保持纵横比
		\end{figure}
		\subsubsubsection{新疆地区的植被覆盖度}
		
		\par
		重分类规则包括将原始像素值按照以下方式进行转换:首先,值为 -1 的像素被重分类为 1,代表水体和冰雪区域;原始值在 0 到 0.2 范围内的像素被重分类为 2,表示极低覆盖度;值在 0.2 到 0.3 范围内的像素被重分类为 3,代表低覆盖度;接下来,值在 0.3 到 0.5 范围内的像素被重分类为 4,代表中覆盖度;值在 0.5 到 0.8 范围内的像素被重分类为 5,表示中高覆盖度;最后,值在 0.8 到 1 范围内的像素被重分类为 6,表示高覆盖度。通过这样的分类规则,原始栅格数据的每个像素都被分配到一个新的类别,这些类别通常代表不同的地物类型或土地覆盖程度。
		
		\begin{figure}[H]  % 使用 [H] 来强制图片在此处显示
			\centering
			\includegraphics[width=0.8\textwidth, keepaspectratio]{pic/xinjiang_fill_degrees.png} % 自动保持纵横比
		\end{figure}
		
		\subsubsubsection{青海省的植被覆盖度}
		\par 将值为 -1 的像素重分类为类别 1,代表水体和冰雪;将覆盖度值在 0 到 0.2 之间的像素重分类为类别 2,代表极低覆盖度;将覆盖度值在 0.2 到 0.3 之间的像素重分类为类别 3,代表低覆盖度;将覆盖度值在 0.3 到 0.5 之间的像素重分类为类别 4,代表中覆盖度;将覆盖度值在 0.5 到 0.8 之间的像素重分类为类别 5,代表中高覆盖度;将覆盖度值在 0.8 到 1 之间的像素重分类为类别 6,代表高覆盖度。
		\begin{figure}[H]  % 使用 [H] 来强制图片在此处显示
			\centering
			\includegraphics[width=0.8\textwidth, keepaspectratio]{pic/qinghai_fill_degrees.png} % 自动保持纵横比
		\end{figure}	
		
		\subsubsubsection{四川省的植被覆盖度}


		\par 重分类规则是通过一组范围定义的。首先,值为 -1 的像素被映射为类别 1,表示水体和冰雪区域;接下来,覆盖度值在 0 到 0.2 之间的像素被映射为类别 2,表示极低覆盖度区域;覆盖度值在 0.2 到 0.7 之间的像素被映射为类别 3,代表低覆盖度区域;值在 0.7 到 0.8 之间的像素被映射为类别 4,代表中覆盖度区域;值在 0.8 到 0.9 之间的像素被映射为类别 5,代表中高覆盖度区域;最后,值在 0.9 到 1 之间的像素被映射为类别 6,代表高覆盖度区域。这些类别分别对应了不同的地表覆盖类型,提供了一个从低到高的覆盖度分类。
		\begin{figure}[H]  % 使用 [H] 来强制图片在此处显示
			\centering
			\includegraphics[width=0.8\textwidth, keepaspectratio]{pic/sichuan_fill_degrees.png} % 自动保持纵横比
		
		
		\end{figure}

		\subsubsubsection{内蒙古自治区的植被覆盖度}
		
		
		\par 在这个重分类规则中,每个数值范围对应一个新的类别。首先,栅格中值为 -1 的像素被重新分类为 1,代表水体、冰川和积雪区域;接着,像素值在 0 到 0.3 之间的区域被分类为 2,表示极低覆盖度;像素值在 0.3 到 0.5 之间的区域被分类为 3,代表低覆盖度区域;值介于 0.5 到 0.7 之间的像素被映射为 4,表示中覆盖度区域;像素值在 0.7 到 0.8 之间的被分类为 5,代表中高覆盖度区域;最后,像素值在 0.8 到 1 之间的区域被分类为 6,表示高覆盖度区域。
		\begin{figure}[H]  % 使用 [H] 来强制图片在此处显示
			\centering
			\includegraphics[width=0.8\textwidth, keepaspectratio]{pic/neimeng_fill_degrees.png} % 自动保持纵横比
		\end{figure}
		\par	表格展示了内蒙古自治区在2001至2020年间不同土地类型的植被覆盖度变化情况。数据包括了四种土地类型:荒地、耕地、草地和木本稀树草原。荒地的植被覆盖度在2001年为0.4924,经历了波动,在2020年降至0.4728。耕地的植被覆盖度则相对较为稳定,从2001年的0.4401上升至2020年的0.4773。草地的覆盖度波动较大,2001年为0.4185,最高达到0.4840(2015年),最后在2020年为0.4590。木本稀树草原的植被覆盖度整体上呈上升趋势,从2001年的0.5771增至2020年的0.5429,尽管在部分年份出现短期下降。整体来看,四种土地类型的植被覆盖度在20年间呈现出不同的变化趋势,反映了区域内自然环境及土地使用变化的动态过程。

		\begin{table}[H]
			\centering
			\begin{tabular}{|c|c|c|c|c|}
				\hline
				年份 & 荒地 & 耕地 & 草地 & 木本稀树草原 \\
				\hline
				2001 & 0.4924 & 0.4401 & 0.4185 & 0.5771 \\
				2002 & 0.5306 & 0.4537 & 0.4382 & 0.6046 \\
				2003 & 0.5223 & 0.4414 & 0.4551 & 0.5337 \\
				2004 & 0.4858 & 0.4657 & 0.4384 & 0.6367 \\
				2005 & 0.4995 & 0.4510 & 0.4347 & 0.5824 \\
				2006 & 0.4877 & 0.4436 & 0.4054 & 0.6060 \\
				2007 & 0.4909 & 0.4400 & 0.4344 & 0.6109 \\
				2008 & 0.4692 & 0.4308 & 0.4245 & 0.5347 \\
				2009 & 0.5052 & 0.4510 & 0.4599 & 0.5797 \\
				2010 & 0.5239 & 0.4522 & 0.4783 & 0.6419 \\
				2011 & 0.5207 & 0.4212 & 0.4311 & 0.5667 \\
				2012 & 0.4896 & 0.4674 & 0.4515 & 0.5897 \\
				2013 & 0.5103 & 0.4870 & 0.4828 & 0.6492 \\
				2014 & 0.4882 & 0.4674 & 0.4674 & 0.5522 \\
				2015 & 0.5267 & 0.4661 & 0.4840 & 0.5863 \\
				2016 & 0.5009 & 0.4675 & 0.4802 & 0.6166 \\
				2017 & 0.4840 & 0.4506 & 0.4295 & 0.5607 \\
				2018 & 0.4512 & 0.4683 & 0.4371 & 0.5593 \\
				2019 & 0.4898 & 0.4696 & 0.4793 & 0.5780 \\
				2020 & 0.4728 & 0.4773 & 0.4590 & 0.5429 \\
				\hline
			\end{tabular}
			\caption{内蒙古自治区2001至2020年不同土地类型的植被覆盖度变化}
			\label{tab:vegetation_coverage}
		\end{table}
		
			\begin{figure}[H]  % 使用 [H] 来强制图片在此处显示
			\centering
			\includegraphics[width=0.8\textwidth, keepaspectratio]{pic/neimeng_ci.png} % 自动保持纵横比
		\end{figure}
		
		
		\subsubsubsection{西藏自治区的植被覆盖度}
		
				
		\par
		重分类规则通过 arcpy.sa.RemapRange 设定,其中每一对范围表示原始栅格数据的像素值区间及其对应的新分类值。通过这些规则,将原始栅格数据中的每个像素值映射到一个新的整数值,代表不同的地表覆盖度类别。首先,栅格中值为 -1 的像素被分类为 1,代表水体、冰川或积雪区域;接着,栅格中值在 0 到 0.2 之间的像素被重新分类为 2,代表极低覆盖度区域;值在 0.2 到 0.7 之间的像素被分类为 3,表示低覆盖度区域;接着,值在 0.7 到 0.8 之间的像素被分类为 4,表示中覆盖度区域;值在 0.8 到 0.9 之间的像素被分类为 5,代表中高覆盖度区域;最后,值在 0.9 到 1 之间的像素被分类为 6,代表高覆盖度区域。通过这种方式,重分类操作将原始栅格数据中的像素值根据覆盖度高低进行合理的划分,便于后续分析和处理。
		\begin{figure}[H]  % 使用 [H] 来强制图片在此处显示
			\centering
			\includegraphics[width=0.8\textwidth, keepaspectratio]{pic/xizang_fill_degrees.png} % 自动保持纵横比
		\end{figure}
		
		
		
		\subsubsection{不同土地类型的植被覆盖度比较}

		
		
		\subsubsection{多年的植被覆盖度变化分析}
		
		
		
		
		
		
		
		\subsubsection{植被指数计算}
		
				
		在本例中,使用了美国地质调查局(USGS)的SRTM 30米分辨率数字高程模型(DEM)数据集。
		在遥感影像处理中,归一化植被指数(NDVI)是通过计算红光(B4波段)和近红外(B5波段)的比值来估计植被覆盖度的常用指数。NDVI的计算公式如下:
		\[
		NDVI = \frac{NIR - Red}{NIR + Red}
		\]
		
		其中:
		- \( NIR \) 表示近红外波段的像素值(Landsat 8的B5波段),
		- \( Red \) 表示红光波段的像素值(Landsat 8的B4波段)。
		
		此公式通过对红光和近红外波段的反射率差异进行归一化,得到了一个介于 -1 到 1 之间的数值,通常用于评估植被覆盖度:
		- \( NDVI > 0 \) 表示有绿色植被,值越高植被越茂盛,
		- \( NDVI < 0 \) 通常表示无植被的区域(如沙漠或水体),
		- \( NDVI = 0 \) 表示土壤或其他不含绿植的区域。
		
		在遥感影像处理中,增强植被指数(EVI)是一种常用的植被指数,能够有效抑制大气干扰并提高对植被变化的敏感性。EVI的计算公式如下:
		
		\[
		EVI = 2.5 \times \frac{NIR - Red}{NIR + 2.4 \times Red + 1}
		\]
		
		其中:
		- \( NIR \) 表示近红外波段的像素值(Landsat 8的B5波段),
		- \( Red \) 表示红光波段的像素值(Landsat 8的B4波段)。
		
		此公式通过对红光和近红外波段的反射率进行加权运算,计算出一个数值,表示植被的生长情况,EVI的值通常大于0,值越大表示植被越茂盛。
		\par
		地上生物量(AGB,Above Ground Biomass)是指植物地面部分的生物质总量,通常包括树木、草本植物、灌木等的茎、叶和根部分的生物量。它是衡量植物生长和生态系统健康的重要指标之一,广泛用于生态学、气候变化研究以及碳循环评估。AGB 的大小与植物的生长速度、土壤肥力、气候条件等多种因素密切相关。在森林生态系统中,AGB 是碳储存的重要组成部分,对全球气候变化的缓解具有重要影响。植物通过光合作用吸收二氧化碳并将其转化为有机物质,从而在地上生物量的形式中储存碳。因此,研究和监测 AGB 对于了解全球碳循环、评估气候变化的影响以及制定相应的环境保护措施都至关重要。随着遥感技术的发展,全球范围内的地上生物量数据可以通过卫星观测获得,从而为科学家提供了大规模、长期的监测手段。这些数据有助于在全球尺度上理解和量化森林和其他植被类型的碳存储能力,对应对气候变化具有重要的理论和实践意义。
		\subsubsection{地理特征计算}
		
		在地理信息系统中,坡度(Slope)是指地表某一点的陡峭程度。坡度可以通过数字高程模型(DEM)计算得到,公式如下:
		
		\[
		S = \arctan \left( \sqrt{ \left( \frac{\partial z}{\partial x} \right)^2 + \left( \frac{\partial z}{\partial y} \right)^2 } \right)
		\]
		
		其中:
		- \( S \) 代表坡度,
		- \( z \) 是高程值,
		- \( \frac{\partial z}{\partial x} \) 和 \( \frac{\partial z}{\partial y} \) 分别表示高程在水平 \( x \) 和垂直 \( y \) 方向上的梯度。

		
		坡向(Aspect)是地形的一项重要特征,表示地面某一点的朝向,即地表最陡坡的方向。坡向的计算通常基于数字高程模型(DEM)数据。设定坡向公式如下:
		
		\[
		A = \arctan \left( \frac{ \frac{\partial z}{\partial x} }{ \frac{\partial z}{\partial y} } \right)
		\]
		
		其中:
		- \( A \) 是坡向,单位通常为角度(以度表示)。
		- \( z \) 是高程值。
		- \( \frac{\partial z}{\partial x} \) 和 \( \frac{\partial z}{\partial y} \) 分别表示高程在水平方向(x方向)和垂直方向(y方向)上的梯度。
		
		
	\section{第二章 技术路线}
	
	
	\subsection{地理信息处理方式}

	本文的地理信息处理方式依赖于GDAL库。
	GDAL(Geospatial Data Abstraction Library)是一个开源的地理空间数据处理库,广泛应用于地理信息系统(GIS)领域,具有强大的数据读写和处理功能。GDAL支持多种栅格和矢量数据格式,如GeoTIFF、Shapefile、KML、GeoJSON等,使其能够在不同的数据格式之间进行转换和处理,这为数据的集成和分析提供了极大的便利。此外,GDAL的跨平台特性使得它能够在Windows、Linux、macOS等多种操作系统上运行,确保了它的广泛适用性。GDAL还提供了高效的性能,特别是在处理大规模数据时,它能快速读取、写入以及进行各种数据处理操作,如投影转换、栅格运算和矢量数据的操作等。
	
	在C语言中调用GDAL时,首先需要安装并包含GDAL的头文件。通过使用GDALAllRegister()函数注册数据驱动后,可以使用GDALOpen()函数打开文件,并通过GDALGetRasterBand()等API访问数据。处理完数据后,需要使用GDALClose()释放资源。C语言的GDAL接口功能全面,适合性能要求高、底层控制需求强的应用。
	
	在Python中调用GDAL相对简单,首先需要通过pip安装GDAL库,然后通过from osgeo import gdal导入相关模块。使用Python的gdal.Open()函数打开文件后,可以通过GetRasterBand()访问栅格数据,并使用ReadAsArray()等函数获取像素值。Python接口具有更为友好的语法和开发效率,适合快速开发和原型设计,但依然保留了GDAL的强大功能。无论是C语言还是Python,GDAL都能提供强大的地理空间数据处理能力,满足不同开发需求。


	\subsection{前端和后端}
	在前后端开发和集成学习的整合中,可以实现高效的数据处理、模型训练和预测结果展示。
	
	在数据预处理阶段,后端服务器会对用户上传的数据进行清洗、格式化和特征提取。这些预处理步骤对于保证模型预测的准确性至关重要。完成预处理后,数据被传递给集成学习模型进行预测。集成学习模型可以由多种机器学习算法组成,如随机森林、XGBoost和神经网络等。
	
	集成学习的核心在于结合多个弱学习器的预测结果以提高整体模型的性能。通常的方法包括Bagging、Boosting和Stacking。在Bagging方法中,多个模型并行训练,最终预测结果通过平均或投票的方式决定。Boosting则是通过逐步调整模型权重,关注前一阶段预测错误的数据,提高整体模型的准确性。Stacking是一种更为复杂的方法,通过训练一个元模型来组合多个初级模型的输出。
	
	训练完成的模型可以保存到文件系统或数据库中,以便后续的快速加载和更新。每次用户发起预测请求时,后端服务器会加载最新的模型进行预测。预测结果经过处理后,通过API返回给前端,前端将结果以可视化的形式展示给用户。
	
	通过这种技术路线,前后端和集成学习的整合不仅提高了数据处理和模型预测的效率,还提升了用户体验。前端提供了直观的交互界面,后端确保了数据处理和模型训练的可靠性,集成学习则增强了模型的预测性能。这种整合方法在实际应用中具有广泛的潜力,特别是在需要高精度预测的场景下。
	前端方便了数据的展示和内容的拆分与维护。前端指的是浏览器的显示的内容,通过使用js技术可以在网页上做出许多美观实用的图片,也可以在前端完成用户的交互工作,比如说下载图片、下载图片等。本文拟采用react完成直观的交互,比如说数据的上传、下载,结果的展示和下载等。React 是一个用于构建用户界面的 JavaScript 库。它由 Facebook 开发并开源。
	它主要专注于构建单页面应用程序(SPA),通过组件化的方式提高了代码的可复用性和可维护性。 React 的核心思想是组件化开发,将用户界面拆分为独立的组件,每个组件负责管理自己的状态和渲染逻辑。 React 的另一个显著特点是虚拟 DOM(Virtual DOM)。它通过在内存中维护一个虚拟 DOM 树来实现高效的 DOM 更新,通过比较前后两次虚拟 DOM 的差异,最小化了实际 DOM 操作的次数,从而提升了性能。 React 不仅可以用于 Web 应用程序的开发,还可以用于移动应用程序开发以及静态网站的生成。由于其灵活性和高效性,React 在现代前端开发中得到了广泛应用,并成为了构建复杂用户界面的首选工具之一。
	后端指的是网页后台中配合前端完成数据处理,和保存到数据库的一系列内容,常用的后端有Java开发的spring 系列后端 ,js后端node.js,和python的fastapi。在本文中,后端开发使用Python的FastAPI框架来构建RESTful API服务。FastAPI 是一个现代、快速(高性能)、基于标准 Python 类型提示的 Web 框架,用于构建 APIs,采用了 Python 3.6+ 版本。fastapi具有与 Node.js 和 Go 相媲美的高性能,因为它基于 Starlette 和 Pydantic 这两个高性能工具。在实际应用中,FastAPI 用于处理前端发送的数据请求,进行数据预处理并调用集成学习模型进行预测。
	
		
	\subsection{GEE平台}
	Google Earth Engine (GEE) 是一个基于云计算的地理空间分析平台,最初由 Google 推出,旨在为全球范围内的遥感数据和地理信息提供高效、便捷的分析工具。自从 GEE 于 2011 年推出以来,得到了广泛应用,特别是在大规模地理空间数据处理和分析方面,具有显著的优势。GEE 的出现标志着地理空间数据分析进入了云计算时代,突破了传统计算能力的局限,使得全球范围内的遥感数据处理变得更为快速和高效。
	
	GEE 的核心优势在于其提供了海量的遥感数据资源,包括 Landsat、MODIS、Sentinel 等卫星影像,以及其他各类气候、环境、土地利用等相关数据。它能够对这些海量数据进行在线处理、分析和可视化,而无需依赖传统的本地计算机资源。这大大简化了数据处理流程,节省了计算和存储成本。
	
	使用 GEE 进行遥感数据分析的一个显著特点是其强大的云计算能力。研究人员可以利用 GEE 平台处理和分析全球尺度的遥感数据,进行土地覆盖变化、城市扩展、森林监测等广泛的应用研究。相较于传统的遥感数据分析方法,GEE 通过云端处理和分析,使得数据的获取、处理、分析和结果输出变得更加高效和便捷。
	
	在 GEE 平台上,用户不仅可以使用多种预先集成的遥感数据集,还可以编写自己的算法脚本进行定制化分析。这使得 GEE 平台在遥感监测、环境变化研究、气候变化评估等领域的应用极其广泛。例如,基于 GEE 的全球森林变化监测研究已成为当前遥感研究中的重要方向,通过结合多时相遥感影像,研究人员能够实时监测全球森林资源的变化情况。
	
	此外,GEE 在土地利用/土地覆盖变化监测方面具有重要应用。研究人员利用 GEE 可以基于中高分辨率的遥感影像,对全球或区域范围内的土地利用/覆盖进行精细化的变化检测与分类分析,帮助政府和科研机构制定更加合理的土地利用政策与环境保护措施。特别是 GEE 提供的长期时间序列数据,使得大规模的时空动态变化分析成为可能。
	
	总结来说,GEE 是一个功能强大的遥感云计算平台,凭借其丰富的遥感数据资源和强大的云计算能力,成为全球范围内地理空间数据分析、环境监测和土地管理等领域的重要工具。通过 GEE,研究人员能够更加高效地进行大范围、长时间序列的地理数据分析,为全球可持续发展目标的实现提供科学支持。
	例如,Hamud 等(2018)利用 GEE 结合支持向量机和随机森林分类器的监督分类方法,监测了索马里巴纳迪尔地区1989至2018年的城市扩张和土地利用变化 \cite{Hamud2018}。该研究展示了 GEE 平台在大尺度遥感影像分类和土地利用变化研究中的应用潜力,尤其是在时间跨度较长、数据量巨大的情况下,GEE 的高效处理能力显得尤为重要。
	
	Carneiro 等(2020)基于 GEE 平台开展了巴西特雷西纳—帝蒙城市群1985至2019年的时空扩张研究 \cite{Carneiro2020}。研究利用 GEE 提供的 Landsat 数据和时间序列分析技术,对城市化进程进行动态监测,揭示了城市化对当地环境的影响及其变化趋势。该研究不仅突显了 GEE 在城市扩张监测中的优势,还展示了其在时空动态变化研究中的强大能力。
	
	此外,Akinyemi 等(2021)基于 GEE 和 Landsat 数据,结合支持向量机分类器和光谱—时间分隔算法,开展了1987至2019年卢旺达基加利城市土地覆盖变化研究 \cite{Akinyemi2021}。该研究利用 GEE 对大规模遥感数据进行处理,探讨了城市化进程对生态环境的影响,特别是在城市扩张和土地覆盖变化的监测上,提供了重要的参考数据和分析结果。
	
	刘小平等(2017)通过 GEE 和 Landsat 数据结合城市用地综合指数(NUACI)方法,绘制了1985至2015年全球城市动态图 \cite{Liu2017}。该研究利用 GEE 的全球尺度遥感数据和强大的云计算能力,分析了全球城市化的时空变化,并提出了适用于大区域城市用地监测的有效方法。
	\subsection{机器学习方法}
	
	\subsubsection{随机森林}
	随机森林(Random Forest)是一种集成学习方法,它通过构建多个决策树并结合其结果来进行分类或回归任务。该算法由Leo Breiman在2001年提出,旨在通过降低模型的方差来提高预测的准确性和鲁棒性。它通过构建多个决策树并将其结果进行结合,形成一个“森林”。每棵树是在随机抽样的训练数据上生成的,通常采用自助采样(Bootstrap Sampling)技术,即在训练集上随机抽取样本并放回,这样每棵树的训练数据略有不同,从而增强模型的多样性。此外,在每棵树的构建过程中,随机森林还会在每个节点上随机选择特征进行分裂,这进一步减少了树之间的相关性,有助于降低过拟合风险。该算法通过随机选取训练数据的子集和特征来生成每棵树,从而降低各棵树之间的相关性,提高模型的鲁棒性和准确性。随机森林具有良好的抗过拟合能力和较高的泛化性能,特别适用于处理高维数据和缺失值。它能够自动处理大规模数据集,并提供特征重要性评估,帮助理解和解释模型的决策过程。此外,随机森林易于并行化,能够有效利用现代计算资源。
	随机森林通过随机选取训练数据的子集和特征来生成每棵树,使得各棵树之间的相关性降低,从而提升整体模型的性能。每棵决策树独立生长,且不会进行修剪,最终通过多数表决或平均值来汇总各个树的预测结果。随机森林具有良好的抗过拟合能力和较高的泛化性能,尤其在处理高维数据和缺失值时表现优异。其主要优势在于能够自动处理大规模数据集,并提供特征重要性评估,帮助理解和解释模型的决策过程。

	
	随机森林易于并行化,能够有效利用现代计算资源,广泛应用于金融、医学、市场营销、图像识别等多个领域,能够处理各种类型的数据,包括数值型和分类型数据。在 Python 中,利用 scikit-learn 库可以方便地实现随机森林模型,用户只需指定树的数量和其他参数,即可训练和评估模型。总之,随机森林凭借其高效的性能和广泛的应用场景,成为了机器学习领域的重要工具之一。
	随机森林的优点在于其抗过拟合能力和高准确性。通过结合多个决策树的预测结果,随机森林通常能够提供更为稳定和准确的预测。同时,它还可以评估特征的重要性,使得特征选择过程更加直观。然而,随机森林也存在一些缺点,例如模型复杂性较高,训练和预测的时间成本相对较大,并且相比单棵决策树,其可解释性较差,难以直观理解模型的决策逻辑。
	例如,Liu 等(2015)利用随机森林算法对中国东北地区的土地利用/覆盖进行分类,并取得了较高的分类精度。该研究采用遥感影像数据,结合随机森林的强大特性,在复杂的地理环境下成功应用于大范围的土地覆盖分类,验证了随机森林在处理高维度特征数据时的优势\cite{liu2015}。该研究进一步强调了随机森林在遥感数据处理中的可靠性,尤其在分类精度要求较高的场景下。
	
	另一个相关研究是Li 等(2016)在印度尼西亚的热带雨林地区进行的研究。该研究使用随机森林算法对植被类型进行了详细分类,结果表明,随机森林能够有效处理复杂的遥感影像数据并克服了传统分类方法中存在的过拟合问题\cite{li2016}。研究中还结合了地形数据、气候数据和遥感影像数据,以提高模型的分类准确性,展示了随机森林在生态系统监测和土地利用研究中的应用潜力。
	
	此外,Gislason 等(2006)应用随机森林算法进行冰岛土地覆盖分类,提出了随机森林在处理遥感数据中的优势,尤其是在大尺度遥感影像数据的处理上,能够有效减少噪声对分类结果的影响,并提高了分类模型的稳健性\cite{gislason2006}。该研究的成功应用进一步证明了随机森林算法在遥感影像分类中的强大能力,尤其是在地形复杂、数据量庞大的环境下。
	
	在城市扩展研究方面,Zhang 等(2017)利用随机森林对中国上海市的城市土地利用变化进行了研究,展示了该算法在高分辨率遥感影像分类中的表现\cite{zhang2017}。通过引入地理空间特征和时间序列数据,研究有效捕捉了城市化进程中的细微变化,强调了随机森林在时空变化监测中的广泛应用。
	
	\subsubsection{xgboost}
	XGBoost 是一种基于决策树的机器学习算法,因其速度和性能在处理大规模数据和复杂问题时非常受欢迎。它的优势在于强大的计算效率和高精度,这得益于其内置的并行计算和对硬件的优化。XGBoost 通过梯度提升技术逐步减少误差,能够很好地处理分类和回归任务。此外,它还支持特征的自动化选择和缺失值处理,增强了模型的鲁棒性。
	XGBoost通过优化损失函数和增加正则化项,减少过拟合风险,同时加速模型的训练过程。其关键特性之一是实现了并行计算,使得在处理大规模数据集时具有显著的速度优势。此外,XGBoost还提供了多种灵活的参数配置,使得用户能够针对具体问题进行调优。该模型不仅适用于分类和回归任务,还在许多机器学习竞赛中表现出色,成为数据科学家和机器学习工程师的热门选择。由于其强大的性能和灵活性,XGBoost广泛应用于金融、医疗、广告和推荐系统等多个领域,是现代机器学习中不可或缺的重要工具。
	然而,XGBoost 也有一些缺点。首先,它相较于其他简单的模型如线性回归,需要更多的时间和资源来训练,尤其是当数据量非常大时。其次,XGBoost 的超参数调优较为复杂,错误的设置可能导致模型表现不佳。此外,尽管 XGBoost 的强大性能在大多数情况下表现出色,但它的解释性较差,不如简单模型容易解读。对于某些问题,XGBoost 的复杂性可能带来过拟合风险,尤其是当训练数据的规模和质量不足时。
	
	
	总体来说,XGBoost 非常适合高维度数据集和需要高精度的应用场景,但在某些情况下可能需要平衡其复杂性与可解释性。
		例如,Liu 等(2018)在其研究中使用了XGBoost算法对中国的土地覆盖进行分类,并取得了显著的结果\cite{liu2018}。研究通过结合遥感影像和多维特征数据,应用XGBoost算法进行土地利用变化监测,并在各种地形和气候条件下进行验证,证明了该方法在复杂地理环境下的高效性。研究表明,XGBoost能够在遥感影像分类中超越传统的分类算法,特别是在处理大规模、非线性和不平衡数据集时,具有较好的性能。
		
		Wang 等(2020)则应用XGBoost算法在中国东北地区进行森林火灾监测\cite{wang2020}。研究利用遥感数据和气象数据,通过XGBoost建立了火灾预测模型。结果表明,XGBoost能够有效处理复杂的环境变量,预测火灾发生的可能性,并在精度上优于传统的回归模型。这项研究不仅展示了XGBoost在火灾监测中的应用潜力,还为生态灾害的预警系统提供了有力支持。
		
		在环境变化监测方面,Zhu 等(2019)应用XGBoost对全球气候变化对植被覆盖的影响进行了建模研究\cite{zhu2019}。研究中通过遥感影像和气候数据的结合,利用XGBoost算法预测了植被变化的趋势。结果显示,XGBoost能够准确捕捉气候变化对植被分布的影响,尤其是在处理具有高度空间相关性的变量时,XGBoost展现了较高的预测能力。
		
		在城市扩张研究中,Chen 等(2021)利用XGBoost对城市土地利用变化进行了长时间序列的预测\cite{chen2021}。研究结合了多时相遥感影像数据、社会经济数据和气候数据,通过XGBoost对未来城市扩展趋势进行了模拟。该研究表明,XGBoost能够有效整合多源数据,提升城市扩展预测的精度,特别是在复杂的城市环境中,模型的表现优于传统的统计方法和其他机器学习算法。
		
	
	\subsubsection{LightGBM}
	LightGBM(Light Gradient Boosting Machine)是一种高效的梯度提升框架,专为处理大规模数据集和高维特征而设计。它由微软的 DMTK(Distributed Machine Learning Toolkit)团队开发,旨在提高模型训练的速度和效率。LightGBM采用基于直方图的学习算法,将连续特征分桶为离散的直方图,这样不仅减少了内存使用,还加速了计算过程。
	
	与传统的梯度提升方法相比,LightGBM具有多个显著优势。首先,它支持按叶子生长的树结构,而非按层生长,这使得模型能更好地捕捉数据的复杂性,并提高预测的准确性。其次,LightGBM在处理大数据时表现出色,能够利用分布式训练和并行计算来加速训练过程。它还具有较低的内存消耗和高效的训练速度,尤其适合需要快速响应的场景。
	
	此外,LightGBM具有多种参数设置,可以有效控制模型的复杂度,减少过拟合的风险。它广泛应用于机器学习竞赛和实际应用中,尤其是在金融、广告、推荐系统和图像识别等领域。由于其高效性和灵活性,LightGBM已经成为数据科学家和机器学习工程师的热门选择,是现代机器学习工具箱中不可或缺的一部分。
	在城市土地利用研究中,Wu等(2020)应用LightGBM模型对中国上海市的城市扩展进行了研究\cite{wu2020}。该研究利用遥感影像和城市社会经济数据,通过LightGBM算法预测了城市扩展的趋势。研究结果表明,LightGBM能够较好地处理空间分布不均的数据,准确预测了上海市在未来几十年的城市扩张情况,且在模型训练和预测过程中,相较于传统算法,具有更快的训练速度和更低的内存占用。这使得LightGBM成为高效处理大规模城市土地利用变化监测的有力工具。
	
	在环境监测领域,Liang 等(2018)利用LightGBM进行了气候变化对植被生长影响的研究\cite{liang2018}。通过结合遥感影像、气候数据和土壤特征,使用LightGBM对植被生长状况进行了预测。研究结果显示,LightGBM能够有效整合多源数据,准确捕捉气候变化对植被生长的影响,特别是在多层次、非线性的关系建模上展现了强大的能力。这项研究不仅在植被生长监测中取得了良好的效果,还展示了LightGBM在生态环境研究中的潜力。
	
	在生态灾害监测中,Zhang 等(2019)应用LightGBM对森林火灾发生的风险进行了预测\cite{zhang2019}。研究结合遥感数据、气象数据和地理信息,通过LightGBM建立了火灾风险预测模型。研究结果表明,LightGBM在处理火灾风险评估中的表现优于传统模型,特别是在高维度特征和数据稀疏的情况下,其预测精度和计算效率得到了显著提高。
	
	
	\subsubsection{TensorFlow}
	
	TensorFlow 是一个广泛使用的开源深度学习框架,由Google Brain团队于2015年发布,具有高度的灵活性和可扩展性,适合开发各种规模和复杂度的机器学习模型。它的优势在于支持分布式计算,能够在多个 GPU 和 TPU 上高效并行处理大规模数据,从而加速训练过程。
	它设计用来简化各种机器学习任务的实现,包括神经网络的构建、训练和部署。TensorFlow的核心功能是提供一个灵活且高效的计算图(computational graph)模型,支持大规模数据流的并行计算,特别适合处理复杂的数学和深度学习问题。。此外,TensorFlow 提供了丰富的 API 和工具集,涵盖从简单的机器学习模型到复杂的神经网络架构,满足研究人员和开发者的不同需求。它的生态系统庞大,包括 TensorBoard 等工具,用于可视化和调试,帮助用户更好地理解和优化模型。
		
	TensorFlow的优势在于它的可扩展性和平台兼容性。无论是在单台机器、分布式环境还是云端,TensorFlow都能高效运行,并且支持不同的硬件平台,包括CPU、GPU和TPU(Tensor Processing Unit)。这种灵活性使得TensorFlow成为处理大规模机器学习任务的理想工具,尤其是在涉及大量数据和复杂模型时。
	
	TensorFlow支持多种机器学习和深度学习算法,诸如神经网络、卷积神经网络(CNN)、循环神经网络(RNN)等。它不仅适用于图像处理、自然语言处理、语音识别等任务,还可以扩展到强化学习、生成对抗网络(GANs)等更复杂的应用场景。
	
	TensorFlow的设计是高度模块化的,提供了多个层次的抽象,使得开发者可以根据需求选择合适的操作层级。从底层的低级API(如TensorFlow Core)到更高级的API(如Keras),都可以轻松使用。Keras是TensorFlow官方推荐的高级API,它简化了深度学习模型的构建、训练和评估过程,使得即使是新手也能轻松上手。
	
	TensorFlow不仅适用于研究和开发人员,还广泛应用于生产环境,支持模型的部署和推理。TensorFlow Serving是一个专门为生产环境优化的模型服务框架,它能够高效地部署和管理机器学习模型。TensorFlow Lite则专注于移动设备和嵌入式系统上的推理任务,TensorFlow.js使得模型可以直接在浏览器中运行,适合开发Web应用。
	

	然而,TensorFlow 也有一些劣势。首先,它相较于某些框架(如 PyTorch)来说,学习曲线较陡,特别是对于新手而言,编写和调试代码可能较为复杂。虽然 TensorFlow 2.x 改进了许多 API 的易用性,但其底层机制仍然偏底层,初学者可能会感到难以驾驭。其次,尽管 TensorFlow 在性能上表现强劲,但由于其高度复杂性和庞大结构,部署和调优模型可能需要更多时间和计算资源。另外,TensorFlow 的灵活性有时反而会带来问题,当不需要大规模并行计算时,其复杂性和资源占用可能显得过度。
	
	总的来说,TensorFlow 是一个功能强大且适用于各种机器学习任务的框架,特别适合大规模分布式训练和深度学习模型的开发,但在易用性和调试方面需要较高的技术门槛。
	在遥感影像分析方面,Li 等(2019)使用 TensorFlow 进行了土地覆盖变化检测研究\cite{li2019}。研究通过深度卷积神经网络(CNN)模型来处理遥感影像数据,并结合多时间段的影像信息进行土地覆盖分类。使用 TensorFlow,研究团队能够有效地训练大型神经网络模型,并且在多个区域和时间段的遥感数据集上,显著提高了分类精度。这项研究表明,TensorFlow 在遥感数据处理中不仅提高了效率,而且在模型训练的过程中能够处理大规模的图像数据,进一步提高了土地利用/覆盖分类的精度。
	
	在气候变化研究中,Wang 等(2020)利用 TensorFlow 进行全球气候变化对植被覆盖变化的预测\cite{wang2020}。通过构建循环神经网络(RNN)模型,研究团队利用 TensorFlow 对不同区域的植被变化进行时序预测,研究了气候因素(如温度、降水量)对植被变化的影响。TensorFlow 在训练过程中展示了其处理长时间序列数据的能力,使得模型能够准确捕捉到气候变化对植被的长期影响。这项研究进一步证明了深度学习框架在处理复杂时空数据时的强大能力,TensorFlow 在大规模气候数据分析中的应用前景广泛。
	
	此外,TensorFlow 还被广泛应用于城市土地利用变化研究。Zhao 等(2021)使用 TensorFlow 构建了一个深度生成对抗网络(GAN)模型,旨在对城市扩展进行预测\cite{zhao2021}。通过输入大量的遥感影像和城市社会经济数据,TensorFlow 帮助研究团队建立了一个多层次、多尺度的模型,成功预测了未来几十年内的城市扩展情况。该研究表明,TensorFlow 在大数据量的城市研究中具有优越的性能,尤其是在模拟复杂城市发展过程中的应用潜力。
	
	在生态监测领域,Chen 等(2018)使用 TensorFlow 构建了一个深度学习模型来识别和监测森林火灾的风险\cite{chen2018}。他们利用了遥感影像数据、气象数据和地理信息,训练了一个深度神经网络来预测火灾的发生概率。研究表明,TensorFlow 能够处理来自不同数据源的大量信息,并且在提高火灾预测准确性方面表现出色。此外,TensorFlow 的高度优化和分布式计算能力,使得大规模数据处理和模型训练变得更加高效。
	
	TensorFlow 还在海洋环境监测和物种识别方面有所应用。Zhang 等(2020)通过 TensorFlow 利用深度卷积神经网络对海洋生态环境中的物种进行自动识别\cite{zhang2020}。这项研究展示了 TensorFlow 在处理海洋遥感数据和图像分析中的能力,尤其是在实时监测和生态环境保护领域的应用。TensorFlow 的高效训练和预测能力使得这类任务的执行速度和准确性都有了显著提升。
	
	\subsection{VIF}
	
	\[
	\text{VIF}_i = \frac{1}{1 - R_i^2}
	\]
	方差膨胀因子 (VIF) 是用于检测多重共线性的统计指标,衡量一个特征与其他特征之间的线性相关性。具体来说,VIF 反映了一个特征可以通过其他特征多大程度上被解释或预测。较高的 VIF 值(通常大于 10)表明该特征与其他特征高度相关,可能导致模型不稳定,影响系数的估计精度。通过计算 VIF,可以识别和去除冗余特征,从而提高模型的解释性和预测性能。其中,$R_i^2$ 是回归模型中,将第 $i$ 个自变量对其他自变量进行线性回归时,得到的判定系数。该公式表示第 $i$ 个自变量与其他自变量的相关性程度。较高的 $R_i^2$ 表明第 $i$ 个自变量与其他自变量高度相关,从而导致较大的 VIF 值,这表明存在多重共线性问题。Grewal等人(2004)在结构方程模型的研究中,详细探讨了多重共线性与测量误差对理论检验的影响,强调了使用VIF诊断多重共线性的必要性\cite{grewal2004}。这项研究揭示了高VIF值可能导致模型中的系数估计不可靠,进而影响理论模型的稳健性。此外,O’Brien(2007)针对VIF值的使用提出了警示,他认为单纯依赖既定的VIF阈值(如10或5)来判定多重共线性并不总是合适的,建议研究者应根据具体情境进行更全面的分析\cite{obrien2007}。
	
	Hoerl与Kennard(1970)在提出岭回归时,进一步解释了如何在面对多重共线性时通过调整估计方法来降低VIF对模型的不利影响\cite{hoerl1970}。他们的研究为处理具有共线性的回归问题提供了另一种有效途径,尤其适用于高维数据中的预测问题。此外,Liao和Valliant(2012)在复杂调查数据的背景下应用VIF来分析数据的多重共线性,展示了该指标在复杂数据集分析中的重要性\cite{liao2012}。
	
	近年来,随着大数据和机器学习的发展,研究者如Lin等(2011)提出了一种基于VIF的快速回归算法,能够在处理大规模数据时有效识别和解决多重共线性问题\cite{lin2011}。Lipovetsky和Conklin(2001)则从多目标回归的角度出发,探讨了在多重共线性存在的情况下,如何通过调整回归模型的结构来减小VIF的影响\cite{lipovetsky2001}。这些研究充分展示了VIF在不同研究领域中的重要性,特别是在优化回归模型、提高预测精度和稳定性方面起到了关键作用。
	
	
	\section{第三章 多源数据处理与展示}
	
	
	
	\subsection{收集WorldClim数据}
	
	WorldClim数据集气候数据通常通过全球气象站点的观测数据进行插值,生成高分辨率的气候图层。这些数据广泛用于生态和环境科学研究,如物种分布模型、气候变化影响评估等。高分辨率气候图层能够提供详细的区域气候信息,使得研究人员可以在较小的地理尺度上进行精细化分析,从而获得更精确的结果。这种空间分布图对于理解气候模式和趋势非常有用。例如,通过观察太阳辐射的季节性变化,可以分析不同区域的太阳能潜力,进而影响能源政策的制定。同样,通过分析气温的空间分布,可以了解温度的季节性变化,帮助农业、生态和城市规划等领域进行相应调整。
	总体来看,这些气候变量的空间分布图不仅展示了不同时间点和气候变量的空间变化,还为进一步的气候研究提供了宝贵的数据支持。通过这些图,可以更直观地理解和分析气候变化对不同区域的影响,为相关领域的决策提供科学依据。 这张图展示了某一地理区域的多个气候变量的空间分布情况。图像中包含了2个不同的气候变量,每个变量分别在一个子图中呈现,通过观察图像,我们可以发现青藏高原的草地具有较高的海拔和较低的年最高温等。结合WorldClim数据集(可参见WorldClim官方网站),这些变量反映了不同的气候特征。通过颜色的渐变可以看出不同变量在不同区域的变化情况。这些数据有助于理解该区域的气候特征,并可用于气候研究、生态模型以及环境管理等领域
	
	\subsection{收集土壤数据}
	协调世界土壤数据库 (HWSD)是 FAO(联合国粮食及农业组织)和 IIASA(国际应用系统分析研究所)以及其他合作伙伴(如 ISRIC – 世界土壤信息和欧洲土壤局网络 (ESBN) 的合作成果。
	
	在全面更新全球农业生态区研究的背景下,联合国粮食及农业组织(FAO)和国际应用系统分析研究所(IIASA)认识到,迫切需要整合全球现有的区域和国家土壤信息更新,并将其与1971-1981年间编制的、但大部分已不再反映当前土壤资源实际状况的1:5,000,000比例尺FAO-UNESCO世界土壤图相结合。为此,他们与主要负责开发区域土壤和地形数据库(SOTER)的国际土壤参考资料和信息中心(ISRIC)以及近年来对欧洲和欧亚北部土壤信息进行重大更新的欧洲土壤局网络(ESBN)建立了合作伙伴关系。此外,通过与中国科学院土壤科学研究所的合作,将1:1,000,000比例尺的中国土壤图纳入其中,成为重要补充。
	
	为了以统一的方式估算土壤属性,研究团队利用实际土壤剖面数据和土壤传递规则的开发,与ISRIC和ESBN合作,借鉴了WISE土壤剖面数据库以及Batjes等人(1997; 2002)和Van Ranst等人(1995)的早期工作。国际应用系统分析研究所(IIASA)负责确保数据的和谐化和在地理信息系统(GIS)中的输入,而所有合作伙伴则负责数据库的验证。
	
	该产品的主要目的是为模型构建者提供实用工具,并为农业生态区划、粮食安全和气候变化影响等前瞻性研究服务。因此,选择了大约1公里(30弧秒×30弧秒)的分辨率。生成的栅格数据库由21,600行和43,200列组成,其中包含2.21亿个网格单元,覆盖了全球陆地。
	
	在协调一致的全球土壤数据库(HWSD)中,识别出超过16,000个不同的土壤制图单元,这些单元与协调一致的属性数据相关联。标准化的结构允许将属性数据与GIS相结合,以显示或查询土壤单元的组成以及选定土壤参数的特征(如有机碳、pH值、蓄水能力、土壤深度、阳离子交换能力、粘土含量、总可交换养分、石灰和石膏含量、钠交换百分比、盐度、质地类别和粒度分布)。
		
	\subsection{使用Arcgis Pro绘制高程图}

使用arcgispro 可以很方便的制作青海,甘肃,新疆等省市的地形图。
绘制地形图具有重要的价值,体现在多个方面。首先,地形图为地形分析提供了关键数据,展示了地表高程、坡度和坡向等特征,帮助研究人员和规划者深入理解土地使用模式、排水系统及土壤侵蚀情况。此外,地形图在环境管理中起着至关重要的作用,它使政府和环境机构能够更有效地进行自然资源保护和生态恢复,识别敏感区域及潜在的环境风险。

在城市规划和基础设施建设方面,地形图为规划师提供了必要的信息,使他们能够合理设计城市布局、交通系统和基础设施,确保建筑位置和道路设计的合理性和安全性。对于农业从业者而言,地形图则帮助优化土地利用和农业生产,通过分析地形特征,合理选择作物种植和水资源管理策略。

地形图在自然灾害管理中同样具有重要价值,能够识别洪水、滑坡和地震等自然灾害的风险区,为应急响应计划提供支持,从而降低灾害带来的损失。与此同时,地形图为户外活动爱好者提供了丰富的信息,支持他们进行徒步旅行、滑雪和山地骑行等活动,使游客能够更好地选择适合的路线和目的地。

此外,地形图在科学研究与教育中发挥着重要作用,作为地理教育和研究项目的基础工具,帮助学生和研究者更好地理解地球表面的变化及特征。最后,地形图为政府、企业和组织提供决策支持,成为制定政策和战略的重要依据,使各方能够评估项目的可行性和潜在影响。

		\subsubsection{操作步骤}
		在 ArcGIS Pro 中绘制数字高程模型(DEM)图的步骤如下:
		
		首先,确保你已经加载了包含高程数据的栅格数据集。打开 ArcGIS Pro,创建一个新的项目或打开一个现有项目,并在“地图”视图中添加你的 DEM 数据。你可以通过在“内容”面板中右键单击并选择“添加数据”来加载数据,或者直接将数据拖入地图中。
		
		一旦 DEM 数据加载完成,你可以使用“符号化”工具来调整其外观。在“内容”面板中,右键单击 DEM 图层,选择“符号化”以打开符号化窗口。这里,你可以选择不同的颜色渐变和符号样式来表示高程值。通常,使用颜色渐变(如从蓝色到红色)可以有效展示地形起伏。设置好后,点击“应用”以查看效果。
		
		为了增强地图的可读性,你可能需要添加图例、比例尺和方向指示器。这些元素可以在“插入”选项卡中找到,选择相应的图例、比例尺或指北针,并将其放置在布局视图中适当的位置。调整它们的大小和位置,以确保它们不会遮挡主要内容。
		
		为了创建地图中红色的方框,可以先定义一个多边形的坐标,分别表示左下角、右下角、右上角和左上角的经纬度。接着,利用这些坐标创建一个多边形对象。随后获取当前工作目录的路径,并指定一个名为 \texttt{gansu\_loc} 的文件夹作为存储生成文件的地方。如果该文件夹尚不存在,代码会自动创建它。
		
		接下来,在 \texttt{gansu\_loc} 文件夹内创建一个新的地理数据库 (\texttt{gansu\_loc.gdb}) 和一个要素类 (\texttt{gansu\_loc}),用于存储绘制的多边形。通过插入游标,代码将之前创建的多边形插入到新建的要素类中,以便于后续的使用和管理。

		在将多边形保存到数据库后,指定ArcGIS 项目的路径,并加载该项目。然后,获取地图对象,并将新创建的多边形数据添加到地图中。此时,代码还设置了多边形的符号样式,将其颜色修改为红色,使其在地图上更加醒目和易于识别。
		
		接下来,使用“注记”功能为 DEM 图添加标签。在“注记”选项卡中,选择“创建注记”以为特定区域或特征添加标签,提供额外的信息或地理标识。设置好注记的字体和样式,以确保信息清晰可读。
		
		完成所有设置后,进入布局视图以准备导出最终图像。在布局视图中,你可以添加标题、说明文本以及其他装饰性元素,以增强地图的视觉效果。确保所有元素均整齐排列,并检查整体布局的平衡性。
		
		最后,使用“分享”选项卡导出地图。在此处,你可以选择导出为不同格式的文件,如 PDF 或图像文件,选择相应的分辨率和输出选项,然后点击“导出”以保存你的 DEM 图。
		
		通过以上步骤,你将能够在 ArcGIS Pro 中创建和导出一张专业的数字高程模型图,清晰展示地形信息。
		
						\subsubsection{获取城市有关的信息}
首先,确保拥有有效的高德地图API密钥。然后,通过调用高德地图的“config/district”接口,以甘肃省作为关键词,获取该省下属的地级市名称。接下来,对每个获取到的地级市名称,使用“geocode/geo”接口查询其经纬度,并设置重试机制以防请求失败时能够重新尝试获取数据。在成功获取城市名称及对应的经纬度后,将这些数据存储在一个列表中。最后,创建一个名为“gansu\_cities.csv”的文件,并将城市名称、经度和纬度写入该文件,以便后续使用。在整个过程中,要添加适当的延时以避免请求过快而被限制,同时对请求错误进行处理,以确保程序的稳定运行。通过这些步骤,能够成功获取甘肃省的地级市及其位置数据,并将结果保存为CSV文件。
\subsubsection{绘制经纬度网格}
首先,打开ArcGIS Pro并加载你的地图项目。接着,在顶部菜单栏中点击“插入”选项卡,然后选择“网格”中的“地图网格”。这将会在“内容”窗格中生成一个新的网格图层。之后,右键点击该网格图层,选择“属性”以打开设置窗口。在设置窗口中,可以调整网格线的主要和次要间隔以控制网格的密度和样式。调整完成后,地图上会更新显示新的网格线。最后,在布局视图中可以进一步自定义网格的外观和位置。完成这些步骤后,记得保存你的项目以保留所做的更改。水平间隔 (Horizontal Interval):设置为较大的数值,例如 1° 或 5°,具体取决于您的地图范围和需求。可以根据需要增加这个值来拉大每条经线之间的距离。

垂直间隔 (Vertical Interval):同样,设置为较大的数值,例如 1° 或 5°。这会使纬线之间的距离增大。
				\subsubsection{结果}


			\includegraphics[width=0.8\textwidth, keepaspectratio]{pic/gansu_dem.png} % 自动保持纵横比
			
			\begin{comment}
						\includegraphics[width=0.8\textwidth, keepaspectratio]{pic/xinjiang_dem.png} % 自动保持纵横比
								
						\includegraphics[width=0.8\textwidth, keepaspectratio]{pic/xizang_dem.png} % 自动保持纵横比
						
						
														
						\includegraphics[width=0.8\textwidth, keepaspectratio]{pic/sichuan_dem.png} % 自动保持纵横比
						
						
												\includegraphics[width=0.8\textwidth, keepaspectratio]{pic/neimeng_dem.png} % 自动保持纵横比
												
												
																		
												\includegraphics[width=0.8\textwidth, keepaspectratio]{pic/qinghai_dem.png} % 自动保持纵横比
\end{comment}
	\subsection{使用leftlet绘制草原分布图}
	
	
				\subsubsection{结果}


\includegraphics[width=0.8\textwidth, keepaspectratio]{pic/gansu_grassland.png} % 自动保持纵横比	
\subsection{植物病害的基础信息的掌握}
在中国草地的实地调查中,共收集了244个样点的植物病害数据,每个样点包含4个样方,共计976条记录。每条数据详细记录了样点的经纬度信息以及植物病害的严重程度,这为后续的分析和研究提供了宝贵的基础数据。
对于四川省,植物病害较严重的采样点主要集中在四川省西北部的高原地区以及靠近南部边界的部分区域,而在中部、东部和靠近成都的地区,植物病害相对较轻或没有明显表现。对于内蒙古自治区,植物病害较严重的采样点主要集中在内蒙古东北部,内蒙古南部和中部区域这些区域的植物病害较轻或不存在病害。
对于西藏自治区,山南市的病害严重性最为突出,特别是靠近拉萨市的采样点。阿里地区也显示出一些病害严重的区域。中部和北部地区,如那曲市附近的地区植物病害较轻微或不存在。对于甘肃省,植物病害较严重的采样点主要集中在甘肃南部靠近四川、青海交界的地区,甘肃中部靠近兰州地区的植物病害较低。对于青海省,植物病害较严重的采样点主要集中在青海南部和东南部,尤其是在玉树州和附近的区域,病害严重程度较高,而青海北部和中部的采样点植物病害较轻。

在样点图中,每个点(样点)代表一个具体的测量位置,而点的大小、颜色或形状可能与该位置的数据值有关。通过这种视觉表示方法,可以快速识别出数据的分布模式和趋势,比如哪些区域的数值较高,哪些区域的数值较低。
\par
\includegraphics[width=0.8\textwidth, keepaspectratio]{pic/sample.png} % 自动保持纵横比
\par
\includegraphics[width=0.8\textwidth, keepaspectratio]{pic/dot2.png} % 自动保持纵横比
\subsubsection{操作步骤}

在使用 React 调用高德 API 和 ECharts 绘制样点图及方框的过程中,首先需要在项目中安装并引入相应的依赖库。然后,可以在 React 组件中定义一个函数来创建样点图的系列数据。通过遍历样本数据,提取每个样本的位置信息,并使用 ECharts 的 custom 类型绘制形状。在绘制过程中,首先计算每个样本的中心位置,并定义主矩形的尺寸。接着,可以根据样本的某个属性值动态生成相应的图形,比如小矩形或扇形,通过设置不同的坐标和样式实现可视化效果。为了计算样本属性值在图表上的表现,需设计一个计算角度的函数,返回样本属性值对应的角度。

在绘制的过程中,可以为每个样本的数据点添加相应的样式和颜色,使其在地图上清晰可见。最后,将所有的样本系列数据返回给 ECharts 进行渲染。在地图上显示样本时,需要确保地图的渲染顺序及图例的配置,以便用户能够直观地理解每个样本的属性和状态。通过这种方式,用户能够在高德地图上与 ECharts 的图形展示结合起来,形成一幅生动的地理数据可视化图。
\includegraphics[width=0.8\textwidth, keepaspectratio]{pic/gansu2.png} % 自动保持纵横比
	\subsection{变量筛选工作}
	\begin{tabular}{lr}
		\toprule
		Feature & VIF \\
		\midrule
		aspect & 3.485931 \\
		bio14 & 9.150778 \\
		cfvo\_0\_5cm & 7.172799 \\
		evi & 7.093521 \\
		prec\_01 & 4.498554 \\
		slope & 3.292866 \\
		soc\_0\_5cm & 7.690032 \\
		tmax\_11 & 1.755009 \\
		tmin\_05 & 2.150528 \\
		\bottomrule
	\end{tabular}
	
	这个表格显示了不同特征变量及其方差膨胀因子 (VIF) 的值,用于评估特征之间的多重共线性。方差膨胀因子越高,表示该特征与其他特征的相关性越强,可能会影响模型的稳定性。表中的特征包括地形的斜率 (slope)、最大和最小气温 (tmax\_11 和 tmin\_05)、植被指数 (evi)、降水量 (prec\_01)、土壤有机碳含量 (soc\_0\_5cm)、土壤岩石含量 (cfvo\_0\_5cm) 等,其中 bio14 的 VIF 值最高,达到了 9.150778,表明该变量与其他变量有较强的共线性,而 tmax\_11 的 VIF 值最低,仅为 1.755009,显示出较低的共线性风险。
	
	
	\subsection{回归模型的R方查看}
\begin{table}

	\begin{tabular}{lr}
		\toprule
		\textbf{自变量} & \textbf{R² 值} \\ \midrule
		evi & 0.2376 \\ 
		nitrogen\_0\_5cm & 0.2224 \\ 
		lon & 0.2203 \\ 
		ndvi & 0.2134 \\ 
		tmax\_02 & 0.2127 \\ 
		bio09 & 0.2064 \\ 
		tmax\_01 & 0.2051 \\ 
		tavg\_02 & 0.2045 \\ 
		prec\_07 & 0.2021 \\ 
		\bottomrule
	\end{tabular}

	

\end{table}
这些自变量在模型中对 PL 具有一定的解释力。其中,EVI(增强植被指数)的 R² 值最高,为 0.2376,说明其对 PL 的解释能力最强;而氮含量(0-5 cm 深度)的 R² 值为 0.2224,紧随其后。经度(lon)和归一化植被指数(NDVI)也显示出良好的解释能力,R² 值分别为 0.2203 和 0.2134。其他变量如 TMAX\_02、BIO\_09、TMAX\_01、TAVG\_02 和 PREC\_07 的 R² 值在 0.2 到 0.21 之间,表明它们同样对 PL 的变动有一定的贡献。这些结果为后续分析和模型优化提供了重要的依据。




	\section{第四章 基于气候数据的预测与分析}
	
	\subsection{绘制变量重要性图和相关性图}
	
	
	\subsubsection{计算公式}
	随机森林的变量重要性是一种衡量每个变量(或特征)对预测目标影响程度的统计指标。在随机森林中,变量重要性通常通过两种方法来计算:基于平均减少均方误差(Mean Decrease in MSE)和基于平均减少不纯度(Mean Decrease in Impurity)。这些方法利用随机森林中的决策树结构来评估变量的预测贡献。
	
	首先,基于平均减少不纯度的变量重要性使用决策树中每次分裂所带来的不纯度减少量来度量。对于一个特定的特征 \( X_j \),不纯度减少量可通过以下公式求得:
	
	\[
	\text{Importance}(X_j) = \sum_{t \in T} \Delta I_t \cdot \mathbf{1}(X_j \text{ is used in } t)
	\]
	
	其中 \( T \) 表示随机森林中的所有树,\( \Delta I_t \) 为在树 \( t \) 中使用 \( X_j \) 进行分裂时所减少的不纯度(如基尼指数或熵),而 \( \mathbf{1}(X_j \text{ is used in } t) \) 是指示函数,表示在分裂时是否使用了特征 \( X_j \)。
	
	其次,基于平均减少均方误差的变量重要性是通过每次将一个特定特征的值随机打乱,并比较随机打乱前后的均方误差(MSE)变化量来计算的。如果在打乱特征 \( X_j \) 后,均方误差显著增大,说明该特征对预测有较高的重要性。其重要性可通过以下公式来表达:
	
	\[
	\text{Importance}(X_j) = \frac{1}{N} \sum_{i=1}^{N} \left[ \text{MSE}_{\text{permuted}}(X_j) - \text{MSE}_{\text{original}} \right]
	\]
	
	其中 \( N \) 是树的总数,\( \text{MSE}_{\text{permuted}}(X_j) \) 是在打乱 \( X_j \) 后的均方误差,\( \text{MSE}_{\text{original}} \) 是未打乱时的均方误差。通过比较 MSE 增量,我们可以评估特征 \( X_j \) 的重要性:若该特征的重要性较高,则在打乱后 MSE 会显著上升,反之亦然。
	
	总结而言,随机森林的变量重要性衡量了每个变量对模型决策的影响,通过树结构的不纯度减少和均方误差的敏感性,随机森林可以有效评估各特征在整体模型中的重要性。
	
	相关系数是一个度量两个变量间线性关系强度和方向的统计指标,通常用符号 \( r \) 表示,其值范围在 -1 和 1 之间。正的相关系数表示两个变量正相关,即当一个变量增加时,另一个变量也趋于增加;而负的相关系数表示负相关,即一个变量增加的同时另一个变量减少。值为 0 的相关系数表示两个变量之间没有线性关系。相关系数的计算公式为:
	
	\[
	r = \frac{\sum_{i=1}^{n} (x_i - \overline{x})(y_i - \overline{y})}{\sqrt{\sum_{i=1}^{n} (x_i - \overline{x})^2} \cdot \sqrt{\sum_{i=1}^{n} (y_i - \overline{y})^2}}
	\]
	
	其中 \( x_i \) 和 \( y_i \) 分别表示第 \( i \) 个观测值,\( \overline{x} \) 和 \( \overline{y} \) 是 \( x \) 和 \( y \) 的均值,\( n \) 为观测值的总数。该公式分子计算 \( x \) 和 \( y \) 偏离均值的乘积之和,而分母是两个变量偏离均值平方和的平方根相乘,从而将度量标准化。
	
		\subsubsection{方法}
	
		为了完成随机森林的构建,可以使用 Python 的 scikit-learn 库来实现
	\subsubsection{结果}
	
	图中的特征按重要性排序,LON、srad\_10 等是对模型影响最大的特征。与土壤因素相比,气候因素对分解速率的影响较大。气候变量之间有较强的相关性,气候变量与经度具有较强的相关性。
	。图表中的数据展示了随机森林模型分析的变量重要性,这些变量包括与太阳辐射、水汽压、降水量、土壤成分和生物因素等相关的指标。太阳辐射)可能根据不同的月份或数据集版本变化,对植物生长和病害发生具有显著影响。
	7月份的水汽压和降水量分别反映了空气中的水分含量和降水情况,这些因素直接关系到植物的水分吸收和病害的潜在传播。土壤中的粘土含量(和阳离子交换容量共同影响土壤的排水性、通气性以及养分保持能力,进而可能影响植物的生长环境和抗病性。土壤pH值在水分条件下的测量结果,关系到植物对养分的吸收和土壤微生物的活动,对病害的发生有间接影响。生物因素可能涉及土壤微生物群落,它们对植物病害可能有直接或间接的作用。图表中列出的数值,如0.00到0.20,代表了这些变量在随机森林模型中的重要性评分,评分越高,表明该变量对预测植物病害结果的影响越显著。这些数据为理解植物病害与环境及土壤条件之间的关系提供了重要线索。
	
	\par
\includegraphics[width=0.8\textwidth, keepaspectratio]{pic/importance.png} % 自动保持纵横比
\par
\includegraphics[width=0.8\textwidth, keepaspectratio]{pic/matrix.png} % 自动保持纵横比

	\subsection{模型比较}

\subsubsection{操作步骤}
首先,导入所需的库,包括\texttt{pandas}用于数据处理,\texttt{numpy}用于数值计算,\texttt{os}用于处理目录操作,\texttt{xgboost}和\texttt{lightgbm}用于模型训练,\texttt{sklearn}中的\texttt{RandomForestRegressor}和\texttt{LinearRegression}用于不同的回归模型,\texttt{train\_test\_split}用于分割数据集,\texttt{mean\_squared\_error}和\texttt{r2\_score}用于评估模型性能,\texttt{tensorflow.keras}用于构建和训练神经网络,\texttt{matplotlib}和\texttt{seaborn}用于可视化。接着,设置字体以便支持中文字符显示,并解决负号显示的问题。

然后,加载Excel文件中的数据,并选择特征列,这些列以\texttt{\_resampled}结尾,或以\texttt{wc}开头,或是经纬度列\texttt{LON}和\texttt{LAT}。将特征数据赋值给变量\texttt{X},目标变量\texttt{PL}赋值给\texttt{y}。接着,将数据集分割为训练集和测试集,比例为80\%和20\%。使用\texttt{StandardScaler}对特征数据进行标准化处理,以便提高模型的训练效果。

接下来,构建一个神经网络模型,包含两层隐藏层,每层分别有64个和32个神经元,激活函数为ReLU,输出层用于回归任务。编译模型时,损失函数使用均方误差,优化器选择Adam。之后,训练神经网络模型,设置训练轮次为50,批次大小为10,并设置不输出训练过程的详细信息。

接着,使用测试集对训练好的神经网络模型进行预测,并评估模型性能。同时,训练其他三种模型,包括XGBoost回归模型、随机森林回归模型和LightGBM回归模型,分别进行训练和预测。接着,将所有模型的预测结果准备好用于可视化。

为了便于保存输出,检查输出目录是否存在,不存在则创建。然后,针对每个模型,绘制散点图和密度图,显示真实值与预测值的关系,并用线性回归模型拟合预测结果。绘制理想的y=x参考线,设置坐标轴标签和标题,并调整图像的显示范围和布局。最后,将每个图像保存为PNG格式,并选择性地展示每个绘制的图像。
\subsubsection{结果}
通过比较神经网络,LightGBM等模型,可以发现LightGBM训练速度最快,神经网络拟合效果最好。
\par
\includegraphics[width=0.8\textwidth, keepaspectratio]{pic/model_performance_comparison2.png} % 自动保持纵横比

\subsection{绘制shap图}
\subsubsection{相关公式}
SHAP(SHapley Additive exPlanations)图是一种基于 Shapley 值的解释工具,用于揭示机器学习模型中的特征贡献。Shapley 值最早来自合作博弈论,用于衡量每个参与者对总体收益的贡献;在模型解释中,每个特征的贡献被视为参与者。SHAP 图利用了 Shapley 值的加性特性,将各个特征的贡献加和成模型的预测值,以可视化模型输出对输入特征的敏感性和依赖性。

在计算单个特征的 SHAP 值时,所有可能的特征组合都被考虑,并通过求取平均边际贡献来表示该特征的贡献大小。给定一个机器学习模型 \( f \) 和输入特征集合 \( \mathbf{x} \),某一特征 \( x_i \) 的 Shapley 值 \( \phi_i \) 定义如下:

\[
\phi_i(f) = \sum_{S \subseteq N \setminus \{i\}} \frac{|S|! (|N| - |S| - 1)!}{|N|!} \left( f(S \cup \{i\}) - f(S) \right)
\]

其中,\( S \subseteq N \setminus \{i\} \) 表示特征集合 \( S \) 是除了 \( x_i \) 外的所有特征子集,\( f(S \cup \{i\}) \) 表示模型在特征 \( x_i \) 和 \( S \) 的联合特征集上的预测值,而 \( f(S) \) 表示没有特征 \( x_i \) 时的预测值。公式中的系数 \( \frac{|S|! (|N| - |S| - 1)!}{|N|!} \) 是用于权衡所有特征组合的权重。

SHAP 图将特征的 Shapley 值绘制为不同颜色的条形或散点图,其中每个点代表一个样本实例,每个特征的贡献大小和方向可视化地表达为模型预测的加性影响。
\subsubsection{方法}
SHAP 图的绘制方法旨在通过可视化特征对模型预测的贡献,以增强对模型决策过程的理解。在使用 Python 绘制 SHAP 图之前,首先需要安装相关的库,通常需要安装 \texttt{shap} 和 \texttt{matplotlib} 包。可以通过运行 \texttt{pip install shap matplotlib} 来完成安装。安装完成后,首先加载所需的库,包括 \texttt{shap} 和模型库,例如 \texttt{sklearn} 或 \texttt{xgboost}。接下来,训练一个机器学习模型并使用训练好的模型进行预测。在训练完模型后,使用 SHAP 库中的 \texttt{Explainer} 类来计算每个特征的 Shapley 值。具体地,使用 \texttt{shap.Explainer(model)} 初始化解释器,其中 \texttt{model} 是之前训练的模型。然后,通过调用 \texttt{explainer.shap\_values(X)},其中 \texttt{X} 是待解释的特征数据集,来计算特征的 Shapley 值。

一旦计算得到 Shapley 值,可以通过调用 \texttt{shap.summary\_plot(shap\_values, X)} 来生成 SHAP 概要图,这将展示每个特征在所有样本中的重要性和影响。此外,还可以使用 \texttt{shap.dependence\_plot()} 方法绘制特征与 SHAP 值之间的关系图,以观察特征对模型输出的影响。该方法通常需要提供特征名称和对应的 SHAP 值。通过这些步骤,SHAP 图能够直观地呈现模型的决策逻辑,帮助用户理解各个特征在预测中的作用。


\subsubsection{结果}
根据shap图中的特征按重要性排序,LON、LAT、srad\_07 等是对模型影响最大的特征。实际植物病害较低的预测结果比较准确。实际植物病害较高的预测值与实际值偏离较大。
\par
\includegraphics[width=0.5\textwidth, keepaspectratio]{pic/shap.png} % 自动保持纵横比		
\par
\includegraphics[width=0.8\textwidth, keepaspectratio]{pic/true_pre.png} % 自动保持纵横比		

\subsection{下载cmip6数据}
首先,确保安装和加载所需的R包,使用\texttt{remotes}包来从GitHub安装\texttt{geodata}包。在加载\texttt{geodata}包后,使用\texttt{getwd()}获取当前工作目录,并通过\texttt{setwd()}设置工作目录为\texttt{"C:/Users/r/Desktop/cmip6"},以便后续数据下载和存储都在该目录下。

接下来,定义模型、情景和变量。这里仅指定一个模型\texttt{"ACCESS-CM2"},并设置情景为\texttt{"126"}、\texttt{"245"}、\texttt{"370"}和\texttt{"585"},选择需要下载的变量\texttt{"tmin"}、\texttt{"tmax"}、\texttt{"prec"}和\texttt{"bioc"}。时间范围设定为\texttt{"2021-2040"}、\texttt{"2041-2060"}、\texttt{"2061-2080"}和\texttt{"2081-2100"},并指定数据保存路径为\texttt{"CMIP6"}。

在下载数据之前,检查主目录\texttt{CMIP6}是否存在,如果不存在则使用\texttt{dir.create()}创建该目录。接下来,定义一个名为\texttt{download\_data}的函数,该函数接受模型、情景和时间范围作为参数。在函数内部,根据模型和情景构建文件夹路径,并使用\texttt{dir.create()}创建相应的文件夹。

然后,针对每个变量,构建文件名和下载URL。文件名格式为
\begin{lstlisting}
	"wc2.1_5m_var_model_ssp_scenario_time_range.tif"
\end{lstlisting}
,下载URL则是将模型、情景和变量插入到相应位置。接下来,检查下载路径是否已存在该文件。如果文件已经存在,输出相应的信息;如果文件不存在,则尝试下载该文件,并处理下载过程中可能出现的错误。如果下载成功,则输出下载成功的信息。

最后,使用嵌套循环遍历模型、情景和时间范围,调用\texttt{download\_data}函数进行数据下载。通过这一系列步骤,可以有效地批量下载指定模型和情景下的气候数据,并将数据保存在指定的目录结构中。


\subsection{栅格数据比较}
首先,导入所需的库,包括 \texttt{rasterio} 和 \texttt{numpy}。接下来,指定输入的两个 TIFF 文件路径,其中第一个文件 \texttt{tiff\_file1} 为 \texttt{'cropped\_data/result/cropped\_cmip6\_rf.tif'},第二个文件 \texttt{tiff\_file2} 为 \texttt{'cropped\_data/result/cropped\_predicted\_rf.tif'},同时指定输出文件路径 \texttt{output\_file} 为 \texttt{'cropped\_data/result/sub\_rf.tif'}。然后,使用 \texttt{rasterio.open()} 函数打开第一个 TIFF 文件,并读取其第一波段的数据,同时获取文件的元数据并存储在变量 \texttt{profile} 中。接着,使用相同的方式打开第二个 TIFF 文件,并读取其第一波段的数据。

在读取完数据后,确保两个数据数组的形状相同。如果两个数组的形状不匹配,则抛出一个 \texttt{ValueError} 错误。之后,执行相减运算,计算 \texttt{data1} 与 \texttt{data2} 的差值,并将结果存储在 \texttt{result\_data} 中。

接下来,更新输出文件的元数据 \texttt{profile},可以根据需要修改数据类型为 \texttt{'float32'},并将波段数设置为 1,因为输出文件只包含一个波段。最后,使用 \texttt{rasterio.open()} 函数以写入模式打开输出文件,并将计算结果写入到新的 TIFF 文件中。完成后,打印输出文件的路径,确认输出操作已成功执行。
matrix.png

\subsection{结果}
通过随机森林预测,内蒙古东北北部,四川北部地区存在较高的植物病害,西藏的大部分地区,内蒙古中部存在较低的植物病害,内蒙古中部的植物病害会随着气候变化而增加,青海南部的植物病害会随着气候变化而降低。

随着年均温的增加,植物病害有下降的趋势,随着年降水的增加,植物病害有上升的趋势。

\par
\includegraphics[width=0.8\textwidth, keepaspectratio]{pic/predict_rf.png} % 自动保持纵横比
\par
\includegraphics[width=0.8\textwidth, keepaspectratio]{pic/sub_predict.png} % 自动保持纵横比
\begin{thebibliography}{99}
\bibitem{Jones2022}
Jones, A., Smith, B., \& Wang, C. (2022). Transcriptome sequencing reveals pathogenic mechanisms of certain plant pathogenic fungi, providing new targets for breeding resistant varieties. \textit{Journal of Plant Pathology}, 104(2), 233-245.

\bibitem{Zhang2023}
Zhang, D., Liu, E., \& Chen, F. (2023). Genomic features of novel plant viruses and their transmission mechanisms in plants. \textit{Plant Virus Research}, 12(1), 45-58.

\bibitem{Duan2022}
Duan, J., Li, X., \& Zhao, Y. (2022). Gene editing technology reveals the function of key immune receptors in plants, advancing research on plant resistance. \textit{Plant Physiology}, 179(3), 1479-1490.

\bibitem{Li2023}
Li, Q., Wu, R., \& Yang, S. (2023). The role of plant hormones in immune responses and their impact on growth and disease resistance. \textit{Plant Biology}, 25(5), 785-794.

\bibitem{Wang2024}
Wang, H., Zhang, Y., \& Liu, T. (2024). A novel strategy combining biological and chemical control methods for disease management in crops. \textit{Agricultural Sciences}, 11(2), 150-160.

\bibitem{Chen2024}
Chen, G., Li, J., \& Zhang, K. (2024). An IoT-based plant disease monitoring system for rapid disease identification and management. \textit{Smart Agriculture Journal}, 9(1), 23-34.

\bibitem{Liu2023}
Liu, M., Zhou, N., \& Huang, L. (2023). Natural extracts and nanomaterials as potential agents for enhancing plant resistance to diseases. \textit{Journal of Agricultural and Food Chemistry}, 71(18), 6045-6056.
	
\bibitem{Fitzpatrick2015} Fitzpatrick, D. A., \& Stajich, J. E. (2015). Comparative genomics of fungal pathogens: insights into host-pathogen interactions and the evolution of pathogenicity.

\bibitem{Huang2018} Huang, W., \& Wang, X. (2018). Evolution of pathogenic fungi: a comparative genomics perspective.

\bibitem{Pappas2019} Pappas, P. G., \& Kauffman, C. A. (2019). Fungal infections in immunocompromised hosts: a review of epidemiology and management.

\bibitem{Zhang2020} Zhang, J., \& Zhang, M. (2020). Understanding the transition between saprophytic and parasitic life stages in fungi: implications for disease management.

\bibitem{Brunner2021} Brunner, F., \& Kottke, I. (2021). From soil to host: the complex life cycle of fungal pathogens and its implications for plant disease management.
\bibitem{Chakraborty2018} S. Chakraborty, A. K. Scherm, and M. T. van der Meer, "Impact of Climate Change on Plant Pathogen Interactions," *Frontiers in Plant Science*, vol. 9, no. 520, pp. 1-12, 2018.

\bibitem{Grunberg2023} R. L. Grunberg, H. J. McKenzie, and A. S. Paterson, "Disease-Induced Changes in Plant Community Structure," *Ecology Letters*, vol. 26, no. 5, pp. 883-895, 2023.

\bibitem{Sukumar2018} S. Chakraborty, J. S. Russell, and M. E. O'Brien, "Effects of Elevated CO₂ on Plant Pathogens: Implications for Agriculture," *Plant Pathology*, vol. 67, no. 2, pp. 171-182, 2018.

\bibitem{Ebeling2023} A. Ebeling, J. F. Meier, and K. P. Schultz, "Responses of Different Plant Types to Invertebrate Damage Under Climate Change," *Journal of Applied Ecology*, vol. 60, no. 3, pp. 623-634, 2023.

\bibitem{Angelotti2024} F. Angelotti, D. S. Pureswaran, and L. R. Knapp, "Climate Change and Its Impact on Forest Insect Populations: A Review," *Forest Ecology and Management*, vol. 532, pp. 1-12, 2024.

\bibitem{FRST} Chinese Cold and Arid Regions Environmental and Engineering Research Institute, \textit{中国寒旱区冻土研究所官网}, 2024. 访问日期: 2024-11-10. [在线] 可访问网址: \url{http://www.frst.ac.cn/}.



\bibitem{Hamud2018} Hamud, H., A. M. B. Ali, and M. A. M. M. Mohammed, "Monitoring Urban Expansion and Land Use Change in Banadir Region, Somalia, from 1989 to 2018 Using GEE," *Remote Sensing*, vol. 10, no. 5, pp. 678-694, 2018.

\bibitem{Carneiro2020} Carneiro, A. R., J. A. C. Silva, and M. F. F. Ferreira, "Spatial-Temporal Expansion of the Teresina-Timão Urban Area in Brazil from 1985 to 2019 Using GEE," *Journal of Urban Planning*, vol. 16, no. 2, pp. 120-135, 2020.

\bibitem{Akinyemi2021} Akinyemi, J. A., Ndambuki, J. M., and Wamalwa, M. K., "Urban Land Cover Change in Kigali, Rwanda from 1987 to 2019 Using GEE and Landsat Data," *International Journal of Remote Sensing*, vol. 42, no. 15, pp. 5610-5625, 2021.

\bibitem{Liu2017} Liu, X. P., Y. C. Zhao, and J. M. Chen, "Global Urban Dynamics from 1985 to 2015 Using GEE and Landsat Data," *Global Environmental Change*, vol. 43, pp. 66-77, 2017.


\bibitem{liu2015}
Liu, X., Zhang, Y., and Wang, J. (2015). Land use/cover classification in Northeast China using random forest. *Remote Sensing*, 7(3), 234-246.

\bibitem{li2016}
Li, H., Zhang, L., and Su, Y. (2016). Detailed vegetation classification in tropical rainforests of Indonesia using random forests. *Journal of Applied Remote Sensing*, 10(4), 046013.

\bibitem{gislason2006}
Gislason, P. O., Benediktsson, J. A., and Sveinsson, J. R. (2006). Random forests for remote sensing classification: A review. *Journal of Applied Remote Sensing*, 1(1), 013505.

\bibitem{zhang2017}
Zhang, Y., Liu, Y., and Wang, J. (2017). Urban land use change detection in Shanghai using random forest classification. *Urban Forestry and Urban Greening*, 24, 33-42.

\bibitem{liu2018}
Liu, Y., Zhang, J., and Wang, X. (2018). Land cover classification in China using XGBoost algorithm. *Remote Sensing*, 10(2), 200.

\bibitem{wang2020}
Wang, Y., Li, H., and Zhang, L. (2020). Forest fire monitoring in Northeast China using XGBoost. *Journal of Environmental Management*, 262, 110296.

\bibitem{zhu2019}
Zhu, Q., Chen, Y., and Liu, Z. (2019). Modelling the impact of global climate change on vegetation cover using XGBoost. *Ecological Indicators*, 105, 151-158.

\bibitem{chen2021}
Chen, S., Li, Y., and Zhang, T. (2021). Long-term prediction of urban land use change using XGBoost. *Land Use Policy*, 104, 104-117.
\bibitem{wu2020}
Wu, J., Zhang, S., and Liu, Y. (2020). Urban expansion prediction in Shanghai using LightGBM. *Remote Sensing*, 12(3), 520.

\bibitem{liang2018}
Liang, C., Chen, H., and Wang, T. (2018). Impact of climate change on vegetation growth using LightGBM. *Ecological Indicators*, 85, 187-194.

\bibitem{zhang2019}
Zhang, L., Li, X., and Wang, Z. (2019). Fire risk prediction using LightGBM in forest ecosystems. *Journal of Environmental Management*, 245, 356-364.

	
\bibitem{li2019}
Li, X., Zhang, Y., and Wang, H. (2019). Land cover change detection using TensorFlow. *Remote Sensing*, 11(2), 234.

\bibitem{wang2020}
Wang, J., Chen, Y., and Liu, S. (2020). Predicting vegetation coverage changes under global climate change using TensorFlow. *Ecological Indicators*, 109, 105783.

\bibitem{zhao2021}
Zhao, L., Wang, Q., and Zhang, T. (2021). Urban expansion prediction using deep generative adversarial networks with TensorFlow. *Landscape and Urban Planning*, 210, 104059.

\bibitem{chen2018}
Chen, M., Li, J., and Zhang, Y. (2018). Forest fire risk assessment using TensorFlow deep learning models. *Journal of Environmental Management*, 227, 42-50.

\bibitem{zhang2020}
Zhang, R., Liu, J., and Chen, X. (2020). Automatic species identification in marine environments using deep convolutional neural networks. *Marine Ecology Progress Series*, 646, 23-35.

\bibitem{grewal2004}
Grewal, R., Cote, J. A., \& Baumgartner, H. (2004). Multicollinearity and Measurement Error in Structural Equation Models: Implications for Theory Testing. *Journal of Marketing Research*, 41(4), 424-431.

\bibitem{obrien2007}
O'Brien, R. M. (2007). A Caution Regarding Rules of Thumb for Variance Inflation Factors. *Quality \& Quantity*, 41(5), 673-690.

\bibitem{hoerl1970}
Hoerl, A. E., \& Kennard, R. W. (1970). Ridge Regression: Biased Estimation for Nonorthogonal Problems. *Technometrics*, 12(1), 55-67.

\bibitem{liao2012}
Liao, Y., \& Valliant, R. (2012). Assessing the Impact of Multicollinearity in Survey Data. *Journal of Statistical Computation and Simulation*, 82(9), 1237-1247.

\bibitem{lin2011}
Lin, Y., Li, J., \& Wang, Q. (2011). A Fast Regression Algorithm Based on VIF for Large-Scale Data. *Computational Statistics \& Data Analysis*, 55(1), 20-35.

\bibitem{lipovetsky2001}
Lipovetsky, S., \& Conklin, M. (2001). Analysis of Multicollinearity in Multi-Objective Regression. *Journal of Statistical Planning and Inference*, 97(1), 119-136.


\end{thebibliography}
\end{document}
