\documentclass{article}
\usepackage{hyperref}
\usepackage{graphicx}
\usepackage{amsmath}
\usepackage{geometry}
\geometry{a4paper, margin=1in}
\usepackage{ctex}  % 处理中文支持
\usepackage{booktabs}

\title{基于机器学习的植物病害分析与预测模型构建}
\author{}
\date{}

\begin{document}
	
	\maketitle
	
	\tableofcontents
	
	\section{第一章 前言}
	\subsection{研究背景}
	\subsubsection{植物病害概述}
	
	植物在生长过程中容易受到包括病原真菌和卵菌在内的多种病原体的侵害,进而引起形态异常、功能受损和生理受限,发生植物病害。
	病原真菌如锈菌、白粉菌以及卵菌中的疫霉菌和霜霉菌是导致植物病害的主要原因之一。锈病、白粉病和叶斑病等是植物病害的主要类型。
	锈病通常表现为叶片和茎秆上的小斑点,严重时会导致叶片脱落和植株枯死。
	白粉病则主要表现为植物表面覆盖一层白色粉状物,严重影响光合作用。
	叶斑病导致叶片上出现各种颜色的斑点,逐渐扩散并导致叶片枯萎。
	
	\subsubsection{ 死体病原菌与活体病原菌}
	病原物大体分为两类:一类病原物杀死寄主,然后从上面获得营养物质,即所谓的死体营养寄生物; 另一类是需要获得寄主以完成它们的生活史,即活体营养寄生物。 活体病原菌的一个短暂阶段代表了半活体营养病原菌。这类真菌在开始转向杀死寄主之前具有一个活体营养生长阶段。
	
	\subsubsection{气候变化与植物病害的关系}
	
	最新研究表明,气候变化和全球变暖导致温度的升高和部分地区降水格局的改变,正在加剧这些病害的发生和传播。温暖潮湿的环境有利于病原体的繁殖和扩散,导致病害在更大范围内更频繁地发生。例如,科学家发现全球变暖导致的温度升高和降水模式的改变,正促使一些病原真菌和卵菌向新的地理区域扩展,这些区域以前并不适合它们的生存和繁殖。气候变化还影响了植物的生理状态,使其更易受到病害侵染。实际上 , 温度和降水是影响叶片真菌病害的主要环境因子。 叶片真菌病害往往在高温、高湿的环境下较为严重。根据样点,使用机器学习方法预测全国病害有助于更好地认识到中国范围内病害的空间格局。
	
	植物病害对全球农业生产力和粮食安全构成重大挑战。及时准确地预测这些病害对于有效的病害管理和减轻策略至关重要。近年来,数据收集技术的进步促使了多样化数据集的获取,涵盖了气象条件、土壤特性、植物物种信息以及植物病害严重程度。
	
	\subsection{国内外研究现状}
		草地对动物产业、土壤保护和生物多样性至关重要,但植物病害会降低产量和营养价值(S. Chakraborty et al. 2018)。病害选择性地影响了某些物种,从而减少了群落内的物种多样性和丰富度。(Rita L. Grunberg et al. 2023)
	植物病理学家Sarah J. Gurr等人使用广义线性模型,研究发现真菌和昆虫每年向两极迁移7公里。相反,蠕虫(线虫)显示出向低纬度地区移动的趋势。对于其他分类群,如螨虫、细菌、双翅目、半翅目、膜翅目、等翅目、卵菌、原生动物、缨翅目和病毒,没有观察到显著的纬度变化趋势。气候变化可能对不同害虫分类群的地理分布产生了影响,其中一些群体正逐渐向两极迁移以适应新的环境条件。CO₂浓度升高,植物病原体感染能力增强(Sukumar Chakraborty et al. 2018)。
	
	Anne Ebeling等的研究通过对不同植物类型在不同年均温度和年均降水条件下受病害和无脊椎动物损害的分析,揭示了它们对环境变化的不同响应。研究发现,杂草在年均降水增加和年均温度升高的条件下,表现出显著的病害和无脊椎动物损害增加的趋势,尤其在高温高湿的环境中更为明显。相反,草类和豆科植物对这些因素的响应相对稳定,没有显示出明显的损害程度增加趋势。
	
	Deepa S. Pureswaran等人探讨了气候变化对森林害虫的影响。他们综合了2013-2017年间的最新文献,包括之前的相关综述,深入讨论了气候变化如何影响昆虫的分布范围、数量、森林生态系统以及昆虫群落的影响。研究发现,气候变化可以促进害虫爆发或破坏食物链,从而减少害虫爆发的严重程度。通过广义线性模型和大尺度空间分析,研究揭示了气候变化对不同昆虫类群的地理分布和生态影响。气候变化导致英国地区部分地区极端天气增多(Francislene Angelotti et al. 2024)。

	\subsection{研究内容和研究区概况}
	
	\subsubsection{研究内容}
	
	根据研究目标,本文需要进行数据采集和数据整合。数据整合,将数据自不同来源的数据整合到一起。数据整合的过程包括数据表的合并、连接或关联,以创建一个包含完整信息的数据集。最后进行预测分析并展现分析后的结果。
	
	\subsubsection{中国草地概述}
	草地在生态系统中扮演着多种角色。首先,它们是重要的碳汇,能够吸收大气中的二氧化碳并将其储存在植被和土壤中。其次,草地是许多野生动植物的栖息地,为它们提供了食物和庇护所。此外,草地还有助于防止土壤侵蚀,通过根系固定土壤,减少水土流失。在农业方面,草地是畜牧业的基础,为牲畜提供食物来源。通常,草地在全球的分布受到气候、地形和人类活动等多种因素的影响。例如,在非洲的萨瓦纳地区、北美的大草原以及南美的潘帕斯草原都是草地生态系统的典型代表。
	
	至于中国,它拥有世界上最大的草地面积之一,约占全球草地面积的百分之10左右。
	
	中国的草地主要分布在西部和北部地区,中国的草地资源非常丰富,草地类型多样,这些地区包括主要分布在内蒙古草原、青藏高原草地、新疆草地、东北草地和黄土高原草地等区域。这些草地生态系统在地理上呈现出明显的纬度和海拔梯度。
	
	内蒙古高原是中国最大的草地区域之一,这里的草地以温带草原为主,是重要的畜牧业基地。内蒙古的草地覆盖了广阔的平原和低山丘陵地带,为众多的牲畜提供了丰富的食物资源。内蒙古自治区是中国最大的草原分布区,涵盖了呼伦贝尔草原、锡林郭勒草原和阿拉善草原。呼伦贝尔草原以其平坦广袤的草地和优质的牧草著称,是优良的天然牧场。锡林郭勒草原以丰富的生物多样性和独特的自然景观闻名,而阿拉善草原则以其干旱和半干旱的生态环境为特色。
	
	青藏高原则以其高海拔草地著称,这里的草地属于高山草甸类型,由于海拔高,气候寒冷,草地生长的植物种类相对较少,但它们对于维持高原生态系统的稳定和生物多样性具有重要作用。青藏高原位于中国西南部,草地主要集中在青海省、西藏自治区和四川省的部分地区。这里的草地包括高寒草甸和高寒草原,以高寒冷湿的气候和复杂的地形为特征,是许多珍稀野生动物的栖息地,如藏羚羊和野牦牛。
	
	新疆地区则有干旱和半干旱的草地,这里的草地生态系统适应了干旱的环境条件,多为耐旱和耐盐碱的植物种类。新疆的草地在支持当地畜牧业和保护生态平衡方面发挥着关键作用。新疆维吾尔自治区的草地主要分布在天山山脉和阿尔泰山脉地区,包括天山草甸草原和阿尔泰山草原。这里的草地气候干旱,植被稀疏,但却是重要的畜牧业基地。
	
	东北地区的草地主要集中在吉林省和黑龙江省的部分地区,如松嫩平原和三江平原,受季风气候影响,夏季湿润,适宜草原植物生长。
	
	黄土高原位于中国西北部,草地主要分布在陕西省和甘肃省的部分地区,由于气候干旱,多为干旱草原和荒漠草原,植被稀疏,土壤贫瘠。
	
	然而,草地生态环境也面临着过度放牧、气候变化和土地荒漠化等挑战,需要加强保护和可持续管理。 整体来看,中国的草地生态系统在地理分布上呈现出多样化的特点,从温带草原到高山草甸,再到干旱草原,它们不仅为畜牧业提供了基础,对水土保持、防风固沙和维护生物多样性具有重要作用,对于维持区域乃至全球的生态平衡具有不可替代的作用。同时,这些草地也是中国重要的自然景观和生态旅游资源,对于促进地方经济发展和生态旅游具有重要意义。
	
	
		\subsubsection{中国草地分类}
		
		中国草地可以分为草原,草原和草丛。
	对于新疆省,草甸草原主要分布在新疆的北部和西部地区,尤其是在靠近阿尔泰山、天山等高海拔地区。这些区域因为海拔较高、降水较多,适合草甸类植被的生长。草原主要出现在新疆西北部和南部的部分山地区域,例如巴音郭楞蒙古自治州周边地区。这些草原植被适应了较为干旱的气候,覆盖范围较大。
		\par 对于青海省,在青海省的东部高山地区,如黄南、海南藏族自治州、果洛藏族自治州等地,这些区域因为海拔较高,气候湿润,适合草甸类植被生长。草原主要分布在青海省的西部和北部区域,例如海西蒙古族藏族自治州和青海湖附近。这些地区气候干燥,降水较少,植被类型多为耐旱的草原。
		\par 对于甘肃省,草甸植被主要分布在甘肃省的东南部和中部的高海拔地区,尤其是临夏回族自治州、甘南藏族自治州等山区,这些地区气候相对湿润,适合草甸植被生长。草原植被广泛分布在甘肃省的西部和北部地区,如酒泉市、张掖市、嘉峪关市等,这些区域气候干燥,降水较少,因此草原植被更为常见。
	
	
	\par 对于四川省,草甸植被主要分布在西部的高原和山区,特别是甘孜藏族自治州、阿坝藏族羌族自治州等地。这些区域海拔较高,气候较为湿润,适合草甸植被的生长。这里的草甸植被通常与丰富的降水和冷凉的气候条件相关。在四川省,草丛分布较为稀疏,集中在部分东南部地区,如攀枝花市及周边地区,这些区域的气候较为温暖,适合草丛植被的生长。
	
		\par 西藏自治区的草地主要以草甸为主,广泛分布在西藏的中部和北部地区,包括那曲市、阿里地区以及部分拉萨市周边的高原区域。这些地区地势较高,气候寒冷湿润,适合草甸的生长,草甸在西藏的高原地带显得尤为广泛。草原主要分布在西藏自治区的西部和南部边缘,如日喀则市、山南市等区域。这些地区的气候相对干燥,草原植被适应了这种相对恶劣的环境条件,因此占据了这些干旱半干旱地区。
		
		\par 对于内蒙古,草原在自治区内占据了广大的面积,是草地的主要类型。草原分布较为广泛,尤其集中在东部的呼伦贝尔、锡林郭勒以及赤峰等地区,这些地区地势平缓,气候适宜,是典型的草原牧区,拥有丰富的草原资源。草甸也出现在内蒙古的一些地方,主要分布在北部和中部地区的局部,如兴安盟和部分呼和浩特市的周边,受当地水源和湿润气候的影响,草甸类型的植被较为丰富。
		
		
		
		\subsubsection{植被指数计算}
		
				
		在本例中,使用了美国地质调查局(USGS)的SRTM 30米分辨率数字高程模型(DEM)数据集。
		在遥感影像处理中,归一化植被指数(NDVI)是通过计算红光(B4波段)和近红外(B5波段)的比值来估计植被覆盖度的常用指数。NDVI的计算公式如下:
		\[
		NDVI = \frac{NIR - Red}{NIR + Red}
		\]
		
		其中:
		- \( NIR \) 表示近红外波段的像素值(Landsat 8的B5波段),
		- \( Red \) 表示红光波段的像素值(Landsat 8的B4波段)。
		
		此公式通过对红光和近红外波段的反射率差异进行归一化,得到了一个介于 -1 到 1 之间的数值,通常用于评估植被覆盖度:
		- \( NDVI > 0 \) 表示有绿色植被,值越高植被越茂盛,
		- \( NDVI < 0 \) 通常表示无植被的区域(如沙漠或水体),
		- \( NDVI = 0 \) 表示土壤或其他不含绿植的区域。
		
		在遥感影像处理中,增强植被指数(EVI)是一种常用的植被指数,能够有效抑制大气干扰并提高对植被变化的敏感性。EVI的计算公式如下:
		
		\[
		EVI = 2.5 \times \frac{NIR - Red}{NIR + 2.4 \times Red + 1}
		\]
		
		其中:
		- \( NIR \) 表示近红外波段的像素值(Landsat 8的B5波段),
		- \( Red \) 表示红光波段的像素值(Landsat 8的B4波段)。
		
		此公式通过对红光和近红外波段的反射率进行加权运算,计算出一个数值,表示植被的生长情况,EVI的值通常大于0,值越大表示植被越茂盛。
		\par
		地上生物量(AGB,Above Ground Biomass)是指植物地面部分的生物质总量,通常包括树木、草本植物、灌木等的茎、叶和根部分的生物量。它是衡量植物生长和生态系统健康的重要指标之一,广泛用于生态学、气候变化研究以及碳循环评估。AGB 的大小与植物的生长速度、土壤肥力、气候条件等多种因素密切相关。在森林生态系统中,AGB 是碳储存的重要组成部分,对全球气候变化的缓解具有重要影响。植物通过光合作用吸收二氧化碳并将其转化为有机物质,从而在地上生物量的形式中储存碳。因此,研究和监测 AGB 对于了解全球碳循环、评估气候变化的影响以及制定相应的环境保护措施都至关重要。随着遥感技术的发展,全球范围内的地上生物量数据可以通过卫星观测获得,从而为科学家提供了大规模、长期的监测手段。这些数据有助于在全球尺度上理解和量化森林和其他植被类型的碳存储能力,对应对气候变化具有重要的理论和实践意义。
		\subsubsection{地理特征计算}
		
		在地理信息系统中,坡度(Slope)是指地表某一点的陡峭程度。坡度可以通过数字高程模型(DEM)计算得到,公式如下:
		
		\[
		S = \arctan \left( \sqrt{ \left( \frac{\partial z}{\partial x} \right)^2 + \left( \frac{\partial z}{\partial y} \right)^2 } \right)
		\]
		
		其中:
		- \( S \) 代表坡度,
		- \( z \) 是高程值,
		- \( \frac{\partial z}{\partial x} \) 和 \( \frac{\partial z}{\partial y} \) 分别表示高程在水平 \( x \) 和垂直 \( y \) 方向上的梯度。

		
		坡向(Aspect)是地形的一项重要特征,表示地面某一点的朝向,即地表最陡坡的方向。坡向的计算通常基于数字高程模型(DEM)数据。设定坡向公式如下:
		
		\[
		A = \arctan \left( \frac{ \frac{\partial z}{\partial x} }{ \frac{\partial z}{\partial y} } \right)
		\]
		
		其中:
		- \( A \) 是坡向,单位通常为角度(以度表示)。
		- \( z \) 是高程值。
		- \( \frac{\partial z}{\partial x} \) 和 \( \frac{\partial z}{\partial y} \) 分别表示高程在水平方向(x方向)和垂直方向(y方向)上的梯度。
		
		
	\section{第二章 技术路线}
	
	
	\subsection{地理信息处理方式}

	本文的地理信息处理方式依赖于GDAL库。
	GDAL(Geospatial Data Abstraction Library)是一个开源的地理空间数据处理库,广泛应用于地理信息系统(GIS)领域,具有强大的数据读写和处理功能。GDAL支持多种栅格和矢量数据格式,如GeoTIFF、Shapefile、KML、GeoJSON等,使其能够在不同的数据格式之间进行转换和处理,这为数据的集成和分析提供了极大的便利。此外,GDAL的跨平台特性使得它能够在Windows、Linux、macOS等多种操作系统上运行,确保了它的广泛适用性。GDAL还提供了高效的性能,特别是在处理大规模数据时,它能快速读取、写入以及进行各种数据处理操作,如投影转换、栅格运算和矢量数据的操作等。
	
	在C语言中调用GDAL时,首先需要安装并包含GDAL的头文件。通过使用GDALAllRegister()函数注册数据驱动后,可以使用GDALOpen()函数打开文件,并通过GDALGetRasterBand()等API访问数据。处理完数据后,需要使用GDALClose()释放资源。C语言的GDAL接口功能全面,适合性能要求高、底层控制需求强的应用。
	
	在Python中调用GDAL相对简单,首先需要通过pip安装GDAL库,然后通过from osgeo import gdal导入相关模块。使用Python的gdal.Open()函数打开文件后,可以通过GetRasterBand()访问栅格数据,并使用ReadAsArray()等函数获取像素值。Python接口具有更为友好的语法和开发效率,适合快速开发和原型设计,但依然保留了GDAL的强大功能。无论是C语言还是Python,GDAL都能提供强大的地理空间数据处理能力,满足不同开发需求。


	\subsection{前端和后端}
	在前后端开发和集成学习的整合中,可以实现高效的数据处理、模型训练和预测结果展示。
	
	在数据预处理阶段,后端服务器会对用户上传的数据进行清洗、格式化和特征提取。这些预处理步骤对于保证模型预测的准确性至关重要。完成预处理后,数据被传递给集成学习模型进行预测。集成学习模型可以由多种机器学习算法组成,如随机森林、XGBoost和神经网络等。
	
	集成学习的核心在于结合多个弱学习器的预测结果以提高整体模型的性能。通常的方法包括Bagging、Boosting和Stacking。在Bagging方法中,多个模型并行训练,最终预测结果通过平均或投票的方式决定。Boosting则是通过逐步调整模型权重,关注前一阶段预测错误的数据,提高整体模型的准确性。Stacking是一种更为复杂的方法,通过训练一个元模型来组合多个初级模型的输出。
	
	训练完成的模型可以保存到文件系统或数据库中,以便后续的快速加载和更新。每次用户发起预测请求时,后端服务器会加载最新的模型进行预测。预测结果经过处理后,通过API返回给前端,前端将结果以可视化的形式展示给用户。
	
	通过这种技术路线,前后端和集成学习的整合不仅提高了数据处理和模型预测的效率,还提升了用户体验。前端提供了直观的交互界面,后端确保了数据处理和模型训练的可靠性,集成学习则增强了模型的预测性能。这种整合方法在实际应用中具有广泛的潜力,特别是在需要高精度预测的场景下。
	前端方便了数据的展示和内容的拆分与维护。前端指的是浏览器的显示的内容,通过使用js技术可以在网页上做出许多美观实用的图片,也可以在前端完成用户的交互工作,比如说下载图片、下载图片等。本文拟采用react完成直观的交互,比如说数据的上传、下载,结果的展示和下载等。React 是一个用于构建用户界面的 JavaScript 库。它由 Facebook 开发并开源。
	它主要专注于构建单页面应用程序(SPA),通过组件化的方式提高了代码的可复用性和可维护性。 React 的核心思想是组件化开发,将用户界面拆分为独立的组件,每个组件负责管理自己的状态和渲染逻辑。 React 的另一个显著特点是虚拟 DOM(Virtual DOM)。它通过在内存中维护一个虚拟 DOM 树来实现高效的 DOM 更新,通过比较前后两次虚拟 DOM 的差异,最小化了实际 DOM 操作的次数,从而提升了性能。 React 不仅可以用于 Web 应用程序的开发,还可以用于移动应用程序开发以及静态网站的生成。由于其灵活性和高效性,React 在现代前端开发中得到了广泛应用,并成为了构建复杂用户界面的首选工具之一。
	后端指的是网页后台中配合前端完成数据处理,和保存到数据库的一系列内容,常用的后端有Java开发的spring 系列后端 ,js后端node.js,和python的fastapi。在本文中,后端开发使用Python的FastAPI框架来构建RESTful API服务。FastAPI 是一个现代、快速(高性能)、基于标准 Python 类型提示的 Web 框架,用于构建 APIs,采用了 Python 3.6+ 版本。fastapi具有与 Node.js 和 Go 相媲美的高性能,因为它基于 Starlette 和 Pydantic 这两个高性能工具。在实际应用中,FastAPI 用于处理前端发送的数据请求,进行数据预处理并调用集成学习模型进行预测。
	
		
	\subsection{GEE平台}
	Google Earth Engine (GEE) 是一个基于云计算的地理空间分析平台,最初由 Google 推出,旨在为全球范围内的遥感数据和地理信息提供高效、便捷的分析工具。自从 GEE 于 2011 年推出以来,得到了广泛应用,特别是在大规模地理空间数据处理和分析方面,具有显著的优势。GEE 的出现标志着地理空间数据分析进入了云计算时代,突破了传统计算能力的局限,使得全球范围内的遥感数据处理变得更为快速和高效。
	
	GEE 的核心优势在于其提供了海量的遥感数据资源,包括 Landsat、MODIS、Sentinel 等卫星影像,以及其他各类气候、环境、土地利用等相关数据。它能够对这些海量数据进行在线处理、分析和可视化,而无需依赖传统的本地计算机资源。这大大简化了数据处理流程,节省了计算和存储成本。
	
	使用 GEE 进行遥感数据分析的一个显著特点是其强大的云计算能力。研究人员可以利用 GEE 平台处理和分析全球尺度的遥感数据,进行土地覆盖变化、城市扩展、森林监测等广泛的应用研究。相较于传统的遥感数据分析方法,GEE 通过云端处理和分析,使得数据的获取、处理、分析和结果输出变得更加高效和便捷。
	
	在 GEE 平台上,用户不仅可以使用多种预先集成的遥感数据集,还可以编写自己的算法脚本进行定制化分析。这使得 GEE 平台在遥感监测、环境变化研究、气候变化评估等领域的应用极其广泛。例如,基于 GEE 的全球森林变化监测研究已成为当前遥感研究中的重要方向,通过结合多时相遥感影像,研究人员能够实时监测全球森林资源的变化情况。
	
	此外,GEE 在土地利用/土地覆盖变化监测方面具有重要应用。研究人员利用 GEE 可以基于中高分辨率的遥感影像,对全球或区域范围内的土地利用/覆盖进行精细化的变化检测与分类分析,帮助政府和科研机构制定更加合理的土地利用政策与环境保护措施。特别是 GEE 提供的长期时间序列数据,使得大规模的时空动态变化分析成为可能。
	
	总结来说,GEE 是一个功能强大的遥感云计算平台,凭借其丰富的遥感数据资源和强大的云计算能力,成为全球范围内地理空间数据分析、环境监测和土地管理等领域的重要工具。通过 GEE,研究人员能够更加高效地进行大范围、长时间序列的地理数据分析,为全球可持续发展目标的实现提供科学支持。
	例如,Hamud 等(2018)利用 GEE 结合支持向量机和随机森林分类器的监督分类方法,监测了索马里巴纳迪尔地区1989至2018年的城市扩张和土地利用变化。该研究展示了 GEE 平台在大尺度遥感影像分类和土地利用变化研究中的应用潜力,尤其是在时间跨度较长、数据量巨大的情况下,GEE 的高效处理能力显得尤为重要。
	
	Carneiro 等(2020)基于 GEE 平台开展了巴西特雷西纳—帝蒙城市群1985至2019年的时空扩张研究。研究利用 GEE 提供的 Landsat 数据和时间序列分析技术,对城市化进程进行动态监测,揭示了城市化对当地环境的影响及其变化趋势。该研究不仅突显了 GEE 在城市扩张监测中的优势,还展示了其在时空动态变化研究中的强大能力。
	
	此外,Anonymous(2021)基于 GEE 和 Landsat 数据,结合支持向量机分类器和光谱—时间分隔算法,开展了1987至2019年卢旺达基加利城市土地覆盖变化研究。该研究利用 GEE 对大规模遥感数据进行处理,探讨了城市化进程对生态环境的影响,特别是在城市扩张和土地覆盖变化的监测上,提供了重要的参考数据和分析结果。
	
	刘小平等(2017)通过 GEE 和 Landsat 数据结合城市用地综合指数(NUACI)方法,绘制了1985至2015年全球城市动态图。该研究利用 GEE 的全球尺度遥感数据和强大的云计算能力,分析了全球城市化的时空变化,并提出了适用于大区域城市用地监测的有效方法。
	\subsection{机器学习方法}
	
	\subsubsection{随机森林}
	随机森林(Random Forest)是一种集成学习方法,它通过构建多个决策树并结合其结果来进行分类或回归任务。该算法由Leo Breiman在2001年提出,旨在通过降低模型的方差来提高预测的准确性和鲁棒性。它通过构建多个决策树并将其结果进行结合,形成一个“森林”。每棵树是在随机抽样的训练数据上生成的,通常采用自助采样(Bootstrap Sampling)技术,即在训练集上随机抽取样本并放回,这样每棵树的训练数据略有不同,从而增强模型的多样性。此外,在每棵树的构建过程中,随机森林还会在每个节点上随机选择特征进行分裂,这进一步减少了树之间的相关性,有助于降低过拟合风险。该算法通过随机选取训练数据的子集和特征来生成每棵树,从而降低各棵树之间的相关性,提高模型的鲁棒性和准确性。随机森林具有良好的抗过拟合能力和较高的泛化性能,特别适用于处理高维数据和缺失值。它能够自动处理大规模数据集,并提供特征重要性评估,帮助理解和解释模型的决策过程。此外,随机森林易于并行化,能够有效利用现代计算资源。
	随机森林通过随机选取训练数据的子集和特征来生成每棵树,使得各棵树之间的相关性降低,从而提升整体模型的性能。每棵决策树独立生长,且不会进行修剪,最终通过多数表决或平均值来汇总各个树的预测结果。随机森林具有良好的抗过拟合能力和较高的泛化性能,尤其在处理高维数据和缺失值时表现优异。其主要优势在于能够自动处理大规模数据集,并提供特征重要性评估,帮助理解和解释模型的决策过程。

	
	随机森林易于并行化,能够有效利用现代计算资源,广泛应用于金融、医学、市场营销、图像识别等多个领域,能够处理各种类型的数据,包括数值型和分类型数据。在 Python 中,利用 scikit-learn 库可以方便地实现随机森林模型,用户只需指定树的数量和其他参数,即可训练和评估模型。总之,随机森林凭借其高效的性能和广泛的应用场景,成为了机器学习领域的重要工具之一。
	随机森林的优点在于其抗过拟合能力和高准确性。通过结合多个决策树的预测结果,随机森林通常能够提供更为稳定和准确的预测。同时,它还可以评估特征的重要性,使得特征选择过程更加直观。然而,随机森林也存在一些缺点,例如模型复杂性较高,训练和预测的时间成本相对较大,并且相比单棵决策树,其可解释性较差,难以直观理解模型的决策逻辑。
	例如,Liu 等(2015)利用随机森林算法对中国东北地区的土地利用/覆盖进行分类,并取得了较高的分类精度。该研究采用遥感影像数据,结合随机森林的强大特性,在复杂的地理环境下成功应用于大范围的土地覆盖分类,验证了随机森林在处理高维度特征数据时的优势。该研究进一步强调了随机森林在遥感数据处理中的可靠性,尤其在分类精度要求较高的场景下。
	
	另一个相关研究是Li 等(2016)在印度尼西亚的热带雨林地区进行的研究。该研究使用随机森林算法对植被类型进行了详细分类,结果表明,随机森林能够有效处理复杂的遥感影像数据并克服了传统分类方法中存在的过拟合问题。研究中还结合了地形数据、气候数据和遥感影像数据,以提高模型的分类准确性,展示了随机森林在生态系统监测和土地利用研究中的应用潜力。
	
	此外,Gislason 等(2006)应用随机森林算法进行冰岛土地覆盖分类,提出了随机森林在处理遥感数据中的优势,尤其是在大尺度遥感影像数据的处理上,能够有效减少噪声对分类结果的影响,并提高了分类模型的稳健性。该研究的成功应用进一步证明了随机森林算法在遥感影像分类中的强大能力,尤其是在地形复杂、数据量庞大的环境下。
	
	在城市扩展研究方面,Zhang 等(2017)利用随机森林对中国上海市的城市土地利用变化进行了研究,展示了该算法在高分辨率遥感影像分类中的表现。通过引入地理空间特征和时间序列数据,研究有效捕捉了城市化进程中的细微变化,强调了随机森林在时空变化监测中的广泛应用。
	
	
	\subsubsection{xgboost}
	XGBoost 是一种基于决策树的机器学习算法,因其速度和性能在处理大规模数据和复杂问题时非常受欢迎。它的优势在于强大的计算效率和高精度,这得益于其内置的并行计算和对硬件的优化。XGBoost 通过梯度提升技术逐步减少误差,能够很好地处理分类和回归任务。此外,它还支持特征的自动化选择和缺失值处理,增强了模型的鲁棒性。
	XGBoost通过优化损失函数和增加正则化项,减少过拟合风险,同时加速模型的训练过程。其关键特性之一是实现了并行计算,使得在处理大规模数据集时具有显著的速度优势。此外,XGBoost还提供了多种灵活的参数配置,使得用户能够针对具体问题进行调优。该模型不仅适用于分类和回归任务,还在许多机器学习竞赛中表现出色,成为数据科学家和机器学习工程师的热门选择。由于其强大的性能和灵活性,XGBoost广泛应用于金融、医疗、广告和推荐系统等多个领域,是现代机器学习中不可或缺的重要工具。
	然而,XGBoost 也有一些缺点。首先,它相较于其他简单的模型如线性回归,需要更多的时间和资源来训练,尤其是当数据量非常大时。其次,XGBoost 的超参数调优较为复杂,错误的设置可能导致模型表现不佳。此外,尽管 XGBoost 的强大性能在大多数情况下表现出色,但它的解释性较差,不如简单模型容易解读。对于某些问题,XGBoost 的复杂性可能带来过拟合风险,尤其是当训练数据的规模和质量不足时。
	
	
	总体来说,XGBoost 非常适合高维度数据集和需要高精度的应用场景,但在某些情况下可能需要平衡其复杂性与可解释性。
		例如,Liu 等(2018)在其研究中使用了XGBoost算法对中国的土地覆盖进行分类,并取得了显著的结果。研究通过结合遥感影像和多维特征数据,应用XGBoost算法进行土地利用变化监测,并在各种地形和气候条件下进行验证,证明了该方法在复杂地理环境下的高效性。研究表明,XGBoost能够在遥感影像分类中超越传统的分类算法,特别是在处理大规模、非线性和不平衡数据集时,具有较好的性能。
	
	Wang 等(2020)则应用XGBoost算法在中国东北地区进行森林火灾监测。研究利用遥感数据和气象数据,通过XGBoost建立了火灾预测模型。结果表明,XGBoost能够有效处理复杂的环境变量,预测火灾发生的可能性,并在精度上优于传统的回归模型。这项研究不仅展示了XGBoost在火灾监测中的应用潜力,还为生态灾害的预警系统提供了有力支持。
	
	在环境变化监测方面,Zhu 等(2019)应用XGBoost对全球气候变化对植被覆盖的影响进行了建模研究。研究中通过遥感影像和气候数据的结合,利用XGBoost算法预测了植被变化的趋势。结果显示,XGBoost能够准确捕捉气候变化对植被分布的影响,尤其是在处理具有高度空间相关性的变量时,XGBoost展现了较高的预测能力。
	
	在城市扩张研究中,Chen 等(2021)利用XGBoost对城市土地利用变化进行了长时间序列的预测。研究结合了多时相遥感影像数据、社会经济数据和气候数据,通过XGBoost对未来城市扩展趋势进行了模拟。该研究表明,XGBoost能够有效整合多源数据,提升城市扩展预测的精度,特别是在复杂的城市环境中,模型的表现优于传统的统计方法和其他机器学习算法。
	
	\subsubsection{LightGBM}
	LightGBM(Light Gradient Boosting Machine)是一种高效的梯度提升框架,专为处理大规模数据集和高维特征而设计。它由微软的 DMTK(Distributed Machine Learning Toolkit)团队开发,旨在提高模型训练的速度和效率。LightGBM采用基于直方图的学习算法,将连续特征分桶为离散的直方图,这样不仅减少了内存使用,还加速了计算过程。
	
	与传统的梯度提升方法相比,LightGBM具有多个显著优势。首先,它支持按叶子生长的树结构,而非按层生长,这使得模型能更好地捕捉数据的复杂性,并提高预测的准确性。其次,LightGBM在处理大数据时表现出色,能够利用分布式训练和并行计算来加速训练过程。它还具有较低的内存消耗和高效的训练速度,尤其适合需要快速响应的场景。
	
	此外,LightGBM具有多种参数设置,可以有效控制模型的复杂度,减少过拟合的风险。它广泛应用于机器学习竞赛和实际应用中,尤其是在金融、广告、推荐系统和图像识别等领域。由于其高效性和灵活性,LightGBM已经成为数据科学家和机器学习工程师的热门选择,是现代机器学习工具箱中不可或缺的一部分。
	在城市土地利用研究中,Wu 等(2020)应用LightGBM模型对中国上海市的城市扩展进行了研究。该研究利用遥感影像和城市社会经济数据,通过LightGBM算法预测了城市扩展的趋势。研究结果表明,LightGBM能够较好地处理空间分布不均的数据,准确预测了上海市在未来几十年的城市扩张情况,且在模型训练和预测过程中,相较于传统算法,具有更快的训练速度和更低的内存占用。这使得LightGBM成为高效处理大规模城市土地利用变化监测的有力工具。
	
	在环境监测领域,Liang 等(2018)利用LightGBM进行了气候变化对植被生长影响的研究。通过结合遥感影像、气候数据和土壤特征,使用LightGBM对植被生长状况进行了预测。研究结果显示,LightGBM能够有效整合多源数据,准确捕捉气候变化对植被生长的影响,特别是在多层次、非线性的关系建模上展现了强大的能力。这项研究不仅在植被生长监测中取得了良好的效果,还展示了LightGBM在生态环境研究中的潜力。
	
	在生态灾害监测中,Zhang 等(2019)应用LightGBM对森林火灾发生的风险进行了预测。研究结合遥感数据、气象数据和地理信息,通过LightGBM建立了火灾风险预测模型。研究结果表明,LightGBM在处理火灾风险评估中的表现优于传统模型,特别是在高维度特征和数据稀疏的情况下,其预测精度和计算效率得到了显著提高。
	
	\subsubsection{TensorFlow}
	
	TensorFlow 是一个广泛使用的开源深度学习框架,由Google Brain团队于2015年发布,具有高度的灵活性和可扩展性,适合开发各种规模和复杂度的机器学习模型。它的优势在于支持分布式计算,能够在多个 GPU 和 TPU 上高效并行处理大规模数据,从而加速训练过程。
	它设计用来简化各种机器学习任务的实现,包括神经网络的构建、训练和部署。TensorFlow的核心功能是提供一个灵活且高效的计算图(computational graph)模型,支持大规模数据流的并行计算,特别适合处理复杂的数学和深度学习问题。。此外,TensorFlow 提供了丰富的 API 和工具集,涵盖从简单的机器学习模型到复杂的神经网络架构,满足研究人员和开发者的不同需求。它的生态系统庞大,包括 TensorBoard 等工具,用于可视化和调试,帮助用户更好地理解和优化模型。
		
	TensorFlow的优势在于它的可扩展性和平台兼容性。无论是在单台机器、分布式环境还是云端,TensorFlow都能高效运行,并且支持不同的硬件平台,包括CPU、GPU和TPU(Tensor Processing Unit)。这种灵活性使得TensorFlow成为处理大规模机器学习任务的理想工具,尤其是在涉及大量数据和复杂模型时。
	
	TensorFlow支持多种机器学习和深度学习算法,诸如神经网络、卷积神经网络(CNN)、循环神经网络(RNN)等。它不仅适用于图像处理、自然语言处理、语音识别等任务,还可以扩展到强化学习、生成对抗网络(GANs)等更复杂的应用场景。
	
	TensorFlow的设计是高度模块化的,提供了多个层次的抽象,使得开发者可以根据需求选择合适的操作层级。从底层的低级API(如TensorFlow Core)到更高级的API(如Keras),都可以轻松使用。Keras是TensorFlow官方推荐的高级API,它简化了深度学习模型的构建、训练和评估过程,使得即使是新手也能轻松上手。
	
	TensorFlow不仅适用于研究和开发人员,还广泛应用于生产环境,支持模型的部署和推理。TensorFlow Serving是一个专门为生产环境优化的模型服务框架,它能够高效地部署和管理机器学习模型。TensorFlow Lite则专注于移动设备和嵌入式系统上的推理任务,TensorFlow.js使得模型可以直接在浏览器中运行,适合开发Web应用。
	

	然而,TensorFlow 也有一些劣势。首先,它相较于某些框架(如 PyTorch)来说,学习曲线较陡,特别是对于新手而言,编写和调试代码可能较为复杂。虽然 TensorFlow 2.x 改进了许多 API 的易用性,但其底层机制仍然偏底层,初学者可能会感到难以驾驭。其次,尽管 TensorFlow 在性能上表现强劲,但由于其高度复杂性和庞大结构,部署和调优模型可能需要更多时间和计算资源。另外,TensorFlow 的灵活性有时反而会带来问题,当不需要大规模并行计算时,其复杂性和资源占用可能显得过度。
	
	总的来说,TensorFlow 是一个功能强大且适用于各种机器学习任务的框架,特别适合大规模分布式训练和深度学习模型的开发,但在易用性和调试方面需要较高的技术门槛。
	在遥感影像分析方面,Li 等(2019)使用 TensorFlow 进行了土地覆盖变化检测研究。研究通过深度卷积神经网络(CNN)模型来处理遥感影像数据,并结合多时间段的影像信息进行土地覆盖分类。使用 TensorFlow,研究团队能够有效地训练大型神经网络模型,并且在多个区域和时间段的遥感数据集上,显著提高了分类精度。这项研究表明,TensorFlow 在遥感数据处理中不仅提高了效率,而且在模型训练的过程中能够处理大规模的图像数据,进一步提高了土地利用/覆盖分类的精度。
	
	在气候变化研究中,Wang 等(2020)利用 TensorFlow 进行全球气候变化对植被覆盖变化的预测。通过构建循环神经网络(RNN)模型,研究团队利用TensorFlow对不同区域的植被变化进行时序预测,研究了气候因素(如温度、降水量)对植被变化的影响。TensorFlow 在训练过程中展示了其处理长时间序列数据的能力,使得模型能够准确捕捉到气候变化对植被的长期影响。这项研究进一步证明了深度学习框架在处理复杂时空数据时的强大能力,TensorFlow 在大规模气候数据分析中的应用前景广泛。
	
	此外,TensorFlow 还被广泛应用于城市土地利用变化研究。Zhao 等(2021)使用 TensorFlow 构建了一个深度生成对抗网络(GAN)模型,旨在对城市扩展进行预测。通过输入大量的遥感影像和城市社会经济数据,TensorFlow 帮助研究团队建立了一个多层次、多尺度的模型,成功预测了未来几十年内的城市扩展情况。该研究表明,TensorFlow 在大数据量的城市研究中具有优越的性能,尤其是在模拟复杂城市发展过程中的应用潜力。
	
	在生态监测领域,Chen 等(2018)使用 TensorFlow 构建了一个深度学习模型来识别和监测森林火灾的风险。他们利用了遥感影像数据、气象数据和地理信息,训练了一个深度神经网络来预测火灾的发生概率。研究表明,TensorFlow 能够处理来自不同数据源的大量信息,并且在提高火灾预测准确性方面表现出色。此外,TensorFlow 的高度优化和分布式计算能力,使得大规模数据处理和模型训练变得更加高效。
	
	TensorFlow 还在海洋环境监测和物种识别方面有所应用。Zhang 等(2020)通过 TensorFlow 利用深度卷积神经网络对海洋生态环境中的物种进行自动识别。这项研究展示了 TensorFlow 在处理海洋遥感数据和图像分析中的能力,尤其是在实时监测和生态环境保护领域的应用。TensorFlow 的高效训练和预测能力使得这类任务的执行速度和准确性都有了显著提升。
	
	\subsection{VIF}
	
	\[
	\text{VIF}_i = \frac{1}{1 - R_i^2}
	\]
	方差膨胀因子 (VIF) 是用于检测多重共线性的统计指标,衡量一个特征与其他特征之间的线性相关性。具体来说,VIF 反映了一个特征可以通过其他特征多大程度上被解释或预测。较高的 VIF 值(通常大于 10)表明该特征与其他特征高度相关,可能导致模型不稳定,影响系数的估计精度。通过计算 VIF,可以识别和去除冗余特征,从而提高模型的解释性和预测性能。其中,$R_i^2$ 是回归模型中,将第 $i$ 个自变量对其他自变量进行线性回归时,得到的判定系数。该公式表示第 $i$ 个自变量与其他自变量的相关性程度。较高的 $R_i^2$ 表明第 $i$ 个自变量与其他自变量高度相关,从而导致较大的 VIF 值,这表明存在多重共线性问题。Grewal等人(2004)在结构方程模型的研究中,详细探讨了多重共线性与测量误差对理论检验的影响,强调了使用VIF诊断多重共线性的必要性。这项研究揭示了高VIF值可能导致模型中的系数估计不可靠,进而影响理论模型的稳健性。此外,O’Brien(2007)针对VIF值的使用提出了警示,他认为单纯依赖既定的VIF阈值(如10或5)来判定多重共线性并不总是合适的,建议研究者应根据具体情境进行更全面的分析。
	
	Hoerl与Kennard(1970)在提出岭回归时,进一步解释了如何在面对多重共线性时通过调整估计方法来降低VIF对模型的不利影响。他们的研究为处理具有共线性的回归问题提供了另一种有效途径,尤其适用于高维数据中的预测问题。此外,Liao和Valliant(2012)在复杂调查数据的背景下应用VIF来分析数据的多重共线性,展示了该指标在复杂数据集分析中的重要性。
	
	近年来,随着大数据和机器学习的发展,研究者如Lin等(2011)提出了一种基于VIF的快速回归算法,能够在处理大规模数据时有效识别和解决多重共线性问题。Lipovetsky和Conklin(2001)则从多目标回归的角度出发,探讨了在多重共线性存在的情况下,如何通过调整回归模型的结构来减小VIF的影响。这些研究充分展示了VIF在不同研究领域中的重要性,特别是在优化回归模型、提高预测精度和稳定性方面起到了关键作用。
	
	\section{第三章 预处理}
	
	
	
	\subsection{收集WorldClim数据}
	
	WorldClim数据集气候数据通常通过全球气象站点的观测数据进行插值,生成高分辨率的气候图层。这些数据广泛用于生态和环境科学研究,如物种分布模型、气候变化影响评估等。高分辨率气候图层能够提供详细的区域气候信息,使得研究人员可以在较小的地理尺度上进行精细化分析,从而获得更精确的结果。这种空间分布图对于理解气候模式和趋势非常有用。例如,通过观察太阳辐射的季节性变化,可以分析不同区域的太阳能潜力,进而影响能源政策的制定。同样,通过分析气温的空间分布,可以了解温度的季节性变化,帮助农业、生态和城市规划等领域进行相应调整。
	总体来看,这些气候变量的空间分布图不仅展示了不同时间点和气候变量的空间变化,还为进一步的气候研究提供了宝贵的数据支持。通过这些图,可以更直观地理解和分析气候变化对不同区域的影响,为相关领域的决策提供科学依据。 这张图展示了某一地理区域的多个气候变量的空间分布情况。图像中包含了2个不同的气候变量,每个变量分别在一个子图中呈现,通过观察图像,我们可以发现青藏高原的草地具有较高的海拔和较低的年最高温等。结合WorldClim数据集(可参见WorldClim官方网站),这些变量反映了不同的气候特征。通过颜色的渐变可以看出不同变量在不同区域的变化情况。这些数据有助于理解该区域的气候特征,并可用于气候研究、生态模型以及环境管理等领域
	
	\subsection{收集土壤数据}
	协调世界土壤数据库 (HWSD)是 FAO(联合国粮食及农业组织)和 IIASA(国际应用系统分析研究所)以及其他合作伙伴(如 ISRIC – 世界土壤信息和欧洲土壤局网络 (ESBN) 的合作成果。
	
	在全面更新全球农业生态区研究的背景下,联合国粮食及农业组织(FAO)和国际应用系统分析研究所(IIASA)认识到,迫切需要整合全球现有的区域和国家土壤信息更新,并将其与1971-1981年间编制的、但大部分已不再反映当前土壤资源实际状况的1:5,000,000比例尺FAO-UNESCO世界土壤图相结合。为此,他们与主要负责开发区域土壤和地形数据库(SOTER)的国际土壤参考资料和信息中心(ISRIC)以及近年来对欧洲和欧亚北部土壤信息进行重大更新的欧洲土壤局网络(ESBN)建立了合作伙伴关系。此外,通过与中国科学院土壤科学研究所的合作,将1:1,000,000比例尺的中国土壤图纳入其中,成为重要补充。
	
	为了以统一的方式估算土壤属性,研究团队利用实际土壤剖面数据和土壤传递规则的开发,与ISRIC和ESBN合作,借鉴了WISE土壤剖面数据库以及Batjes等人(1997; 2002)和Van Ranst等人(1995)的早期工作。国际应用系统分析研究所(IIASA)负责确保数据的和谐化和在地理信息系统(GIS)中的输入,而所有合作伙伴则负责数据库的验证。
	
	该产品的主要目的是为模型构建者提供实用工具,并为农业生态区划、粮食安全和气候变化影响等前瞻性研究服务。因此,选择了大约1公里(30弧秒×30弧秒)的分辨率。生成的栅格数据库由21,600行和43,200列组成,其中包含2.21亿个网格单元,覆盖了全球陆地。
	
	在协调一致的全球土壤数据库(HWSD)中,识别出超过16,000个不同的土壤制图单元,这些单元与协调一致的属性数据相关联。标准化的结构允许将属性数据与GIS相结合,以显示或查询土壤单元的组成以及选定土壤参数的特征(如有机碳、pH值、蓄水能力、土壤深度、阳离子交换能力、粘土含量、总可交换养分、石灰和石膏含量、钠交换百分比、盐度、质地类别和粒度分布)。
		
	\subsection{使用Arcgis Pro绘制高程图}

使用arcgispro 可以很方便的制作青海,甘肃,新疆等省市的地形图。
绘制地形图具有重要的价值,体现在多个方面。首先,地形图为地形分析提供了关键数据,展示了地表高程、坡度和坡向等特征,帮助研究人员和规划者深入理解土地使用模式、排水系统及土壤侵蚀情况。此外,地形图在环境管理中起着至关重要的作用,它使政府和环境机构能够更有效地进行自然资源保护和生态恢复,识别敏感区域及潜在的环境风险。

在城市规划和基础设施建设方面,地形图为规划师提供了必要的信息,使他们能够合理设计城市布局、交通系统和基础设施,确保建筑位置和道路设计的合理性和安全性。对于农业从业者而言,地形图则帮助优化土地利用和农业生产,通过分析地形特征,合理选择作物种植和水资源管理策略。

地形图在自然灾害管理中同样具有重要价值,能够识别洪水、滑坡和地震等自然灾害的风险区,为应急响应计划提供支持,从而降低灾害带来的损失。与此同时,地形图为户外活动爱好者提供了丰富的信息,支持他们进行徒步旅行、滑雪和山地骑行等活动,使游客能够更好地选择适合的路线和目的地。

此外,地形图在科学研究与教育中发挥着重要作用,作为地理教育和研究项目的基础工具,帮助学生和研究者更好地理解地球表面的变化及特征。最后,地形图为政府、企业和组织提供决策支持,成为制定政策和战略的重要依据,使各方能够评估项目的可行性和潜在影响。

		\subsubsection{操作步骤}
		在 ArcGIS Pro 中绘制数字高程模型(DEM)图的步骤如下:
		
		首先,确保你已经加载了包含高程数据的栅格数据集。打开 ArcGIS Pro,创建一个新的项目或打开一个现有项目,并在“地图”视图中添加你的 DEM 数据。你可以通过在“内容”面板中右键单击并选择“添加数据”来加载数据,或者直接将数据拖入地图中。
		
		一旦 DEM 数据加载完成,你可以使用“符号化”工具来调整其外观。在“内容”面板中,右键单击 DEM 图层,选择“符号化”以打开符号化窗口。这里,你可以选择不同的颜色渐变和符号样式来表示高程值。通常,使用颜色渐变(如从蓝色到红色)可以有效展示地形起伏。设置好后,点击“应用”以查看效果。
		
		为了增强地图的可读性,你可能需要添加图例、比例尺和方向指示器。这些元素可以在“插入”选项卡中找到,选择相应的图例、比例尺或指北针,并将其放置在布局视图中适当的位置。调整它们的大小和位置,以确保它们不会遮挡主要内容。
		
		接下来,使用“注记”功能为 DEM 图添加标签。在“注记”选项卡中,选择“创建注记”以为特定区域或特征添加标签,提供额外的信息或地理标识。设置好注记的字体和样式,以确保信息清晰可读。
		
		完成所有设置后,进入布局视图以准备导出最终图像。在布局视图中,你可以添加标题、说明文本以及其他装饰性元素,以增强地图的视觉效果。确保所有元素均整齐排列,并检查整体布局的平衡性。
		
		最后,使用“分享”选项卡导出地图。在此处,你可以选择导出为不同格式的文件,如 PDF 或图像文件,选择相应的分辨率和输出选项,然后点击“导出”以保存你的 DEM 图。
		
		通过以上步骤,你将能够在 ArcGIS Pro 中创建和导出一张专业的数字高程模型图,清晰展示地形信息。
		
		
		
				\subsubsection{结果}


			\includegraphics[width=0.8\textwidth, keepaspectratio]{pic/gansu_dem3.png} % 自动保持纵横比

	\subsection{变量筛选工作}
	\begin{tabular}{lr}
		\toprule
		Feature & VIF \\
		\midrule
		aspect & 3.485931 \\
		bio14 & 9.150778 \\
		cfvo\_0\_5cm & 7.172799 \\
		evi & 7.093521 \\
		prec\_01 & 4.498554 \\
		slope & 3.292866 \\
		soc\_0\_5cm & 7.690032 \\
		tmax\_11 & 1.755009 \\
		tmin\_05 & 2.150528 \\
		\bottomrule
	\end{tabular}
	
	这个表格显示了不同特征变量及其方差膨胀因子 (VIF) 的值,用于评估特征之间的多重共线性。方差膨胀因子越高,表示该特征与其他特征的相关性越强,可能会影响模型的稳定性。表中的特征包括地形的斜率 (slope)、最大和最小气温 (tmax\_11 和 tmin\_05)、植被指数 (evi)、降水量 (prec\_01)、土壤有机碳含量 (soc\_0\_5cm)、土壤岩石含量 (cfvo\_0\_5cm) 等,其中 bio14 的 VIF 值最高,达到了 9.150778,表明该变量与其他变量有较强的共线性,而 tmax\_11 的 VIF 值最低,仅为 1.755009,显示出较低的共线性风险。
	
	
	\subsection{回归模型的R方查看}
\begin{table}[h]
	\centering
	\begin{tabular}{@{}ll@{}}
		\toprule
		\textbf{自变量} & \textbf{R² 值} \\ \midrule
		evi & 0.2376 \\ 
		nitrogen\_0\_5cm & 0.2224 \\ 
		lon & 0.2203 \\ 
		ndvi & 0.2134 \\ 
		tmax\_02 & 0.2127 \\ 
		bio09 & 0.2064 \\ 
		tmax\_01 & 0.2051 \\ 
		tavg\_02 & 0.2045 \\ 
		prec\_07 & 0.2021 \\ 
		\bottomrule
	\end{tabular}
	\caption{R² 值大于 0.2 的自变量}
	
这些自变量在模型中对 PL 具有一定的解释力。其中,EVI(增强植被指数)的 R² 值最高,为 0.2376,说明其对 PL 的解释能力最强;而氮含量(0-5 cm 深度)的 R² 值为 0.2224,紧随其后。经度(lon)和归一化植被指数(NDVI)也显示出良好的解释能力,R² 值分别为 0.2203 和 0.2134。其他变量如 TMAX\_02、BIO\_09、TMAX\_01、TAVG\_02 和 PREC\_07 的 R² 值在 0.2 到 0.21 之间,表明它们同样对 PL 的变动有一定的贡献。这些结果为后续分析和模型优化提供了重要的依据。
\end{table}

	\subsection{植物病害的基础信息的掌握}
	在中国草地的实地调查中,共收集了244个样点的植物病害数据,每个样点包含4个样方,共计976条记录。每条数据详细记录了样点的经纬度信息以及植物病害的严重程度,这为后续的分析和研究提供了宝贵的基础数据。
	对于四川省,植物病害较严重的采样点主要集中在四川省西北部的高原地区以及靠近南部边界的部分区域,而在中部、东部和靠近成都的地区,植物病害相对较轻或没有明显表现。对于内蒙古自治区,植物病害较严重的采样点主要集中在内蒙古东北部,内蒙古南部和中部区域这些区域的植物病害较轻或不存在病害。
	对于西藏自治区,山南市的病害严重性最为突出,特别是靠近拉萨市的采样点。阿里地区也显示出一些病害严重的区域。中部和北部地区,如那曲市附近的地区植物病害较轻微或不存在。对于甘肃省,植物病害较严重的采样点主要集中在甘肃南部靠近四川、青海交界的地区,甘肃中部靠近兰州地区的植物病害较低。对于青海省,植物病害较严重的采样点主要集中在青海南部和东南部,尤其是在玉树州和附近的区域,病害严重程度较高,而青海北部和中部的采样点植物病害较轻。
	
	在样点图中,每个点(样点)代表一个具体的测量位置,而点的大小、颜色或形状可能与该位置的数据值有关。通过这种视觉表示方法,可以快速识别出数据的分布模式和趋势,比如哪些区域的数值较高,哪些区域的数值较低。
	\par
	\includegraphics[width=0.8\textwidth, keepaspectratio]{pic/sample.png} % 自动保持纵横比
	\section{第四章 数据分析}
	\subsection{4.1 随机森林变量重要性图输出与分析}
	为了完成随机森林的构建,可以使用 Python 的 scikit-learn 库来实现。图表中的数据展示了随机森林模型分析的变量重要性,这些变量包括与太阳辐射、水汽压、降水量、土壤成分和生物因素等相关的指标。太阳辐射)可能根据不同的月份或数据集版本变化,对植物生长和病害发生具有显著影响。
	7月份的水汽压和降水量分别反映了空气中的水分含量和降水情况,这些因素直接关系到植物的水分吸收和病害的潜在传播。土壤中的粘土含量(和阳离子交换容量共同影响土壤的排水性、通气性以及养分保持能力,进而可能影响植物的生长环境和抗病性。土壤pH值在水分条件下的测量结果,关系到植物对养分的吸收和土壤微生物的活动,对病害的发生有间接影响。生物因素可能涉及土壤微生物群落,它们对植物病害可能有直接或间接的作用。图表中列出的数值,如0.00到0.20,代表了这些变量在随机森林模型中的重要性评分,评分越高,表明该变量对预测植物病害结果的影响越显著。这些数据为理解植物病害与环境及土壤条件之间的关系提供了重要线索。
	
	\section{第五章 可视化}
	
	\subsection{展示样点信息}

	
	
	\subsection{绘制变量重要性图}
	
	图中的特征按重要性排序,LON、srad\_10 等是对模型影响最大的特征。与土壤因素相比,气候因素对分解速率的影响较大。气候变量之间有较强的相关性,气候变量与经度具有较强的相关性。

		
	\subsection{模型比较}
	通过比较神经网络,LightGBM等模型,可以发现LightGBM训练速度最快,神经网络拟合效果最好。
	
	\subsection{预测结果}
	
	
	通过随机森林预测,内蒙古东北北部,四川北部地区存在较高的植物病害,西藏的大部分地区,内蒙古中部存在较低的植物病害,内蒙古中部的植物病害会随着气候变化而增加,青海南部的植物病害会随着气候变化而降低。
		
	随着年均温的增加,植物病害有下降的趋势,随着年降水的增加,植物病害有上升的趋势。
	根据shap图中的特征按重要性排序,LON、LAT、srad\_07 等是对模型影响最大的特征。实际植物病害较低的预测结果比较准确。实际植物病害较高的预测值与实际值偏离较大。
	

\end{document}
